\documentclass{article}
\usepackage[left=-1.5mm, right=-1.5mm, top=-0.5mm, bottom=-1.5mm]{geometry}
\usepackage[document]{ragged2e}
\usepackage{graphicx}
\usepackage{amsmath, mathtools, nccmath, amssymb}
\usepackage[dvipsnames]{xcolor}
\usepackage{array, makecell}
\usepackage{moresize}
\usepackage[T1]{fontenc}
\usepackage{noto-sans}
\renewcommand\cellalign{cl}
\renewcommand\theadalign{lc}
\newcommand{\ns}{\\ \vspace{-0.03in}}
\newcommand{\mc}[1]{\makecell[cl]{\vspace{-0.065in} \\ #1}}
\newcommand{\mcc}{\makecell[{{c}}]}
\newcommand{\mcr}{\makecell[{{r}}]}
\newcommand{\pic}{\includegraphics[scale=0.3]}
\newcommand{\form}[1]{\vspace{-0.03in}\\ \small{\textcolor{Red}{\(#1\)}} \\ \vspace{-0.03in}}
\newcommand{\forms}[1]{\vspace{-0.01in}\\ \scriptsize{\textcolor{Red}{\(#1\)}\\ \vspace{-0.01in}}}
\newcommand{\noskipform}[1]{\small{\textcolor{Red}{\(#1\)}}}
\newcommand{\fpic}{\vspace{-0.05in} \\ \includegraphics[scale=0.3]}
\newcommand{\dx}[1]{\scriptsize{\textcolor{Red}{\(\dfrac{d}{dx}#1\)}}}
\newcommand{\dxs}[1]{\ssmall{\(\dfrac{d}{dx}#1\)}}
\newcommand{\dxo}{\dfrac{d}{dx}}
\renewcommand{\b}{\textbf}
\renewcommand{\r}[1]{\textcolor{Red}{#1}}
\newcommand{\lims}[1]{\lim\limits_{#1}}
\newcommand{\atb}[1]{\int_a^b #1 \,dx}
%\renewcommand{\arraystretch}{1.5}
\graphicspath{{./Pictures/}}
\begin{document}
    \begin{table}[ht!]
        \begin{tiny}
          \begin{tabular}{ccc}
          \begin{tabular}[t]{|@{\hskip1pt}p{10cm}|}
              \hline
              \mc{\footnotesize{Derivation Rules} \\
              \dx{(x^a) = a * x^{a-1} \text{ given: } x,a \in \mathbb{R} \text{ \& } x > 0}\\\\
              \ssmall{subexamples:}\\
              \dxs{x = 1 \to \dxo(x^1) = 1 * x^{1-1}}\\
              \dxs{x^2 = 2x \to \dxo(x^2) = 2 * x^{2-1}}\\
              \dxs{\dfrac{1}{x} = -\dfrac{1}{x^2} \to \dxo(x^{-1} = -1 * x^{-1-1})}\\
              \dxs{\sqrt{x} = \frac{1}{2 * \sqrt{x}} \to \dxo(x^{\frac{1}{2}}) = \frac{1}{2 * x^{\frac{1}{2}}}}\\
              \dx{(c) = 0 \text{ given: } c \in \mathbb{R} \text{ \& c is constant \& \r{c != factor}} }\\\\
              \dx{(e^x) = e^x \to \dxo(e^x) = ln(e) * e^x * x' = 1 * 1 * e^x}\\\\
              \dx{(a^x) = ln(a) * a^x \to \dxo(a^x) = x' * ln(x) * a^x \text{ because: } e^{x * ln(a)} = a^x }\\\\
              \ssmall{Note for this rule:}\\
              \ssmall{\( \dxo(2^{2x + 1}) = ln(2x + 1) * 2^{2x + 1} * (2x + 1)' = ln(2x + 1) * 2^{2x + 1} * 2  \)}\\\\
              \dx{(ln(x)) = \frac{1}{x}}\\\\
              \dx{log_b(x) = \frac{1}{ln(b) * x } \to \text{special case for} \dxo(\frac{ln(x)}{ln(b)})}\\\\
              \ssmall{this is the case because of base change in lograrithmic functions!}\\
              \ssmall{\( log_a(x) = \dfrac{ln(x)}{ln(a)} = \dfrac{log_c(x)}{log_c(a)} \to \dxo log_a(x) = \dfrac{ln(x)}{ln(a)} -> \dxo ln(x) = \dfrac{\dfrac{1}{x}}{ln(a)} = \dfrac{1}{ln(a) * x}\)}\\
              \ssmall{c can be any number!}\\\\
              \dx{sin(x) = cos(x)} \hspace{0.5in} \dx{csc(x) = (-1) * csc(x) * cot(x)}\\\\
              \dx{cos(x) = -sin(x)} \hspace{0.5in} \dx{sec(x) = sec(x) * tan(x) = }\\\\
              \dx{tan(x) = \frac{1}{cos^2(x)}} \hspace{0.5in} \dx{cot(x) = (-1) * csc^2(x) }\\\\
              \scriptsize{\r{Note, sin(x) is done via chain rule -> sin(x) * 1 !!}}\\
              \scriptsize{\r{For more than x inside sin, FULL CHAIN RULE!! \(sin(x^2) -> sin(x^2) * x\)}}\\
              \scriptsize{\r{For integration, you need the substitution rule!}}\\
              \dx{tan(x) = 1 + tan^2(x)}\\\\
              \dx{(ax) = a \to \dxo(ax) = a * 1 \to \text{ we derive x NOT a!!}}\\\
              \dx{(3x) = a \to \dxo(3x) = 3 * 1 \to \text{ 3 is a factor!}}\\\\
              }\\
              \hline
              \mc{\footnotesize{All of these derive from:} \\
                \form{f'(a) = \lims{h\to0} \frac{f(a+h) - f(a)}{h}}\\
              }\\
              \hline
              \mc{\footnotesize{Sum Rule} \\
              \form{(f + g)' = f' + g'}\\
              }\\
              \hline
              \mc{\footnotesize{Difference Rule} \\ 
              \form{(f - g)' = f' - g'}\\
              }\\
              \hline
              \mc{\footnotesize{Product Rule} \\ 
              \form{(f * g)' = f * g' + f' * g }\\
              }\\
              \hline
              \mc{\footnotesize{Quotient Rule} \\ 
              \form{\left(\dfrac{f(x)}{g(x)}\right)' =\dfrac{f' * g - f * g' }{g^2}}\\
              }\\
              \hline
              \mc{\footnotesize{Chain Rule} \\ 
              \form{[f(g(x))]' = f'(g(x)) * g'(x)}\\
              \scriptsize{Example:}\\
              \form{\dxo\sqrt{x^2 + 1} = (x^2 + 1)^{\frac{1}{2}} = \frac{1}{2} * (x^2 + 1)^{-\frac{1}{2}} * 2x}
              \form{\frac{1}{2} * \frac{1}{\sqrt{x^2 + 1}} * 2x = \frac{2x}{2 * \sqrt{x^2 + 1}} = \frac{x}{\sqrt{x^2 + 1}}}\\
              \scriptsize{Note: \((x^2 + 1)\) is g(x), while f is the exponent function}\\
              \scriptsize{More examples:}\\
              \form{\dxo sin^2(x) = \dxo(sin(x))^2 = 2 * sin(x) * cos(x)}\\
              \form{\dxo \left(\dfrac{x - 1}{x + 1}\right)^2 = 2 * \left(\dfrac{x - 1}{x + 1}\right) * \left(\dfrac{x - 1}{x + 1}\right)'}\\
              \form{\dxo (x + 2)^3(x)^4 = (x + 2)^3 * ((x)^4)' + ((x + 2)^3)' * (x)^4 }
              \form{((x + 2)^3)' = 3 * (x + 2)^2 * (x + 2)' = 3 * (x + 2)^2 * 1}\\
              \form{\dxo sin(cos[tan(x)]) = cos(cos[tan(x)]) * -sin(tan(x)) * \dfrac{1}{cos^2(x)} }
              }\\ 
              \hline
            \end{tabular}
            \hspace{-0.07in}
            \begin{tabular}[t]{|@{\hskip1pt}p{10.90cm}|}
              \hline
              \mc{\footnotesize{Implicit Differentiation} \\ 
                \dx{(x^2 + y^2 = 9) \to 2x + \dfrac{d}{dx}((y)^2) * \dfrac{dy}{dx}(y) = 0 \to 2x + (2y * y') = 0 \to y' = \dfrac{-2x}{2y}}\\\\
                \scriptsize{!! Remember that this is only necessary if y needs to be derived !!}\\
              }\\
              \hline
              \mc{\footnotesize{Higher Derivatives} \\ 
              \scriptsize{The best idea for higher derivatives is distance s, velocity v and acceleration a.}\\
              \scriptsize{\(\dfrac{d}{dt}(s(t)) = v(t) = s'(t) \text{ || } \dfrac{d}{dt}(v(t)) = a(t) = s''(t) = v'(t) \)}\\\\
              \scriptsize{\r{This is why the acceleration on earth -> gravity is constant!! HOLY FUCK}}
              }\\
              \hline
              \mc{\footnotesize{Taking Derivations higher than 3} \\ 
              \scriptsize{\(1:f' \to 2:f'' \to 3:f''' \to \r{4:f^{(4)}} \to \r{n:f^{(n)}}\)}}\\\\
              \hline
              \mc{\footnotesize{Related Rates}\\
              \scriptsize{In a Sphere, the rate of change of V is \(100cm^3/s\)}\\
              \scriptsize{calculate the rate of change in r = 25cm given rate of change in V}\\
              \scriptsize{\(\dfrac{dV}{dt} = 100cm^3/s\) , \(r=25cm\)}\\
              \form{\dfrac{dV}{dt} = (4 * \pi * r^2) * \dfrac{dr}{dt} \to \dfrac{dr}{dt} = \dfrac{100cm^2/s}{4 * \pi * (25cm)^2} = \dfrac{dr}{dt} =  \dfrac{1}{25 * \pi}cm/s}\\
              }\\
              \hline
              \mc{\footnotesize{Local Maximum and Minimum}\\
              \scriptsize{The \r{first} derivative of local maximum and minimum MUST be 0!}\\
              \scriptsize{This includes turning points, aka slope is just 0!}\\\\
              \scriptsize{Local Minimum \r{lMin}:}\\
              \form{f''(x) > 0 \to \text{ given f(x) = 0}}\\
              \scriptsize{OR}\\
              \form{f'(lMin - 1) < 0 \text{ \&\& } f'(lMin) = 0 \text{ \&\& } f'(lMin + 1) \geq 0}\\
              \scriptsize{Essentially the minimum is where the slope goes from negative to positive}\\
              \scriptsize{with the turning point being the minimum with slope 0}\\\\
              \scriptsize{Local Maximum \r{lMax}:}\\
              \form{f''(x) < 0 \to \text{ given f(x) = 0}}\\
              \scriptsize{OR}\\
              \form{f'(lMax - 1) \geq 0 \text{ \&\& } f'(lMax) = 0 \text{ \&\& } f'(lMax + 1) < 0}\\
              \scriptsize{Essentially the maximum is where the slope goes from positive to negative}\\
              \scriptsize{with the turning point being the maximum with slope 0}\\\\
              \scriptsize{Absolute minimum and maximum will never be exceeded -> sine absolute-max = 1}\\
              }\\
              \hline
              \mc{\footnotesize{Inflection Point}\\
              \scriptsize{This is the point where the function stops its increase or decrease in slope.}\\
              \scriptsize{Therefore it is the second derivative and is equal to 0}\\
              \pic{220615-1}\\
              }\\
              \hline
              \mc{\footnotesize{Use of Maxima}\\
              \scriptsize{Building a fence adjacent to a river. length l = 2x + y!}\\
              \scriptsize{Given length of 2400m how big do x and y need to be for the maximum area A?}\\
              \form{l = y + 2x \to y = 2400m - 2x \to A = (2400m - 2x) * x }\\
              \scriptsize{Remember when you had to use the UI function on the calculator? Yeah, no more!}\\
              \form{Max \to f'(A) = f'((244m - 2x) * x = 0 \to x = \dfrac{2400m}{4} = 600m \to y = 1200m}
              }\\
              \hline
              \mc{\footnotesize{Limits:}\\
              \scriptsize{The limit expresses that a variable is approaching a value}\\
              \form{\lims{x\to\infty} \text{x approaching infinity}}\\
              \scriptsize{This is often used when trying to determine functions that might give an invalid result at x}\\
              \scriptsize{\(f(x)=\dfrac{x - 1}{x^2 - 1} \to f(1)=?? \)}\\
              \scriptsize{With limit we can say what we would expect the value to be, if the function would continue as before}\\
              \scriptsize{aka what is the value of f(1) if the function would not show this abnormality?}\\
              \scriptsize{\(\lims{x\to1}f(1)=0.5\)}\\\\
              \scriptsize{This also applies to functions that go to infinity, or functions that are constant for a range.}\\
              \pic{220615-2} \\ 
              }\\
              \hline
            \end{tabular}
          \end{tabular}
          \hfill
        \end{tiny}
    \end{table}
    \pagebreak
    \begin{table}[ht!]
        \begin{tiny}
          \begin{tabular}{ccc}
            \begin{tabular}[t]{|@{\hskip1pt}p{10cm}|}
              \hline
              \mc{\pic{220615-3}\\
              \scriptsize{Limit Rules}\\
              \scriptsize{Addition:}\\
              \form{\lims{x\to a}[f(x) + g(x)] = \lims{x\to a} f(x) + \lims{x\to a} g(x)}\\
              \scriptsize{Subtraction:}\\
              \form{\lims{x\to a}[f(x) - g(x)] = \lims{x\to a} f(x) - \lims{x\to a} g(x)}\\
              \scriptsize{Multiplication:}\\
              \form{\lims{x\to a}[f(x) * g(x)] = \lims{x\to a} f(x) * \lims{x\to a} g(x)}\\
              \scriptsize{Division:}\\
              \form{\lims{x\to a}\left[\dfrac{f(x)}{g(x)}\right] = \dfrac{\lims{x\to a} f(x)}{\lims{x\to a} g(x)} \to \text{ given lim != 0}}\\
              \scriptsize{Multiplication by constant:}\\
              \form{\lims{x\to a}[c * f(x)] = c * \lims{x\to a}f(x) \to \text{ given c is constant} }\\
              \scriptsize{Exponent:}\\
              \form{\lims{x\to a}[f(x)]^2 = [\lims{x\to a} f(x)]^2}\\
              \scriptsize{Root:}\\
              \form{\lims{x\to a}\sqrt[n]{[f(x)]} = \sqrt[n]{\lims{x\to a} f(x)}}\\
              \scriptsize{x to a:}\\
              \form{\lims{x\to a}(x) = a}\\
              \scriptsize{x to a with exponent:}\\
              \form{\lims{x\to a}(x^n) = a^n}\\
              \scriptsize{x to a with root:}\\
              \form{\lims{x\to a}(\sqrt[n]{x}) = \sqrt[n]{a}}\\
              \scriptsize{limit of a constant:}\\
              \form{\lims{x\to a}(c) = c \to \text{ given c is constant}}\\\\
              \scriptsize{Examples:}\\
              \form{\lims{x\to -2}\left(\dfrac{x^3 + 2x^2 - 1}{5 - 3x}\right) = \dfrac{\lims{x\to -2}(x^3 + 2x^2 - 1)}{\lims{x\to -2}(5 -3x)}}
              \form{\dfrac{\lims{x\to -2}(x^3) + \lims{x\to -2}
              (2x^2) - \lims{x\to -2}(1)}{\lims{x\to -2}(5) - \lims{x\to -2}(3x)}  = \dfrac{-8 + 8 - 1}{5 + 6} = -\dfrac{1}{11}}\\
              \scriptsize{Sometimes we need to eliminate terms in order to move on}\\\\
              \scriptsize{limit and differntiation -> \r{L'Hospital's Rule:}}\\\\
              \scriptsize{\r{If either the left side -> \(\dfrac{f(x)}{g(x)}\) is indeterminate}}\\
              \scriptsize{\r{then we can use this rule! Otherwise it doesn't work, and doesn't make sense!}}\\\\
              \form{\lims{x\to a}\left(\dfrac{f(x)}{g(x)}\right) = \lims{x\to a}\left(\dfrac{f'(x)}{g'(x)}\right)}\\
              \scriptsize{Examples:}\\
              \form{\lims{x\to 0}\left(\dfrac{sin x}{x}\right) = \lims{x\to 0}\left(\dfrac{cos x}{1}\right) = \lims{x\to 0}(cos(x)) = 1 }\\
              \pic{220615-4}\\ 
              }\\
              \hline
            \end{tabular}
            \hspace{-0.07in}
            \begin{tabular}[t]{|@{\hskip1pt}p{10.9cm}|}
              \hline
              \mc{\pic{220615-5}\\ \pic{220615-6} \\ 
              }\\
              \hline
              \mc{\footnotesize{Reforming terms for L'Hospital} \\ 
              \scriptsize{reforming a product}\\
              \pic{220615-7}\\ 
              \scriptsize{reforming an exponent with logarithm}\\
              \pic{220615-8} \\}\\
              \hline
              \mc{\footnotesize{Infinity calculation rules} \\
                {\begin{tabular}[t]{|c|c|c|}
              \hline
              \mc{\forms{\infty + c = \infty}} &
              \mc{\forms{\infty + \infty = \infty}} &
              \mc{\forms{\infty - \infty = NaN}} \\
              \hline
              \mc{\forms{\infty * c = \infty \to c \neq 0}} &
              \mc{\forms{\infty * \infty = \infty}} &
              \mc{\forms{\infty * 0 = NaN} }\\
              \hline
              \mc{\forms{\dfrac{c}{0} = \pm \infty \to c \neq 0}}&
              \mc{\forms{\dfrac{c}{\infty} = 0}}&
              \mc{\forms{\dfrac{\infty}{c} = \infty \to c \neq 0}}\\
              \hline
              \mc{\forms{\dfrac{\infty}{0} = \infty}}&
              \mc{\forms{\dfrac{0}{0} = NaN}}&
              \mc{\forms{\dfrac{\infty}{\infty} = NaN}}\\
              \hline
              \mc{\forms{0^c \to c > 0 \backslash{} (c = 1) = 0}}&
              \mc{\forms{0^0 = \text{1 or NaN}}}&
              \mc{\forms{\infty^0 = NaN}}\\
              \hline
              \mc{\forms{0^c \to c < 0 = \infty}}&
              \mc{\forms{k^{\infty} \to k > 1 = \infty}}&
              \mc{\forms{k^{\infty} \ to 0 < k < 1 = 0}}\\
              \hline
              \mc{\forms{0^{\infty} = 0}}&
              \mc{\forms{\infty^{\infty} = \infty}}&
              \mc{\forms{1^{\infty} = NaN}}\\
              \hline
              \end{tabular}}
              }\\\\
              \hline
              \mc{\small{\r{Integration}}}\\
              \hline
              \mc{\footnotesize{Similarity to limit}\\
              \scriptsize{Just like limit, you can do it by intuition, by simply adding}\\
              \scriptsize{more and more rectangles into a function to get the area of said function.}\\
              \pic{220615-9} \\ \pic{220615-10}\\
              \scriptsize{But just like with limit, there is a more elegant and generalized way. -> \r{\(\int_{a}^{b} x \,dx\)}}\\
              }\\
              \hline
              \mc{\footnotesize{Definite Integrals and Integral Terms} \\ 
              \form{\int_a^b f(x) \,dx = F(b) - F(a)}\\
              \scriptsize{-- The weird symbol is called integral sign}\\
              \scriptsize{-- The a and b are the upper and lower limits respectively}\\
              \scriptsize{-- f(x) is the integrand, the function to be integrated}\\
              \scriptsize{-- \r{F(a) or F(b) is the antiderivative -> opposite calculation to derivation}}\\
              \scriptsize{-- dx is the infinitesimal, no real use, but is required for notation}\\\\
              \scriptsize{An integral with specific limits -> range is called a \r{Definite Integral}\\
              \scriptsize{\r{Here the range is a to b}}}\\
              }\\
              \hline
            \end{tabular}
          \end{tabular}
          \hfill
        \end{tiny}
    \end{table}
    \pagebreak
    \begin{table}[ht!]
        \begin{tiny}
          \begin{tabular}{ccc}
            \begin{tabular}[t]{|@{\hskip1pt}p{10cm}|}
                \hline
                \mc{\footnotesize{Indefinite Integrals} \\ 
                \scriptsize{Since we can't put in values with infinite integrals, we instead}\\
                \scriptsize{just evaluate the antiderivative F(x), which in itself is yet another function}\\
                \form{\int f(x) \, dx = F(x) + C \to \text{ look at that, the holy constant C}}\\
                \scriptsize{\r{Note that the C always has to be written, as the integral function}}\\
                \scriptsize{\r{covers a range of values with F(x) plus some constant! Hence + C!}}\\\\
                \scriptsize{hence we can also go back again -> reversibility of integrals and derivations}\\
                \scriptsize{In other words, we differentiate the antiderivative!}\\
                \form{F'(x) = f(x) \to [F(x) + C]' = f(x) \to \text{ C vanishes -> constant!}}\\
                \scriptsize{One might ask now, why do we not consider it with definite integrals?}\\
                \scriptsize{Check how the C would affect a - b:}\\
                \form{\atb{x^2} = \left(\dfrac{b^3}{3} + C \right) - \left(\dfrac{a^3}{3} + C\right) = \dfrac{b^3}{3} - \dfrac{a^3}{3} \to C - C = 0 }\\
                \scriptsize{as one can see, the C simply gets canceled.}\\
              }\\
              \hline
              \mc{\footnotesize{Integral Rules} \\
              \scriptsize{The Integral of a to b is the same as the negative integral of b to a}\\
              \form{\int^a_b x \,dx = - \int^b_a x \,dx}\\
              \scriptsize{The Integral of a to a is 0 -> as the area would be 0. a - a = 0}\\
              \form{\int_a^a x \,dx = 0}\\
              \scriptsize{Since we are talking about areas, 2 areas in the same function add up:}\\
              \pic{220615-11}\\ \vspace{-0.5in} \\ \hspace{1in} \noskipform{\int_a^b x \,dx + \int_b^c x \,dx = \int_a^c x \,dx}\\ \vspace{0.3in} \\
              \scriptsize{If we integrate a constant, then the constant will multiple with x = 1: }\\
              \form{\int_a^b c \,dx = c * (b - a) \to \int_1^5 3 \,dx = 3 * (5 - 1) = 12 \to \text{ given c is constant}}\\
              \pic{220615-12}\\
              \form{\int c * f(x) \,dx = c * \int f(x) \, dx \to \text{ given c is constant}}\\
              \scriptsize{multiplying the integrand with a function can be done outside of the integral!}\\
              \form{\int c * f(x) \,dx = c * \int f(x) \, dx}\\
              \scriptsize{Sum of Integrals}\\
              \form{\int [f(x) + g(x)] = \int f(x) + \int g(x)}\\
              \scriptsize{Difference of Integrals}\\
              \form{\int [f(x) - g(x)] = \int f(x) - \int g(x)}\\
              \scriptsize{Integrals with y >= 0 and y < 0}\\
              \pic{220615-13}\\
              \scriptsize{Area of integrals with y below and above 0 at some point}\\
              \pic{220615-14}\\
              \scriptsize{often used:}\\
              \form{\int x^n \, dx = \dfrac{x^{n+1}}{n + 1} + C}\\
              \small{\r{And lastly the most important function!}}\\\\
              \normalsize{\r{\(\dxo \int f(x) \, dx = f(x)\)}}\\\\
              }\\
              \hline
              \mc{\footnotesize{Example for Integral calculation}\\
              \form{\int_0^3 (x + 5) \,dx = F(3) - F(0) \to F(x) = (\dfrac{x^2}{2} + 5x)} }
              \form{\to F(3) - F(0) = \left(\dfrac{3^2}{2} + 5 * 3 \right) - \left(\dfrac{0^2}{2} + 5 * 0\right) = \dfrac{39}{2} }\\
              \hline
            \end{tabular}
            \hspace{-0.07in}
            \begin{tabular}[t]{|@{\hskip1pt}p{10.9cm}|}
              \hline
              \mc{\normalsize{\r{Integrals do not have the product rule}} \\\\ 
                \footnotesize{\r{This means that we need to find a different way to remove factors}}\\
              \footnotesize{\r{In fact, Integrals can only be taken over sums and differences}}\\
              \pic{220615-15}\\
              }\\
              \hline
              \mc{\Large{Substitution Rule \r{usually the best!}} \\ 
              \scriptsize{This turns complicated nested integrands into smaller pieces}\\
              \scriptsize{This is the correspondent technique to the chain rule! Note f(x) can be f(x) = x}\\\\
              \Large{\r{\(\int f[g(x)] * g'(x)\, dx = \int f(u) du \)}}\\\\
              \Large{\r{OR: \(\int g(x) * g'(x)\, dx = \int u \, du \)}}\\\\
              \Large{\r{\(\ u = g(x) \text{ || } du = g'(x) dx\)}}\\\\
              \pic{220615-16}\\
              \scriptsize{Example:}\\
              \form{\int 2x * cos(x^2 + 1) \, dx \to 2x = g'(x) \to cos(x^2 + 1) = f[g(x)] \to (x^2 + 1) = g(x)}\\
              \form{(x^2 + 1) = u \to (x^2 + 1)' = 2x \text{ this means we can use subtitution!}}\\
              \form{\int cos(u) 2x \, dx = \int cos(u) \, du = sin(u) + C = sin(x^2 + 1) + C }\\
              \scriptsize{Second example with factors}\\
              \form{\int x^2 \sqrt{x^3 + 1} \, dx \to x^2 = g'(x) \to f(g(x)) = \sqrt{x^3 + 1} \to g(x) = (x^3 + 1)}\\
              \form{(x^3 + 1)' = \dfrac{1}{3}x^2 \text{ note 1/3 is a factor, it can be removed from the term.}}\\
              \form{\dfrac{1}{3}* \int \sqrt{u} * x^2 \, dx = \dfrac{1}{3} * \int u^{\frac{1}{2}} \, du = \dfrac{1}{3} * -\dfrac{x^{\frac{3}{2}}}{\frac{3}{2}} }\\
              \form{\frac{1}{3} * \frac{2}{3} * u^{\frac{3}{2}} + C = \frac{2}{9} * (x^3 + 1)^{\frac{3}{2}} + C}\\
              }\\
              \hline\mc{\Large{Integration by Parts} \\ 
              \scriptsize{This turns complicated nested integrands into smaller pieces}\\
              \scriptsize{This is the correspondent technique to the product rule!}\\\\
              \Large{\r{\(\int [f(x) * g'(x) + f'(x) * g(x)] \, dx = f(x) * g(x) \)}}\\\\
              \Large{\r{\(\int [f(x) * g'(x)] \,dx + \int [f'(x) * g(x)] \, dx = f(x) * g(x) \)}}\\\\
              \Large{\textcolor{Blue}{\(\int [f(x) * g'(x)] \, dx =  f(x) * g(x) - \int [f'(x) * g(x)] \, dx \)}}\\\\
              \scriptsize{\r{Simply the opposite to the Product rule!!}}\\
              \scriptsize{\r{Unlike the substitution method, this works for EVERY product!}}\\\\
              \Large{\textcolor{Blue}{\(\int u \, dv = uv - \int v \, du \)}}\\\\
              \scriptsize{The same as before, just with a simplified view.}\\
              \scriptsize{Please note: this technique doesn't always simplify the term, sometimes}\\
              \scriptsize{a different method is better!}\\
              \pic{220615-17}\\ \pic{220615-18}\\
              }\\
              \hline
            \end{tabular}
          \end{tabular}
          \hfill
        \end{tiny}
    \end{table}
    \pagebreak
    \begin{table}[ht!]
        \begin{tiny}
          \begin{tabular}{ccc}
            \begin{tabular}[t]{|@{\hskip1pt}p{10cm}|}
              \hline
              \mc{\pic{220615-19}\\ \pic{220615-20} }\\
              \hline
              \mc{\scriptsize{Example Integration by Parts:} \\ 
                \form{\int \dfrac{ln(x)}{x^2} \, dx \to ln(x) = f(x) \to x^2 = g'(x)}\\
                \form{\int \dfrac{ln(x)}{x^2} \, dx = -\left(\dfrac{ln(x)}{x}\right) - \int \dfrac{1}{x} * -\dfrac{1}{x} }\\
                \form{\int \dfrac{ln(x)}{x^2} \, dx = -\left(\dfrac{ln(x)}{x}\right) + \int \dfrac{1}{x^2}}\\
                \form{\int \dfrac{ln(x)}{x^2} \, dx = -\left(\dfrac{ln(x)}{x}\right) - \dfrac{1}{x} + C}\\
              }\\
              \hline
              \mc{\Large{Integration by Trigonometric Substitution}\\
              \footnotesize{This can be done with the following 3 situations}\\\\
              \Large{\r{\(\sqrt{a^2 - x^2} \text{ || } \sqrt{a^2 + x^2} \text{ || } \sqrt{x^2 - a^2}\)}}\\
              \Large{\r{Where a can be any positive number}}\\\\
              \pic{220615-21}\\ \pic{220615-22}\\
              \Large{SohCahToa / ChoShaCao}\\
              \scriptsize{\(sin = \dfrac{opposite}{hypotenuse}\)}
              \hspace{0.1in}
              \scriptsize{\(cos = \dfrac{adjacent}{hypotenuse}\)}
              \hspace{0.1in}
              \scriptsize{\(tan = \dfrac{opposite}{adjacent}\)}\\\\
              \scriptsize{\(csc = \dfrac{hypotenuse}{opposite}\)}
              \hspace{0.1in}
              \scriptsize{\(sec = \dfrac{hypotenuse}{adjacent}\)}
              \hspace{0.1in}
              \scriptsize{\(cot = \dfrac{adjacent}{opposite}\)}\\\\
              \pic{220616-1}\\
              \pic{220616-2}\\
              }\\
              \hline
            \end{tabular}
            \hspace{-0.07in}
            \begin{tabular}[t]{|@{\hskip1pt}p{10.9cm}|}
              \hline
              \mc{\Large{\r{Steps for Integration by Trigonometric Substitution}}\\\\
              \large{\r{1. Check if the term matches one of the 3 possible scenarios}}\\
              \large{\r{2. Replace all x with one of the signatures}}\\
              \large{\r{3. Replace the dx with d\(\Theta\) \(\dfrac{dx}{d\Theta}\) }}\\ 
              \large{\r{-- This means taking the derivation of x }}\\
              \large{\r{4. simplify and integrate, don't forget the + C}}\\
              \large{\r{5. Put in the values of x into the triangle to get the result}}\\
              \large{\r{6. Simplify if needed.}}\\
              }\\
              \hline
              \mc{\Large{\r{Integral Lookup table}}\\
              \pic{220616-3}\\
              }\\
              \hline
              \mc{\Large{\r{Substitution without 2 functions}}\\
              \pic{220616-4}\\
              }\\
              \hline
              \mc{\scriptsize{}Another example as brainfuck:}\\
              \pic{220616-5}\\
              \hline
              \mc{\Large{\r{Factor plays}} \\\\
              \large{\r{\(\int x * sin(x^2) \, dx\)}}\\\\
              \normalsize{This seems like there is no way to take the integral.}\\
              \normalsize{However, look at x, we only need the factor 2 to make it work.}\\
              \normalsize{How about we just slap a 1/2 infront to make up for the missing 2?}\\
              \large{\r{!! \(\int x * sin(x^2) \, dx == \dfrac{1}{2} * \int 2x * sin(x^2) \, dx\) !!}}\\\\
              \normalsize{Now we can just do regular substitution to get this:}\\\\
              \large{\r{\( \dfrac{-cos(x^2)}{2} + C \)}}\\\\
              }\\
              \hline
              \mc{\scriptsize{Another Example:} 
                \form{\int x^3 * sin(x^2) \, dx}\\
                \scriptsize{strategy: change term}\\
                \form{\int x^2 * x * sin(x^2) \, dx}\\
                \scriptsize{strategy: integral by parts}\\
                \form{\int x^2 \to u * (x sin(x^2)) \to du = uv - \int v * du}\\
                \scriptsize{strategy: integrate dv to v with strategy substitution and factor play}\\
                \form{\r{\dfrac{1}{2}}\int \r{2} * x sin(x^2) \, dx = \dfrac{1}{2}-cos(x^2) }\\
                \form{\int u * du = x^2 * \dfrac{-1}{2}cos(x^2) - \int 2x * \dfrac{-1}{2}cos(x^2) }\\
              }\\
              \hline
            \end{tabular}
          \end{tabular}
        \hfill
      \end{tiny}
    \end{table}
    \pagebreak
    \begin{table}[ht!]
        \begin{tiny}
          \begin{tabular}{ccc}
            \begin{tabular}[t]{|@{\hskip1pt}p{10cm}|}
              \hline
                \mc{
                \scriptsize{strategy: integrate v du with strategy substitution and factor play}\\
                \form{\int 2x * \dfrac{-1}{2}cos(x^2) = \int x * -cos(x^2)  = \r{\dfrac{1}{2}} \int \r{2}x * -cos(x^2) = \dfrac{-1}{2}sin(x^2) }\\
                \form{\int u * du = \r{x^2 * \dfrac{-1}{2}cos(x^2) + \dfrac{1}{2}sin(x^2) \b{ + C }}}\\
                \scriptsize{C is only necessary at the END of the integral.}\\
                \scriptsize{Putting it in before isn't wrong, but there is no point!}\\
                }\\
                \hline
                \mc{\footnotesize{Improper Integrals}\\
                \scriptsize{These are integrals with strange ranges:}\\
                \large{\r{\(\int_a^\infty f(x) \, dx\)}} \hspace{0.5in} \large{\r{\(\int_{-\infty}^b f(x) \, dx\)}}\\\\
                \large{\r{\(\int_{-\infty}^\infty f(x) \, dx\)}} \hspace{0.5in} \large{\r{\(\int_a^b f(x) \, dx\)}}\\
                \hspace{1.2in} \large{ contains a discontinuity}\\\\
                \pic{220616-6}\\ \pic{220616-7} \\ \pic{220616-8}\\
                }\\
              \hline
              \mc{\small{\r{Taylor Series}}\\\\
              \large{\r{\(g(x) = \sum_{k=0}^\infty a_k (x - x_0)^k\)}}\\\\
              \large{\r{\(a_k = \dfrac{g^{(k)}(x0)}{k!}\)}}\\\\
              \footnotesize{The difference between a taylor series and a fourier series}\\
              \footnotesize{is that a taylor series can be any polynomial, fourier is made of trig functions}
              }\\\\
              \hline
              \mc{\Large{Fourier Series}\\\\
                \footnotesize{First we need to understand why the fourier series even works}\\
                \footnotesize{Every 90 degrees, the sin wave will be 0, -> sin(0), sin(90) ...}\\
                \footnotesize{This is regardless of how fast the frequency is!.}\\
                \footnotesize{So no matter what you multiply a sine wave with, 0 will be 0 90 will be 0}\\
                \pic{220616-9}\\
                \footnotesize{Adding to this, the fourier series can already be read out of this.}\\
                \footnotesize{check the top of these sine waves, the maximum almost looks like }\\
                \footnotesize{a rectangle, aka a -1 0 1 signal!!, with just 4 different sine waves!}\\\\
                \Large{At last, the actual series:}\\\\
                \Large{\r{\(s_N(x) = a_0 + \sum_{k=1}^\infty a_k \, cos(\omega k \, x) + b_k \, sin(\omega k \, x)\)}}\\\\
              }\\
              \hline
            \end{tabular}
            \hspace{-0.07in}
            \begin{tabular}[t]{|@{\hskip1pt}p{10.9cm}|}
              \hline
              \mc{\footnotesize{Fourier Terms}\\\\
              \footnotesize{\r{p = Period }}\\
              \footnotesize{\r{\(\alpha\) = variable to multiply a period -> \(f(\alpha x)\) >> default 0}}\\\\
              \footnotesize{\r{T = Calculated Period when \(a \geq 1\) ->\(\dfrac{p}{a}\) can also be called the primitive period.}}\\\\
              \footnotesize{\r{\(\omega = \text{Frequency} \dfrac{2 \pi}{T}\)}}\\\\
              \footnotesize{\r{k/m/n factors for frequency!}}\\\\
              \scriptsize{\r{smallest period == primitive period}}\\
              \scriptsize{The fastest time that the function repeats itself -> sin(360)}\\
              \scriptsize{\(f(x+p) = f(x) \to sin(x + 360) = sin(x) \) }\\
              \scriptsize{Even every multiple of 360 gives the same period, just multiple times.}\\
              \scriptsize{\(f(x+2p) = f(x) \to sin(x + 2 * 360) = sin(x + 720) = sin(x))\)}\\\\
              \scriptsize{\r{Constant periodic function}}\\
              \scriptsize{For a constant function, there is no primitive period.}\\
              \scriptsize{I mean how, a constant doesn't change so the minimum period is 0....}\\
              \scriptsize{for a constant function, every p > 0 is a valid period}\\\\
              \scriptsize{\r{Multiplying a periodic function}}\\
              \scriptsize{Take a look at sin(x), if we multiply x with a number, what happens to the period?}\\
              \scriptsize{\(sin(2x) \to sin(2*0) = 0 \to sin(2 * 180) = sin(360) = 0\)}\\
              \scriptsize{As you can see the period gets halved, it is now 180 instead of 360.}\\
              \scriptsize{This can be calculated by the following}\\
              \scriptsize{\r{\(f(\alpha x) \to f = \dfrac{p}{a} \to sin(2x) \to f = \dfrac{360}{2} = 180\)}}\\\\
              \scriptsize{\r{Periodic Function Addition}}\\
              \scriptsize{Adding a function with period p to another function with period p}\\
              \scriptsize{results in another function with period p!}\\
              \scriptsize{\r{\(h(x) = f(x) + g(x) \to pf = pg\)}}\\
              \footnotesize{\r{This is also called a Linear Combination!}}\\\\
              \scriptsize{\r{}}\\
              \scriptsize{}\\
              \scriptsize{\r{}}\\
              \scriptsize{}\\
              \scriptsize{\r{}}\\
              \scriptsize{}\\
              }\\
              \hline
              \mc{\foonotesize{Additionterms}\\
              \large{\r{\(sin(x + y) = sin(x) * cos(y) \pm sin(y) * cos(x) \)}}\\\\
              \large{\r{\(cos(x + y) = cos(x) * cos(y) \pm sin(y) * sin(x) \)}}\\\\
              \large{\r{\(\text{for } \omega = \dfrac{2\pi}{T} \text{ and m/n} \in \mathbb{N}\)}}\\\\
              \large{\r{
                \(\int_0^{2\pi} sin(\omega m \, x) \, sin(\omega n \, x) \, dx =\)
              \begin{equation}
                \begin{cases}
                0, \, &  m \neq n\\
                \pi, &  \, m = n
                \end{cases}
              \end{equation}
              }}\\\\
              \large{\r{
                \(\int_0^{2\pi} cos(\omega m \, x) \, cos(\omega n \, x) \, dx =\)
              \begin{equation}
                \begin{cases}
                0, \, &  m \neq n\\
                \pi, &  \, m = n
                \end{cases}
              \end{equation}
              }}\\\\
              \large{\r{\(\int_0^{2\pi} cos(\omega m \, x) \, sin(\omega n \, x) \, dx = 0\)}}\\\\
              \normalsize{\r{\( \int sin(k * x) \, dx \to f[g(x)] = sin(x * k) \to g'(x) = dx * \b{k} \)}}\\
              \normalsize{\r{the k is artificially added!}}\\
              \normalsize{\r{\( \dfrac{\b{1}}{\b{k}} \int sin(u) \, du = \dfrac{1}{k} * -cos(u) = \dfrac{-cos(x * k)}{k}\)}}\\\\
              }\\
              \hline
              \mc{\normalsize{T-Periodic}\\\\
              \large{\r{\( a_0 = \dfrac{1}{T} \int_0^T f(t) \, dt \)}}\\\\
              \large{\r{\( a_m = \dfrac{2}{T} \int_0^T f(t) cos(\omega lt) \, dt \)}}\\\\
              \large{\r{\( b_m = \dfrac{2}{T} \int_0^T f(t) sin(\omega lt) \, dt \)}}\\\\
              \large{\r{\( \int_0^\infty (f(t) - S_N(t))^2 \, dt \to^{N\to \infty} 0 \)}}\\\\
              }\\
              \hline
              \mc{\pic{220617-1}}\\
              \hline
            \end{tabular}
          \end{tabular}
        \hfill
      \end{tiny}
    \end{table}
    \pagebreak
    \begin{table}[ht!]
        \begin{tiny}
          \begin{tabular}{ccc}
            \begin{tabular}[ht!]{|@{\hskip1pt}p{10cm}|}
              \hline
              \mc{\pic{220617-2} \\ \pic{220617-3}}\\
              \hline
              \\\\\\\\\\\\\\\\\\\\\\\\\\\\\\
              \\\\\\\\\\\\\\\\\\\\\\\\\\\\\\
              \\\\\\\\\\\\\\\\\\\\\\\\\\\\\\
              \\\\\\\\\\\\\\\\\\\\\\\\\\\\\\
              \\\\\\\\\\\\\\\\\\\\\\\\\\\\\\
              \\\\\\\\\\\\\\\\\\\\\\\\\\\\\\
              \\\\\\\\\\
              \hline
            \end{tabular}
          \end{tabular}
        \hfill
      \end{tiny}
    \end{table}
\end{document}
