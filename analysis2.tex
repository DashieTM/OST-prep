\documentclass{article}
\usepackage[left=-1.5mm, right=-1.5mm, top=-1mm, bottom=-1.5mm]{geometry}
\usepackage[document]{ragged2e}
\usepackage{graphicx}
\usepackage{amsmath, mathtools, nccmath, amssymb}
\usepackage[dvipsnames]{xcolor}
\usepackage{array, makecell}
\usepackage{moresize}
\usepackage[T1]{fontenc}
\usepackage{noto-sans}
\renewcommand\cellalign{cl}
\renewcommand\theadalign{lc}
\newcommand{\ns}{\\ \vspace{-0.03in}}
\newcommand{\mc}[1]{\makecell[cl]{\vspace{-0.065in} \\ #1}}
\newcommand{\mcc}{\makecell[{{c}}]}
\newcommand{\mcr}{\makecell[{{r}}]}
\newcommand{\pic}{\includegraphics[scale=0.3]}
\newcommand{\form}[1]{\vspace{-0.03in}\\ \small{\textcolor{Red}{\(#1\)}} \\ \vspace{-0.03in}}
\newcommand{\noskipform}[1]{\small{\textcolor{Red}{\(#1\)}}}
\newcommand{\fpic}{\vspace{-0.05in} \\ \includegraphics[scale=0.3]}
\newcommand{\dx}[1]{\scriptsize{\(\dfrac{d}{dx}#1\)}}
\newcommand{\dxs}[1]{\ssmall{\(\dfrac{d}{dx}#1\)}}
\newcommand{\dxo}{\dfrac{d}{dx}}
\renewcommand{\b}{\textbf}
\renewcommand{\r}[1]{\textcolor{Red}{#1}}
%\renewcommand{\arraystretch}{1.5}
\graphicspath{{./Pictures/}}
\begin{document}
    \begin{table}[ht!]
        \begin{tiny}
          \begin{tabular}{ccc}
          \begin{tabular}[t]{|@{\hskip1pt}p{10cm}|}
              \hline
              \mc{\footnotesize{Derivation Rules} \\
              \dx{(x^a) = a * x^{a-1} \text{ given: } x,a \in \mathbb{R} \text{ \& } x > 0}\\\\
              \ssmall{subexamples:}\\
              \dxs{x = 1 \to \dxo(x^1) = 1 * x^{1-1}}\\
              \dxs{x^2 = 2x \to \dxo(x^2) = 2 * x^{2-1}}\\
              \dxs{\dfrac{1}{x} = -\dfrac{1}{x^2} \to \dxo(x^{-1} = -1 * x^{-1-1})}\\
              \dxs{\sqrt{x} = \frac{1}{2 * \sqrt{x}} \to \dxo(x^{\frac{1}{2}}) = \frac{1}{2 * x^{\frac{1}{2}}}}\\
              \dx{(c) = 0 \text{ given: } c \in \mathbb{R} \text{ \& c is constant \& \r{c != factor}} }\\\\
              \dx{(e^x) = e^x \to \dxo(e^x) = ln(e) * e^x * x' = 1 * 1 * e^x}\\\\
              \dx{(a^x) = ln(a) * a^x \to \dxo(a^x) = x' * ln(x) * a^x \text{ because: } e^{x * ln(a)} = a^x }\\\\
              \ssmall{Note for this rule:}\\
              \ssmall{\( \dxo(2^{2x + 1}) = ln(2x + 1) * 2^{2x + 1} * (2x + 1)' = ln(2x + 1) * 2^{2x + 1} * 2  \)}\\\\
              \dx{(ln(x)) = \frac{1}{x}}\\\\
              \dx{log_b(x) = \frac{1}{ln(b) * x } \to \text{special case for} \dxo(\frac{ln(x)}{ln(b)})}\\\\
              \ssmall{this is the case because of base change in lograrithmic functions!}\\
              \ssmall{\( log_a(x) = \dfrac{ln(x)}{ln(a)} = \dfrac{log_c(x)}{log_c(a)} \to \dxo log_a(x) = \dfrac{ln(x)}{ln(a)} -> \dxo ln(x) = \dfrac{\dfrac{1}{x}}{ln(a)} = \dfrac{1}{ln(a) * x}\)}\\
              \ssmall{c can be any number!}\\\\
              \dx{sin(x) = cos(x)}\\\\
              \dx{cos(x) = -sin(x)}\\\\
              \dx{tan(x) = \frac{1}{cos^2(x)}}\\\\
              \dx{tan(x) = 1 + tan^2(x)}\\\\
              \dx{(ax) = a \to \dxo(ax) = a * 1 \to \text{ we derive x NOT a!!}}\\\
              \dx{(3x) = a \to \dxo(3x) = 3 * 1 \to \text{ 3 is a factor!}}\\\\
              }\\
              \hline
              \mc{\footnotesize{All of these derive from:} \\
                \form{f'(a) = \lim_{h\to0} \frac{f(a+h) - f(a)}{h}}\\
              }\\
              \hline
              \mc{\footnotesize{Sum Rule} \\
              \form{(f + g)' = f' + g'}\\
              }\\
              \hline
              \mc{\footnotesize{Difference Rule} \\ 
              \form{(f - g)' = f' - g'}\\
              }\\
              \hline
              \mc{\footnotesize{Product Rule} \\ 
              \form{(f * g)' = f * g' + f' * g }\\
              }\\
              \hline
              \mc{\footnotesize{Quotient Rule} \\ 
              \form{\left(\dfrac{f(x)}{g(x)}\right)' =\dfrac{f' * g - f * g' }{g^2}}\\
              }\\
              \hline
              \mc{\footnotesize{Chain Rule} \\ 
              \form{[f(g(x))]' = f'(g(x)) * g'(x)}\\
              \scriptsize{Example:}\\
              \form{\dxo\sqrt{x^2 + 1} = (x^2 + 1)^{\frac{1}{2}} = \frac{1}{2} * (x^2 + 1)^{-\frac{1}{2}} * 2x}
              \form{\frac{1}{2} * \frac{1}{\sqrt{x^2 + 1}} * 2x = \frac{2x}{2 * \sqrt{x^2 + 1}} = \frac{x}{\sqrt{x^2 + 1}}}\\
              \scriptsize{Note: \((x^2 + 1)\) is g(x), while f is the exponent function}\\
              \scriptsize{More examples:}\\
              \form{\dxo sin(x^2) = sin(x^2)' * (x^2)' = cos(x^2) * 2x}\\
              \form{\dxo sin^2(x) = \dxo(sin(x))^2 = 2 * sin(x) * cos(x)}\\
              \form{\dxo \left(\dfrac{x - 1}{x + 1}\right)^2 = 2 * \left(\dfrac{x - 1}{x + 1}\right) * \left(\dfrac{x - 1}{x + 1}\right)'}\\
              \form{\dxo (x + 2)^3(x)^4 = (x + 2)^3 * ((x)^4)' + ((x + 2)^3)' * (x)^4 }
              \form{((x + 2)^3)' = 3 * (x + 2)^2 * (x + 2)' = 3 * (x + 2)^2 * 1}\\
              \form{\dxo sin(cos[tan(x)]) = cos(cos[tan(x)]) * -sin(tan(x)) * \dfrac{1}{cos^2(x)} }
              }\\ 
              \hline
            \end{tabular}
            \hspace{-0.07in}
            \begin{tabular}[t]{|@{\hskip1pt}p{10.90cm}|}
              \hline
              \mc{\footnotesize{Implicit Differentiation} \\ 
                \dx{(x^2 + y^2 = 9) \to 2x + \dfrac{d}{dx}((y)^2) * \dfrac{dy}{dx}(y) = 0 \to 2x + (2y * y') = 0 \to y' = \dfrac{-2x}{2y}}\\\\
                \scriptsize{!! Remember that this is only necessary if y needs to be derived !!}\\
              }\\
              \hline
              \mc{\footnotesize{Higher Derivatives} \\ 
              \scriptsize{The best idea for higher derivatives is distance s, velocity v and acceleration a.}\\
              \scriptsize{\(\dfrac{d}{dt}(s(t)) = v(t) = s'(t) \text{ || } \dfrac{d}{dt}(v(t)) = a(t) = s''(t) = v'(t) \)}\\\\
              \scriptsize{\r{This is why the acceleration on earth -> gravity is constant!! HOLY FUCK}}
              }\\
              \hline
              \mc{\footnotesize{Taking Derivations higher than 3} \\ 
              \scriptsize{\(1:f' \to 2:f'' \to 3:f''' \to \r{4:f^{(4)}} \to \r{n:f^{(n)}}\)}}\\\\
              \hline
              \mc{}\\
              \hline
              \mc{}\\
              \hline
              \mc{}\\
              \hline
              \mc{}\\
              \hline
              \mc{}\\
              \hline
              \mc{}\\
              \hline
              \mc{}\\
              \hline
            \end{tabular}
          \end{tabular}
          \hfill
        \end{tiny}
    \end{table}
    \pagebreak
    \begin{table}[ht!]
        \begin{tiny}
          \begin{tabular}{ccc}
            \begin{tabular}[t]{|@{\hskip1pt}p{6.9cm}|}
              \hline
              \mc{}\\
              \hline
            \end{tabular}
            \hspace{-0.07in}
            \begin{tabular}[t]{|@{\hskip1pt}p{6.9cm}|}
              \hline
              \mc{}\\
              \hline
            \end{tabular}
            \hspace{-0.07in}
            \begin{tabular}[t]{|@{\hskip1pt}p{6.9cm}|}
              \hline
              \mc{}\\
              \hline
            \end{tabular}
          \end{tabular}
          \hfill
        \end{tiny}
    \end{table}
    \pagebreak
    \begin{table}[ht!]
        \begin{tiny}
          \begin{tabular}{ccc}
            \begin{tabular}[t]{|@{\hskip1pt}p{6.9cm}|}
              \hline
              \mc{}\\
              \hline
            \end{tabular}
            \hspace{-0.07in}
            \begin{tabular}[t]{|@{\hskip1pt}p{6.9cm}|}
              \hline
              \mc{}\\
              \hline
            \end{tabular}
            \hspace{-0.07in}
            \begin{tabular}[t]{|@{\hskip1pt}p{6.9cm}|}
              \hline
              \mc{}\\
              \hline
            \end{tabular}
          \end{tabular}
          \hfill
        \end{tiny}
    \end{table}
    \pagebreak
    \begin{table}[ht!]
        \begin{tiny}
          \begin{tabular}{ccc}
            \begin{tabular}[t]{|@{\hskip1pt}p{6.9cm}|}
              \hline
              \mc{}\\
              \hline
            \end{tabular}
            \hspace{-0.07in}
            \begin{tabular}[t]{|@{\hskip1pt}p{6.9cm}|}
              \hline
              \mc{}\\
              \hline
            \end{tabular}
            \hspace{-0.07in}
            \begin{tabular}[t]{|@{\hskip1pt}p{6.9cm}|}
              \hline
              \mc{}\\
              \hline
            \end{tabular}
          \end{tabular}
          \hfill
        \end{tiny}
    \end{table}
\end{document}
