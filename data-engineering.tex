\documentclass{article}
\usepackage[left=0mm, right=0mm, top=0mm, bottom=0mm]{geometry}
\usepackage[document]{ragged2e}
\usepackage{graphicx}
\usepackage{array, makecell}
\usepackage{verbatimbox}
\usepackage[T1]{fontenc}
\usepackage{tabularray}
\usepackage{tabulary}
\newcommand{\mc}{\makecell[l]}
\newcommand{\mcc}{\makecell[{{c}}]}
\newcommand{\mcr}{\makecell[{{r}}]}
\renewcommand{\b}{\textbf}
\newcommand{\pic}{\includegraphics[scale=0.3]}
\newcommand{\fpic}{\vspace{-0.05in} \\ \includegraphics[scale=0.3]}
\newcommand{\hl}{\hline}
\renewcommand{\arraystretch}{1.5}
\graphicspath{{./Pictures/}}
\begin{document}
\begin{flushleft}
  \begin{table}[ht!]
  \begin{flushleft}
    \begin{tiny}
      \begin{tabular}{lll}
        \begin{tabular}[t]{|l|}
        \hl
        \mcc{\b{Functions in PGPLSQL}} \\
        \hl
        \mc{functions in psql \\ create or replace function funcName() \\ return \b{s} returntype as \b{ \$\$ } \\ begin \\  raise notice 'Hello Birb!'; \\ end; \\ \b{\$\$ }
        language langName \\ The two \$ are always necessary. Also note the \\ returns with an s and the language at the end. \\ Which \b{MUST} be a \b{PROCEDURAL LANGUAGE}, \\
        so c++ doesn't work here. } \\
        \hl
        \mc{Parameters are handled like in any language \\ func(x bigint, y bigint) \\ you can also define multiple return types\\ func(variadic a numeric[]) \\ or a generic return \\ 
        func(param anyelement)} \\ 
        \hl
        \mc{Variable Declaration: \\ returns void as \$\$ \\ DECLARE \\ x bigint; y bigint; \\ BEGIN ....} \\ 
        \hl
        \mc{Variable manipulation: x \b{:=} 6 + 4} \\
        \hl
        \mc{if: IF n = 0 THEN RETURN 1; \\ (optional)ELSE RETURN 2; END IF; \\ ELSIF also possible (note elsif not elseif)} \\ 
        \hl
        \mc{case x when 1, 2 then msg := 'one or two'; end case; \\ essentially this checks if x is 1 \b{OR} 2}\\
        \hl
        \mc{case when x between 0 and 10 then .... \\ similar but with a range, both can be\\ simulated by if else.}\\
        \hl
        \mc{Exceptions: BEGIN z:= x / y;\\ EXCEPTION WHEN division-by-zero \\ THEN z:= 0; (or error rather) END; \\ if you want to catch all: WHEN others THEN} \\ often used after exception: RAISE;
        (show error) \\ 
        \hl
        \mc{For Loop: For var IN query LOOP \\ statements END LOOP; \\ for r in SELECT * FROM ang LOOP \\ RETURN NEXT r; END LOOP; RETURN; END; \\ note that the return next doesn't return \\ you store 
        it in a buffer \\ and return it at the end of the function. \\ for infinite loops: FOR i IN 1..\_max LOOP;} \\
        \hl
        \mc{update and insert: INSERT INTO ANG VALUES(...); \\ UPDATE ang set salary = salary + 500 \\where name = 'dashie'; \\ interestingly, after the where name = 'dashie' \\ you can use if not found
        then (handle error)\\ this allows for easier error handling.}\\
        \hl
        \mc{queries: \b{execute} 'SELECT * from ang' \\ || into result; return result; END;} \\
        \hl
        \mc{comments are done by either -- or /* */ for multiline} \\ 
        \hl
        \mc{anonymous function: you can omit \\the name and just write do \$\$ ...} \\ 
        \hl
        \mc{cursor: declare curs CURSOR FOR query; \\BEGIN OPEN curs; LOOP do something CLOSE\\ curs; END;\\ Cursors are essentially just iterables. \\ cursors can also be unbound curs1 refcursor \\
        or they can be parameterized curs3 cursor(arg)}\\
        \hl
        \mc{ \fpic{220607-3} \\ arrays: SELECT '{1,2,3}'::int[] \\ or SELECT ARRAY[1,2,3] \\ var int[] only in variable declaration. \\ \b{!!arrays start with 1 in psql !!}}\\
        \hl
        \mc{return types: all of the above AND void, \\ SETOF type (array of a type), TABLE, Trigger} \\
        \hl
        \mc{ \fpic{220608-2} \\ \pic{220608-3}}\\
        \hl
        \mc{hstore / map:\\ \pic{220608-4} \\ \pic{220608-5} \\ \pic{220608-6}\\ hstore supports GIST/GIN indexing}\\
        \hl
      \end{tabular}
      \begin{tabular}[t]{|l|}
        \hl
        \mc{\fpic{220607-1} \\ also note that type\%rowtype is used like this: \\ r ang\%rowtype -> for r in select * from ang;}\\
        \hl
        \mc{important to know, you can always use these \\functions to manipulate queries, for example \\ select upper(name) from ang; \\ depending on the function you can also \\ 
        select generatetab(1,10)}\\
        \hl
        \mc{\fpic{220607-2}}\\
        \hl
        \mc{cast: cast(input as type); \\ cast(record.id as text);}\\
        \hl
        \mc{stored procedures are nothing but a chaining of functions:\\ \pic{220607-4}} \\
        \hl
        \mc{some good to know things: plain SQL is more efficient. \\ write variables lower case for sql \\ use cast over typename -> not select date '2022-06-07'}\\
        \hl
        \mcc{\b{Triggers}}\\
        \hl
        \mc{\fpic{220607-5}}\\
        \hl
        \mc{\fpic{220607-6} \\ statement is once, row means once \\per row, aka for the entire table. \\ \pic{220607-7}} \\ \pic{220607-8} \pic{220607-10}\\ \pic{220607-9} \pic{220607-11}\\ 
        \hl
        \mc{Return types: RETURN NEW -> returns a new table/row \\ RETURN OLD -> returns the old table/row \\ (but could change other rows!) \\ RETURN NULL -> cancel operation.}\\
        \hl
        \mc{running order: before statement, before row, \\ after row, after statement -> \\ and of course alphabetically.}\\
        \hl
        \mc{inside the Trigger functions you can use the \\ variables that don't matter aka can be ANY -> user \\ or the entered user from the trigger -> NEW.user \\ or explicitly the old one ->
        OLD.user \\ and last user defined stuff like -> SELECT 'I' \\ which just places an I as the variable \\ or something like now() for timestamps.}\\
        \hl
        \mc{Triggers make the database slower and harder to maintain. \\ some databases therefore let you disable them if you want.\\ On a table basis. \\ Also watch out for cascading effects of 
        triggers \\ they might cause something else to be deleted.}\\
        \hl
        \mc{Stored Procedures are really helpful for security \\ They have all the prviliges, but only \\ allow the user to do what the \\ creator has predefined.}\\
        \hl
      \end{tabular}
      \begin{tabular}[t]{|l|}
        \hl
        \mc{\fpic{220608-1}}\\
        \hl
        \mcc{\b{Optimization and indexing}} \\ 
        \hl
        \mc{The basics of indexing is that is saves \\ time on queries, but it uses more space, \\ 
        and needs to be redone on udpate/insert.}\\
        \hl
        \mc{Data stored in Pages / Heap (Collection of Pages) \\ Indexing on either can be possible. \\
        ex: Page index -> Primary key index\\ ex: Heap Index -> index on tables \\ PSQL does table 
        cluster indexing \\ instead of integrated indexing -> key value (both indexed)}\\
        \hl
        \mc{\b{B-Tree:} the default, can index multiple entries (only btree!) \\ CREATE UNIQUE INDEX name ON table (column [1,2...])}\\
        \hl
        \mc{\b{Hash Index:} just like hashmap in programming\\ good for single or small multiple queries \\
        bad for entire tables etc -> collision \\ hashing might take a long time with a lot of data. \\ B-tree almost always better!}\\
        \hl
        \mc{\b{GiST:} balanced/treelike , Range/neighbor/fulltext search \\ used for geometric datatypes. SP-Gist for unbalanced trees.}\\
        \hl
        \mc{\b{GIN:} General Inverted Index "list of words \\ that point to documents" wtf?. Good for duplicates. \\ Good for hstore,Json,Arrays \\ \pic{220608-12}}\\
        \hl
        \mc{\b{Bitmap indexing:} Bitmap -> 0 1 stores Booleans/Enums \\ very fast read / slow update \\ in postgres \b{only implicit} use Brin instead.}\\
        \hl
        \mc{\b{BRIN:} Block Range Index, stores min/max values as blocks \\ good for range search, sorted data, small disk usage \\ data is naturally sorted, address next to postal code.}\\
        \hl
        \mc{Bloom Index -> equality search, \\Trigram Index -> Full text search \\ RUM , non-default-GIN jsonb-path-ops}\\
        \hl
        \mc{creating index:\\CREATE INDEX <indexname> ON <table(attribute)>;  \\ and: DROP INDEX <indexname>;}\\
        \hl
        \mc{default index order for psql is btree,ASC,NULL first}\\
        \hl
        \mc{\fpic{220608-7} \\ \pic{220608-8}}\\
        \hl
        \mc{PG planner join strategies: Nested Loop,Merge,Hash \\ \b{-Nested Loop:} for r in right row r == for l in left row.... \\ good for small tables, easy to setup \\ 
        \b{-Merge:} Merge rows one after the other \\ higher starting cost,
        good for bigger tables \\ \b{-Hash:} Hash the row then compare to other row\\ equality check only , high starting cost, low execution cost}\\
        \hl
        \mc{PG planner scans: Full,Index,Index Only, Bitmap \\ \b{-full} scans the entire table \\ \b{-index} scans index and more (if necessary) \\ \b{-index only} only scans index \\ 
        \b{-Bitmap} scans the bitmap generated by an index.}\\
        \hl
        \mc{\b{The steps of optimization}}\\
        \hl
        \mc{1. generate the plan of transaction \\ 2. reform the term to optimize performance \\ without knowledge of the internal structure. \\ -> all values are considered equal \\
        3. optimization based on: available indexes, analysis \& costs \\ 4. generate all possible plans to calculate cost \\ 5. analyze said plans -> how many tuples, what kind... \\ 6. profit?}\\
        \hl
        \mc{\b{selectivety} this is the ratio of tuples a query returns \\ low selectivety would mean high number of rows \\ an example is select * from table where sex='Male'
        \\ the opposite would be high selectivety.}\\
        \hl
        \mc{\b{density} this is the ratio of duplicates a query returns \\ the more duplicates the higher the density. \\ you can therefore also make graphs about the distribution \\ 
        of density -> names a-c low g-l high}\\
        \hl
        \mc{\b{best practices:} index only when selectivety over 10\% \\ numeric comparison over text, join over subquery, \\use short attributes, understand the query -> user, \\ don't select * from,
        don't use cross products.}\\
        \hl
        \mcc{\b{distributed database systems}}\\
        \hl
        \mc{the advantages and disadvantages are obvious: \\ + better performance       -- more complex \\ +better reliability      + better management}\\
        \hl
        \mc{\b{homogeneous database system:}\\ > all nodes have identical software \\> all nodes know about each other and work together \\> shows itself to the user as one big system }\\
        \hl
        \mc{\b{heterogeneous database system:}\\> nodes can have different software \\> as well as different schemas -> can lead to problems \\> nodes might NOT know about each other.}\\
        \hl
        \mc{heterogeneous systems can be federated or unfederated \\ > unfederated -> no local users \\ > federated -> either tightly coupled with global schema \\ or loosely coupled using export schema}\\
        \hl
      \end{tabular}
    \end{tabular}
    \hfill
    \end{tiny}
  \end{flushleft}
  \end{table}
  \clearpage
    \begin{table}[ht!]
      \begin{flushleft}
        \begin{tiny}
          \begin{tabular}{lll}
            \begin{tabular}[t]{|l|}
              \hl
              \mc{\b{fragmentation:} this is the splitting of schemas \\ into multiple Nodes -> table 1 in node1 \\ table 2 in node 2.}\\
              \hl
              \mc{In Psql the horizontal fragmentation happens in 3 ways: \\ > 1 or more attributes for "partitioning key" \\ > "list"  explicit designation \\ > hash function (ex: Modulo)}\\
              \hl
              \mc{In Graph stores this would be called "sharding" \\ an example for this is the MongoDB \\ horizontal partitioning and allocation within a single node}\\
              \hl
              \mc{\b{replication:} this is the duplication of data in schemas \\ this means table 1 might be on node1 and 2. \\ \b{vertical -> splitting of columns} \\ row1 in node1, row2 in node2\\ 
              \b{horizontal -> splitting of rows} \\ part of column in node1 part of column in node2} \\
              \hl
              \mc{\b{unidirectional:} Single-Master}\\
              \hl
              \mc{\b{bidirectional:} Multi-Master}\\
              \hl
              \mc{\b{synchronous OR asynchronous}}\\
              \hl
              \mc{\b{allocation:} this is the distribution of work to the nodes \\ node1 might handle query or part of query1 \\ while node2 handles something else.}\\
              \hl
              \mc{as already stated above, the user doesn't see anything about \\ fragmentation or similar, the user simply interacts \\ with the schema and executes transactions. \\ these \b{transactions are 
              always local} \\ This means that the user will talk to \b{Transaction Manager}\\ that will handle the transaction and call the necessary \\functions on the \b{Resource Managers} (inside nodes).}\\
              \hl
              \mc{\b{Two Phase Commit Protocol}}\\
              \hl
              \mc{\fpic{220608-13} \\ \pic{220608-14}}\\
              \hl
              \mc{\b{case TM failed/restart:} \\ > if the TM crashes before the commit message -> abort \\ > if the TM crashes after RM respond ready -> block RM \\ this is one of the main problems btw...}\\
              \hl
              \mc{\b{case RM failed/restart:} \\ > if no entry in log, RM aborts \\ > if READY-Entry available -> RM asks TM what to do. \\ > if COMMIT-Entry available -> RM redoes transaction}\\
              \hl
              \mc{\b{case message dropped:} \\ > if the prepare statement gets lost,\\ or the RM doesn't respond, then the TM\\ simply aborts the transaction for all. \\ 
              > if RM doesn't get a response in READY state \\ then the RM will remind the TM until it gets one.}\\
              \hl
              \mc{There are different 2PC protocols, PSQL, MYSQL, X/Open, \\ Java Transaction API, Orace Transaction Manager, Microtroll \\ + often used / proven system   + guarantees ACID \\
              -- slow -- blocks transactions often \\ !!only use this when the complexity calls for it!!}\\
              \hl
              \mc{\b{Query handling:} \\there are 3 ways of handling a queries distributed databases:\\ > Ship Whole -> run query locally and remove duplicates \\ > Push Down -> split query when needed \\ 
              > Fetch-as-needed -> Send join attributes to the correct node}\\
              \hl
              \mcc{\b{NoSQL: Key/Value Stores}}\\
              \hl
              \mc{Why even NoSQL? -> fast, lots of data, web-based, scalable \\ RDBMS -> relational has the problem of complexity\\ it doesn't integrate seamlessly into programming languages \\ it therefore
              doesn't make sense to use this as \\ something like a webstore. -> !!! OR MAPPING !!!}\\
              \hl
              \mc{\b{attributes of NoSQL:} \\ > easy to use API (http)    > made for big volumes of data \\ > provided as cloud storage   > not relational, schema free \\ > BASE instead of ACID   
              > no query norm (other than SQL)}\\
              \hl
              \mc{\b{The Cap Theorem}\\ The CAP theorem dicates, that you can only have \\ 2 of 3 desired traits of a database, these include \\ \b{consistency}, \b{parition tolerance} and \b{availability}.\\
                traditional databases have the \b{ACID} philosophy which is both \\ \b{consistent and tolerant}, however it often blocks transactions. \\ here comes \b{NoSQL with the BASE} philosophy \\ 
              it is both \b{available} at all times and \b{tolerant},\\however it is \b{NOT consistent during transactions}. \\ Only after those have stopped will the system get consistent. \\
              >tolerance means the system will work despite partial outage<}\\
              \hl
              \mc{\b{BASE THEOREM}\\ >> \b{Basically Available:} \\The database will not block transactions \\ any and all requests will be responded to (can still fail though!) \\
              >> \b{Soft state:} \\ The state of the system can change even without input, \\ therefore we consider this to be a soft (not fixed) state \\ >> \b{Eventual consistency:}\\ The system will become
              consistent after end of transactions.\\ }\\
              \hl
            \end{tabular}
            \begin{tabular}[t]{|l|}
              \hl
              \mc{\fpic{220608-15}}\\
              \hl
              \mc{notable examples: Riak, Redis, Memcached, Berkeley DB \\ HamsterDB , Amazon DynamoDB, Project Voldemort}\\
              \hl
              \mc{Key/Value Stores are usually used for this: \\ >> Storing Session Information \\ >> User Profiles, Preferences, Configs \\ >> Shopping carts (LMAO) \\ They are \b{NOT} to be used in these cases: 
              \\ << relationship among data , multioperation transactions \\ << query by data, operation by sets}\\
              \hl
              \mcc{\b{NoSQL Aggregation database}}\\
              \hl
              \mc{> Arrays (Array Store) \\ > Dictionaries (Key/Value Stores) \\ > nested Structures(Document-databases)}\\
              \hl
              \mcc{\b{NoSQL Document Store}}\\
              \hl
              \mc{>> Maps a key to a structured Document \\ >> flexible schema , Document stored in \b{JSON} or BSON\\ >> Examples: \b{MongoDB}, CouchDB \\ term comparison to RDBMS \\ 
              \b{instance -> instance/database, table -> collection }\\ \b{row -> Document , rowid -> \_id/objectId} }\\
              \hl
              \mcc{\b{MongoDB}}\\
              \hl
              \mc{ >> JSON, partitioning via sharding, FOSS :) \\ >> own query language, Document Store (no Schema)}\\
              \hl
              \mc{\fpic{220608-16} \\ \pic{220608-17}} \\ 
              \hl
              \mc{\b{The documents are actually stored as BSON} \\ \b{the binary version of JSON with a bit more data.} \\ Also there are certain restriction on naming...}\\
              \hl
              \mc{\fpic{220608-18} \\ \pic{220608-19} \\ \pic{220608-20}}\\
              \hl
            \end{tabular}
            \begin{tabular}[t]{|l|}
              \hl
              \mcc{\b{sharding}}\\
              \hl
              \mc{\b{horizontal fragmentation into nodes} \\ reduces IO due to outsourcing to other servers}\\
              \hl
              \mc{\b{The shard key -> fragmentation key should be:} \\ >> distributed equally in the data \\ >> is a field/function or a combination of fields \\ 
              should be not unique, but can be a hash on PK?....\\
              default is hash on \_id aka PK\\ shards can usually not be changed -> redo database.}\\
              \hl
              \mc{MongoDB only offers few constraints like \\ the unique index constraint.}\\
              \hl
              \mc{\b{advantages:}\\ >> simple query language, >> no schema \\ \b{disadvantages:} \\ >> no constraints \\ >> no real joins (workaround with lookup)  \\ >> slow/questionable transactions -> BASE \\
              >> security -> default open in the web}\\
              \hl
              \mc{\b{MongoDB Transactions}}\\
              \hl
              \mc{updates work similar to increasing an array \\ it even automatically pushes it to a new document \\ if the current one is too big! \\ however you can avoid relocation by using \\ 
              referencing instead of embedding.}\\
              \hl
            \mc{\b{Performance hindering features:} \\ >> Atomicity of writes | >> Document Growth \\ >> Sharding | >> Indexes and Capped Collections  }\\
              \hl
              \mc{MongoDB works with \b{Replica Sets} \\ child node has the data as well \\ in case of outage you can still get your data \\ due to the nature of MongoDB this happens \\ without having
              a specific distributed database.\\ Also the nodes will pick a new Master if the\\ old one goes down.}\\
              \hl
            \mc{\b{embedded:} ALL IN ONE FILE \\ Preferred for >> ONE-TO-ONE \\ >> ONE-TO-MANY without many overlaps \\ Warning with this method, uncontrolled growth of document}\\
              \hl
              \mc{\b{referenced:} Stored in multiple files \\ Preferred when dealing with complex relationships \\ this is the advantage Document stores have over key/value! \\ example of reference: \\ 
              \pic{220608-22} \\ \pic{220608-23}} \\ \b{and here how to create it:} \\ \pic{220608-24}\\
              \hl
              \mc{in general, only use references for things that make sense, \\ for example don't make a reference for a blog post \\ or for a persons gender etc. \\
              embed the small stuff, reference the big stuff}\\
              \hl
              \mc{\fpic{220608-25} \\ \pic{220608-26}\\ \pic{220608-27}}\\
              \hl
            \end{tabular}
          \end{tabular}
        \end{tiny}
      \end{flushleft}
    \end{table}
  \clearpage
    \begin{table}[ht!]
      \begin{flushleft}
        \begin{tiny}
          \begin{tabular}{lll}
            \begin{tabular}[t]{|l|}
              \hl
              \mcc{\b{MongoDB CRUD}}\\
              \hl
              \mc{\fpic{220608-28} \\ \pic{220608-29} \\ \pic{220608-30} \\ \pic{220608-31} \\ \pic{220608-32} \\ \pic{220608-33}}\\
              \hl
              \mcc{\b{NoSQL Graph Stores}}\\
              \hl
              \mc{made for compact objects with typed aggregations \\ sometimes without schema -> implicit schema \\ or not uniform data \\ 
                \b{The idea of this database is that everything} \\ 
              \b{is an aggregation, --> hence graphs.} \\ Notable examples are: \b{Neo4J}, FlockDB, InfiniteGraph \\ biggest gaining database system right now!}\\
              \hl
              \mc{\b{Neo4J} \\ java.... ok fuck this trash \\ GPL3, embedded, disk based \\ \b{transaction based on MVCC, scalable} \\
              \b{delarative}, SQL like, pattern matching (SPARQL) \\ Collection like Haskell/Python}\\
              \hl
              \mc{\fpic{220609-2} \pic{220609-1}}\\
              \hl
            \end{tabular}
            \begin{tabular}[t]{|l|}
              \hl
              \mc{\fpic{220609-3} \\ after we match to a variable called email \\ we check for a chain of "replied\_to" \\ aka we check at least one and max 4 \\ 
              then we get the person who sent \\ the mail and return the depth - 1 \\ also note that the arrow moves to the left \\
            meaning the relation is towards the matching node \\ --> away from node <-- towards node}\\
              \hl
              \mc{\b{Node:} \\ >> set of labels (usually 1)\\ >> set of properties} \\ \b{Edge:} \\ >> Name , >> Node->Node direction \\ >> set of properties \\ \b{Property:} key/value of primitive types "name:Dashie" 
              \\ \b{types:} Node, Edge, path, map, list, integer, floating point\\ boolean and string\\
              \hl
              \mc{\b{syntax:}\\ (var:Node)-[var:edge]->(var:Node)\\ (var:Node)<-[var:Edge]-(var:Node) \\ var may be empty but the ':' has to be written!!\\ -> [:ALIAS\_OF] \\ you also get the properties of a node by just
              \\ typing in a variable name -> (varsInsideNode)\\ could be 1 variable, could be a list}\\
              \hl
              \mc{\b{Joins:} Because we can match multiple 'queries' \\ we don't have joins, instad we just \\ MATCH (query1) , (query2) <-Note the comma.}\\
              \hl
              \mc{\b{uses for Neo4J:} connected data, routing, dispatch, \\ location based services and recommendation engines. \\ \b{when not to use:} few egdes/interactions, update on all nodes \\ big data.. ??nice word}\\
              \hl
              \mcc{NoSQL Column 'Family' store}\\
              \hl
              \mc{These databases store a full row, within\\ a single cell of a column \\ |1 | Dashie | 85 | f| -> |1|name:Dashie level:85 sex:f } \\ the | symbol is to seperate columns\\
              \hl
              \mc{Notable column family stores: \\ BigTable (Google), HBase (Apache), Hypertable,\\ Cassandra (Apache) PNUTS (Yahoo)}\\
              \hl
              \mcc{\b{Cassandra}}\\
              \hl
              \mc{>> Keyspace: aka database within a Node \\ >> Node: data storage -> like document for MongoDB \\ >> Data Center: Collection of nodes \\ >> Cluster: collection of data centers  }\\
              \hl
              \mc{ \fpic{220609-6} \\ \pic{220609-4} \\ \pic{220609-5} \\ \pic{220609-7}}\\
              \hl
            \end{tabular}
            \begin{tabular}[t]{|l|}
              \hl
              \mc{\fpic{220609-8} \\ \pic{220609-9}}\\
              \hl
              \mc{\b{Use cases:} \\>> Event logging (ex: error logging) \\ good since cassandra scales with writes. \\ >> Content management systems, Blogging platforms \\ -> good for handling post entries. \\ 
              >> counters in web applications (ex: count visitors)}\\
              \hl
              \mc{\b{when not to use:} \\ >> ACID requirement \\ >> no aggregate functions available \\ >> prototypes, query changes can take time.}\\
              \hl
              \mcc{\b{Column Store}}\\
              \hl
              \mc{\b{OLTP:} Online Transaction Processing \\ >> low data volume | >> High volume of transactions \\ >> Typically normalized data | >> ACID compliance \\ >> high availability || >> \b{relational db (psql)} 
              \\ >> data management optimized}\\
              \hl
              \mc{\b{OLAP:} Online Analytical Processing \\ >> high data volume | >> low transaction volume \\ >> denormalized data | >> \b{not necessarily} ACID compliant \\ >> usually don't need high availability} \\ 
              >> read optimized system\\
              \hl
              \mc{Usually these two are separated, but recently there \\ have been efforts to combine those two \\ just like BASE and ACID are getting closer.}\\
              \hl
              \mc{\fpic{220609-10}}\\
              \hl
              \mc{\b{encoding and compression} \\ in databases it is the default to \\ \b{compress data when you have high selectivety }\\ -> male / female \\ instead of storing male/female in every column or row \\
              we can instead compress this attribute in order to save space \\ \pic{220609-11} \\ the way this is done is by storing \\ 
              the possibilities, aka male/female in another table. \\ This means that we can then reference the id of male or female.\\ 
              \b{Note that this is all done implicitly,}\\ \b{nothing has to be defined.} \\ \pic{220609-12} }\\
              \hl
              \mc{\b{Column store advantages:} \\ >> better compression | >> less disk usage \\ >> automatic indexing | >> late materialization
              \\ \b{Column stores disadvantages:}\\ >> bad writes | >> \b{!!! don't Select * from !!!} | >> \b{OLAP} }\\
              \hl
              \mc{\b{Row stores advantages:} \\ >> easy partitioning and parallization \\ >> good writes | >> seq row scans \\ \b{Row stores disadvantages:} \\ 
              >> column read == read all rows... | >> more storage \\ >> early materialization | \b{OLTP}}\\
              \hl
              \mc{\b{late vs early materialization} \\ early means that everything has to be read first \\ while late means that we wait until the end of the query. \\ 
              return just the column with nothing else.}\\
              \hl
              \mcc{\b{NoSQL In Memory Stores (IMDB)}}\\
              \hl
              \mc{this database stores its values \b{inside the RAM} \\ However, it still stores \b{regular Snapshots} of its system \\ on the regular disk, in order to prevent
              data loss. \\ It also creates logs, to restore previous\\ states in case of a shutdown. Small note: the \b{pagefile}\\ is also loaded from the disk on startup}\\
              \hl
              \mc{\b{advantages:} >> fast read | >> ACID \\ >> indexing on the fly \\ >> no buffer for write to disk -> happens with snapshots \\ less power requirement due to less software\\ 
              \b{disadvantages:} >> small storage...\\ >> the values are compressed to save storage -> slow writes}\\
              \hl
            \end{tabular}
          \end{tabular}
        \end{tiny}
      \end{flushleft}
    \end{table}
  \clearpage
    \begin{table}[ht!]
      \begin{flushleft}
        \begin{tiny}
          \begin{tabular}{lll}
            \begin{tabular}[t]{|l|}
              \hl
              \mcc{\b{MonetDB and NoSQL end}}\\
              \hl
              \mc{IMDB examples: SAP HANNA, Oracle TimesTen,\\ IBM SoliDB, \b{MonetDB}}\\
              \hl
              \mc{>> \b{Column oriented Database} | >> FOSS, written in C \\ >> \b{SQL,C,R scripts on database itself} \\ >> supports (application): C,C++,Java,Javascript,\\Perl,PHP,Python,R,Rust \\  
              >> \b{used for Data mining, analysis and text search} \\ >> every column mapped to 1 file | >> columns are compressed \\ >> columns are automatically indexed \\ >> multi-core parallel execution of queries \\
              >> data-manipulation language SQL 2003,\\ sql Explain, Xquery XML \\ >> \b{OLAP} | >> no secondary indexes (only automatic ones) \\ >> \b{ACID/Optimistic Concurrency Control} -> OCC \\ 
              >> 100x faster than psql in TPC-H benchmark -> readonly }\\
              \hl
              \mc{so then, why not use NoSQL all the time? \\ \b{!NoSQL lacks multimodel databases!} \\ always just one -> document, key/value, graph, etc}\\
              \hl
              \mc{\b{Database benchmarking:} >> TPC.org, pgbench (psql) \\ >> timing | >> data usage | >> query types | >> data amount..}\\
              \hl
              \mc{\b{execution models:} \\ >> \b{volcano execution model:} row by row -> ex:psql \\ \b{vectorized execution model:} value by value ->MonetDB}\\
              \hl
              \mcc{\b{Database as a Service (DBAAS)}}\\
              \hl
              \mc{Providers: Amazon AWS, Microsoft Azure, Google Cloud, ..}\\
              \hl
              \mcc{\b{Pagnation Paging}}\\
              \hl
              \mc{this is essentially just an offset -> where x > 10000}\\
              \hl
              \mcc{\b{GraphQL}}\\
              \hl
              \mc{>> declarative query language over the web \\ >> schema with types | >> description of data via API \\ >> one endpoint for interaction \\ >> easy to develop API | >> clients can request data \\
              >> \b{Query and Data manipulation language (DML)} \\ >> \b{Data-Type-System (DDL)} \\ >> \b{supported by all big databases (psql)} \\ >> results represent JSON docs but uses edges and nodes}\\
              \hl
              \mc{In the end GraphQL handles input from the user \\ towards a database like PSQL or others \\ it is in itself not a database!!}\\
              \hl
              \mc{\fpic{220609-13}}\\
              \hl
              \mc{\b{schema-first vs code-first:} \\ schema first allows for better and faster integration \\ however it is obviously more complicated to execute.}\\
              \hl
              \mcc{\b{PostGraphile: GraphQL and PSQL}}\\
              \hl
              \mc{>> library for integrating PSQL into GraphQL \\ >> PSQL database, NodeJS http-server, GraphQL webAPI \\ 
              >> integrates GraphiQL (GUI tool to write queries) \\ schema first!}\\
              \hl
              \mc{\fpic{220609-14} \\ and here the query as an example: \\ \pic{220609-15}}\\
              \hl
              \mc{>> the default creation of createUser etc can be disabled \\ you can also redefine what the table returns on query\\
              ex: user , userByRowId, allUsers, allUsersList \\ keep in mind that the first 2 return 1 user\\ the other return a list \\ the third one can also be used to \\ 
              get more info about the nodes and egdes \\ see "pagnierung above"}\\
              \hl
              \mc{\b{best practices:} >> don't create nullable fields \\ >> use mutations -> convert long types to mutations}\\
              \hl
              \mc{\b{GraphQL vs SQL} >> GrapQL is very limited \\ >> controlled | >> expandable}\\
              \hl
            \end{tabular}
            \begin{tabular}[t]{|l|}
              \hl
              \mc{here an example for CRUD functions \\ \fpic{220609-16}}\\
              \hl
              \mcc{\b{OR MAPPER}}\\
              \hl
              \mc{Application -> Persistance API -> JPA Provider -> \\ JDBC API -> JDBC Driver -> Database}\\
              \hl
              \mc{>> start with @ entity | >> inheritance and interfaces possible \\>> can be astract | >> no constructor arguments (unless private) \\ >> needs primary key |>> class may not be final \\
              >> fields should be private or protected | >> Read commited!}\\
              \hl
              \mc{\fpic{220609-17} \\ \b{the red line are optional, if the names in} \\ \b{the java object and in the db are the same !!}\\ supported datatypes: primitive types, wrapper types \\
              strings, enums, date, serializable classes\\ supported relations: collection, set, list, map of entities \\ 
              }\\
              \hl
              \mc{\b{execute statements} \\ \pic{220609-18} \\ !!this is the base, we always need to call this!! \\ note that entity managers are independent \\ 
              comparing the same result from 2 entity managers \\ will not be considered equal}\\
              \hl
              \mc{\b{insert} \\  \pic{220609-19}}\\
              \hl
              \mc{\b{delete} \\ \pic{220609-20}}\\
              \hl
              \mc{\b{update} \\ \pic{220609-21}}\\
              \hl
              \mc{\b{query} \\ \pic{220609-22}}\\
              \hl
              \mc{further type details can be specified as such: \\ \pic{220609-23}}\\
              \hl
              \mc{\b{Field Access \& Property Acess} \\ in OR mapping you don't need to specify setId or get Id \\ they are automatically created. -> Field Access \\ Should you still create them -> Property Access
              \\ for field access the @id must be above the variable declaration \\ for the property access must to be above the getID() function.}\\
              \hl
              \mc{\b{order of mapping:}\\ XML / user defined mapping / default mapping}\\
              \hl
              \mc{\b{One-To-One} \\ \pic{220609-24}}\\
              \hl
              \mc{\b{Bi-One-To-One} \\ this is an \b{optional ADDITION} to above \\ it just creates a manual mapping for the bankcustomer \\ \pic{220609-25}}\\
              \hl
              \mc{\b{Inverse One-To-One} \\ \pic{220609-26}}\\
              \hl
              \mc{\b{For all many to x} \\ \pic{220609-28} \\ instead of private something = new ..}\\
              \mc{\b{Many-To-One} -> change the @OneToOne with @ManyToOne} \\
              \hl
              \mc{\b{Bi-Many-To-One}\\ -> change the second @OneToOne with @OneToMany }\\
              \hl
            \end{tabular}
            \begin{tabular}[t]{|l|}
              \hl
              \mc{\b{Many-To-Many} \\ \pic{220609-27}}\\
              \hl
              \mc{\b{Bi-Many-To-Many} -> Bi-OneToOne with @ManyToMany}\\
              \hl
              \mc{\b{Fetch Types} \\ Eager -> default OneToOne/ManyToOne entity loaded directly \\ Lazy -> default OneToMany/ManyToMany load only on access \\ @relation(fetch = FetchType.LAZY/EAGER for override)}\\
              \hl
              \mc{\b{O/R Mapping Variants} \\ >> Top Down Code first | >> Bottom up DB first \\ >> Inside-out Mapping first | >> Outside in DB Code together}\\
              \hl
              \mc{\fpic{220609-30}}\\
              \hl
              \mc{\b{Generated Ids} \\ \pic{220610-1} \\ This is the same as SERIAL in SQL \\ \pic{220610-2} \\ \pic{220610-3}}\\
              \hl
              \mc{There is also the option for cascade, both persist and delete \\ \pic{220610-4}  
              \pic{220610-5}}\\
              \hl
              \mc{\b{Aggregation and Composition} \\ \pic{220610-6}} \\ \pic{220610-7}\\
              \hl
              \mc{for bidirectional relation,\b{MANUALLY sync ANY updates}.\\they are NOT automatic!!} \\ 
              \hl
              \mc{\b{inheritance strategy} \\ Single Table \\ \pic{220610-8} \\ Joined Table \\ 
              \pic{220610-9} \\ Multi Table \\ \pic{220610-10} \\ child tables (multi table without 
              discriminator) \\ \pic{220610-11}}\\
              \hl
              \mc{\b{JPQL} Query language optimized for OR} \\ essentially the same other than this \\
              \pic{220610-12}\\
              \hl
              \mc{dynamic queries: \\ \pic{220610-13}}\\
              \hl
              \mc{Predefined query \\ \pic{220610-14}}\\
              \hl
              \mc{Locktype: either pessimistic or optimistic \\ \pic{220610-15}}\\
              \hl
              \mc{merge, add to existing or create new \\ \pic{220610-16}} \\
              \hl
              \mc{refresh entity manager \\ \pic{220610-17}}\\
              \hl
              \mc{JSON to PSQL in seperate file. OPTIONAL!}\\
              \hl
            \end{tabular}
          \end{tabular}
        \end{tiny}
      \end{flushleft}
    \end{table}
  \end{flushleft} 
\end{document}
