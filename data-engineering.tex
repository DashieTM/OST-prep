\documentclass{article}
\usepackage[left=0mm, right=0mm, top=0mm, bottom=0mm]{geometry}
\usepackage[document]{ragged2e}
\usepackage{graphicx}
\usepackage{array, makecell}
\usepackage{FiraMono}
\usepackage{tabulary}
\newcommand{\mc}{\makecell[{{l}}]}
\newcommand{\mcc}{\makecell[{{c}}]}
\newcommand{\mcr}{\makecell[{{r}}]}
\renewcommand{\b}{\textbf}
\newcommand{\pic}{\includegraphics[scale=0.3]}
\newcommand{\hl}{\hline}
%\renewcommand{\arraystretch}{3.5}
\graphicspath{{./Pictures/}}
\begin{document}
\begin{flushleft}
  \begin{table}[ht!]
  \begin{flushleft}
    \begin{tiny}
      \begin{tabular}{lll}
        \begin{tabular}[t]{|l|}
        \hl
        \mcc{\b{Functions in PGPLSQL}} \\
        \hl
        \mc{functions in psql \\ create or replace function funcName() \\ return \b{s} returntype as \b{ \$\$ } \\ begin \\  raise notice 'Hello Birb!'; \\ end; \\ \b{\$\$ }
        language langName \\ The two \$ are always necessary. Also note the \\ returns with an s and the language at the end. \\ Which \b{MUST} be a \b{PROCEDURAL LANGUAGE}, \\
        so c++ doesn't work here. } \\
        \hl
        \mc{Parameters are handled like in any language \\ func(x bigint, y bigint) \\ you can also define multiple return types\\ func(variadic a numeric[]) \\ or a generic return \\ 
        func(param anyelement)} \\ 
        \hl
        \mc{Variable Declaration: \\ returns void as \$\$ \\ DECLARE \\ x bigint; y bigint; \\ BEGIN ....} \\ 
        \hl
        \mc{Variable manipulation: x \b{:=} 6 + 4} \\
        \hl
        \mc{if: IF n = 0 THEN RETURN 1; \\ (optional)ELSE RETURN 2; END IF; \\ ELSIF also possible (note elsif not elseif)} \\ 
        \hl
        \mc{case x when 1, 2 then msg := 'one or two'; end case; \\ essentially this checks if x is 1 \b{OR} 2}\\
        \hl
        \mc{case when x between 0 and 10 then .... \\ similar but with a range, both can be\\ simulated by if else.}\\
        \hl
        \mc{Exceptions: BEGIN z:= x / y;\\ EXCEPTION WHEN division-by-zero \\ THEN z:= 0; (or error rather) END; \\ if you want to catch all: WHEN others THEN} \\ often used after exception: RAISE;
        (show error) \\ 
        \hl
        \mc{For Loop: For var IN query LOOP \\ statements END LOOP; \\ for r in SELECT * FROM ang LOOP \\ RETURN NEXT r; END LOOP; RETURN; END; \\ note that the return next doesn't return \\ you store 
        it in a buffer \\ and return it at the end of the function. \\ for infinite loops: FOR i IN 1..\_max LOOP;} \\
        \hl
        \mc{update and insert: INSERT INTO ANG VALUES(...); \\ UPDATE ang set salary = salary + 500 \\where name = 'dashie'; \\ interestingly, after the where name = 'dashie' \\ you can use if not found
        then (handle error)\\ this allows for easier error handling.}\\
        \hl
        \mc{queries: \b{execute} 'SELECT * from ang' \\ || into result; return result; END;} \\
        \hl
        \mc{comments are done by either -- or /* */ for multiline} \\ 
        \hl
        \mc{anonymous function: you can omit \\the name and just write do \$\$ ...} \\ 
        \hl
        \mc{cursor: declare curs CURSOR FOR query; \\BEGIN OPEN curs; LOOP do something CLOSE\\ curs; END;\\ Cursors are essentially just iterables. \\ cursors can also be unbound curs1 refcursor \\
        or they can be parameterized curs3 cursor(arg)}\\
        \hl
        \mc{\pic{220607-3} \\ arrays: SELECT '{1,2,3}'::int[] \\ or SELECT ARRAY[1,2,3] \\ var int[] only in variable declaration. \\ \b{!!arrays start with 1 in psql !!}}\\
        \hl
        \mc{return types: all of the above AND void, \\ SETOF type (array of a type), TABLE, Trigger} \\
        \hl
        \mc{\pic{220608-2} \\ \pic{220608-3}}\\
        \hl
        \mc{hstore / map:\\ \pic{220608-4} \\ \pic{220608-5} \\ \pic{220608-6}\\ hstore supports GIST/GIN indexing}\\
        \hl
      \end{tabular}
      \begin{tabular}[t]{|l|}
        \hl
        \mc{\pic{220607-1} \\ also note that type\%rowtype is used like this: \\ r ang\%rowtype -> for r in select * from ang;}\\
        \hl
        \mc{important to know, you can always use these \\functions to manipulate queries, for example \\ select upper(name) from ang; \\ depending on the function you can also \\ 
        select generatetab(1,10)}\\
        \hl
        \mc{\pic{220607-2}}\\
        \hl
        \mc{cast: cast(input as type); \\ cast(record.id as text);}\\
        \hl
        \mc{stored procedures are nothing but a chaining of functions:\\ \pic{220607-4}} \\
        \hl
        \mc{some good to know things: plain SQL is more efficient. \\ write variables lower case for sql \\ use cast over typename -> not select date '2022-06-07'}\\
        \hl
        \mcc{\b{Triggers}}\\
        \hl
        \mc{\pic{220607-5}}\\
        \hl
        \mc{\pic{220607-6} \\ statement is once, row means once \\per row, aka for the entire table. \\ \pic{220607-7}} \\ \pic{220607-8} \pic{220607-10}\\ \pic{220607-9} \pic{220607-11}\\ 
        \hl
        \mc{Return types: RETURN NEW -> returns a new table/row \\ RETURN OLD -> returns the old table/row \\ (but could change other rows!) \\ RETURN NULL -> cancel operation.}\\
        \hl
        \mc{running order: before statement, before row, \\ after row, after statement -> \\ and of course alphabetically.}\\
        \hl
        \mc{inside the Trigger functions you can use the \\ variables that don't matter aka can be ANY -> user \\ or the entered user from the trigger -> NEW.user \\ or explicitly the old one ->
        OLD.user \\ and last user defined stuff like -> SELECT 'I' \\ which just places an I as the variable \\ or something like now() for timestamps.}\\
        \hl
        \mc{Triggers make the database slower and harder to maintain. \\ some databases therefore let you disable them if you want.\\ On a table basis. \\ Also watch out for cascading effects of 
        triggers \\ they might cause something else to be deleted.}\\
        \hl
        \mc{Stored Procedures are really helpful for security \\ They have all the prviliges, but only \\ allow the user to do what the \\ creator has predefined.}\\
        \hl
      \end{tabular}
      \begin{tabular}[t]{|l|}
        \hl
        \mc{\pic{220608-1}}\\
        \hl
        \mcc{\b{Optimization and indexing}} \\ 
        \hl
        \mc{The basics of indexing is that is saves \\ time on queries, but it uses more space, \\ 
        and needs to be redone on udpate/insert.}\\
        \hl
        \mc{Data stored in Pages / Heap (Collection of Pages) \\ Indexing on either can be possible. \\
        ex: Page index -> Primary key index\\ ex: Heap Index -> index on tables \\ PSQL does table 
        cluster indexing \\ instead of integrated indexing -> key value (both indexed)}\\
        \hl
        \mc{\b{B-Tree:} the default, can index multiple entries (only btree!) \\ CREATE UNIQUE INDEX name ON table (column [1,2...])}\\
        \hl
        \mc{\b{Hash Index:} just like hashmap in programming\\ good for single or small multiple queries \\
        bad for entire tables etc -> collision \\ hashing might take a long time with a lot of data. \\ B-tree almost always better!}\\
        \hl
        \mc{\b{GiST:} balanced/treelike , Range/neighbor/fulltext search \\ used for geometric datatypes. SP-Gist for unbalanced trees.}\\
        \hl
        \mc{\b{GIN:} General Inverted Index "list of words \\ that point to documents" wtf?. Good for duplicates. \\ Good for hstore,Json,Arrays \\ \pic{220608-12}}\\
        \hl
        \mc{\b{Bitmap indexing:} Bitmap -> 0 1 stores Booleans/Enums \\ very fast read / slow update \\ in postgres \b{only implicit} use Brin instead.}\\
        \hl
        \mc{\b{BRIN:} Block Range Index, stores min/max values as blocks \\ good for range search, sorted data, small disk usage \\ data is naturally sorted, address next to postal code.}\\
        \hl
        \mc{Bloom Index -> equality search, \\Trigram Index -> Full text search \\ RUM , non-default-GIN jsonb-path-ops}\\
        \hl
        \mc{creating index:\\CREATE INDEX <indexname> ON <table(attribute)>;  \\ and: DROP INDEX <indexname>;}\\
        \hl
        \mc{default index order for psql is btree,ASC,NULL first}\\
        \hl
        \mc{\pic{220608-7} \\ \pic{220608-8}}\\
        \hl
        \mc{PG planner join strategies: Nested Loop,Merge,Hash \\ \b{-Nested Loop:} for r in right row r == for l in left row.... \\ good for small tables, easy to setup \\ 
        \b{-Merge:} Merge rows one after the other \\ higher starting cost,
        good for bigger tables \\ \b{-Hash:} Hash the row then compare to other row\\ equality check only , high starting cost, low execution cost}\\
        \hl
        \mc{PG planner scans: Full,Index,Index Only, Bitmap \\ \b{-full} scans the entire table \\ \b{-index} scans index and more (if necessary) \\ \b{-index only} only scans index \\ 
        \b{-Bitmap} scans the bitmap generated by an index.}\\
        \hl
        \mc{\b{The steps of optimization}}\\
        \hl
        \mc{1. generate the plan of transaction \\ 2. reform the term to optimize perfomance \\ without knowledge of the internal structure. \\ -> all values are considered equal \\
        3. optimization based on: available indexes, analysis & costs \\ 4. generate all possible plans to calculate cost \\ 5. analyze said plans -> how many tuples, what kind... \\ 6. profit?}\\
        \hl
        \mc{\b{selectivety} this is the ratio of tuples a query returns \\ low selectivety would mean high number of rows \\ an example is select * from table where sex='Male'
        \\ the opposite would be high selectivety.}\\
        \hl
        \mc{\b{density} this is the ratio of duplicates a query returns \\ the more duplicates the higher the density. \\ you can therefore also make graphs about the distribution \\ 
        of density -> names a-c low g-l high}\\
        \hl
        \mc{\b{best practices:} index only when selectivety over 10\% \\ numeric comparison over text, join over subquery, \\use short attributes, understand the query -> user, \\ don't select * from,
        don't use cross products.}\\
        \hl
        \mcc{\b{distributed database systems}}\\
        \hl
        \mc{the advantages and disadvantages are obvious: \\ + better performance       -- more complex \\ +better reliability      + better management}\\
        \hl
        \mc{\b{homogeneous database system:}\\ > all nodes have identical software \\> all nodes know about each other and work together \\> shows itself to the user as one big system }\\
        \hl
        \mc{\b{heterogeneous database system:}\\> nodes can have different software \\> as well as different schemas -> can lead to problems \\> nodes might NOT know about each other.}\\
        \hl
        \mc{heterogeneous systems can be federated or unfederated \\ > unfederated -> no local users \\ > federated -> either tightly coupled with global schema \\ or loosely coupled using export schema}\\
        \hl
      \end{tabular}
    \end{tabular}
    \hfill
    \end{tiny}
  \end{flushleft}
  \end{table}
  \pagebreak
    \begin{table}[ht!]
      \begin{flushleft}
        \begin{tiny}
          \begin{tabular}{lll}
            \begin{tabular}[t]{|l|}
              \hl
              \mc{\b{fragmentation:} this is the splitting of schemas \\ into multiple Nodes -> table 1 in node1 \\ table 2 in node 2.}\\
              \hl
              \mc{In Psql the horizontal fragmentation happens in 3 ways: \\ > 1 or more attributes for "partitioning key" \\ > "list"  explicit designation \\ > hash function (ex: Modulo)}\\
              \hl
              \mc{In Graph stores this would be called "sharding" \\ an example for this is the MongoDB \\ horizontal partitioning and allocation within a single node}\\
              \hl
              \mc{\b{replication:} this is the duplication of data in schemas \\ this means table 1 might be on node1 and 2. \\ \b{vertical -> splitting of columns} \\ row1 in node1, row2 in node2\\ 
              \b{horizontal -> splitting of rows} \\ part of column in node1 part of column in node2} \\
              \hl
              \mc{\b{unidirectional:} Single-Master}\\
              \hl
              \mc{\b{bidirectional:} Multi-Master}\\
              \hl
              \mc{\b{synchronous OR asynchronous}}\\
              \hl
              \mc{\b{allocation:} this is the distribution of work to the nodes \\ node1 might handle query or part of query1 \\ while node2 handles something else.}\\
              \hl
              \mc{as already stated above, the user doesn't see anything about \\ fragmentation or similar, the user simply interacts \\ with the schema and executes transactions. \\ these \b{transactions are 
              always local} \\ This means that the user will talk to \b{Transaction Manager}\\ that will handle the transaction and call the necessary \\functions on the \b{Resource Managers} (inside nodes).}\\
              \hl
              \mc{\b{Two Phase Commit Protocol} \\ \pic{220608-13} \\ \pic{220608-14}}\\
              \hl
              \mc{\b{case TM failed/restart:} \\ > if the TM crashes before the commit message -> abort \\ > if the TM crashes after RM respond ready -> block RM \\ this is one of the main problems btw...}\\
              \hl
              \mc{\b{case RM failed/restart:} \\ > if no entry in log, RM aborts \\ > if READY-Entry available -> RM asks TM what to do. \\ > if COMMIT-Entry available -> RM redoes transaction}\\
              \hl
              \mc{\b{case message dropped:} \\ > if the prepare statement gets lost,\\ or the RM doesn't respond, then the TM\\ simply aborts the transaction for all. \\ 
              > if RM doesn't get a response in READY state \\ then the RM will remind the TM until it gets one.}\\
              \hl
              \mc{There are different 2PC protocols, PSQL, MYSQL, X/Open, \\ Java Transaction API, Orace Transaction Manager, Microtroll... \\ + often used / proven system   + guarantees ACID \\
              -- slow -- blocks transactions often \\ !!only use this when the complexity calls for it!!}\\
              \hl
              \mc{\b{Query handling:} \\there are 3 ways of handling a queries distributed databases:\\ > Ship Whole -> run query locally and remove duplicates \\ > Push Down -> split query when needed \\ 
              > Fetch-as-needed -> Send join attributes to the correct node}\\
              \hl
              \mcc{\b{NoSQL: Key/Value Stores}}\\
              \hl
              \mc{Why even NoSQL? -> fast, lots of data, webbased, scalable \\ RDBMS -> relational has the problem of complexity\\ it doesn't integrate seamlessly into programming languages \\ it therefore
              doesn't make sense to use this as \\ something like a webstore. -> !!! OR MAPPING !!!}\\
              \hl
              \mc{\b{attributes of NoSQL:} \\ > easy to use API (http)    > made for big volumes of data \\ > provided as cloud storage   > not relational, schema free \\ > BASE instead of ACID   
              > no query norm (other than SQL)}\\
              \hl
              \mc{\b{The Cap Theorem}\\ The CAP theorem dicates, that you can only have \\ 2 of 3 desired traits of a database, these include \\ \b{consistency}, \b{parition tolerance} and \b{availability}.\\
                traditional databases have the \b{ACID} philosophy which is both \\ \b{consistent and tolerant}, however it often blocks transactions. \\ here comes \b{NoSQL with the BASE} philosophy \\ 
              it is both \b{available} at all times and \b{tolerant},\\however it is \b{NOT consistent during transactions}. \\ Only after those have stopped will the system become consistent. \\
              >tolerance means the system will work despite partial outage<}\\
              \hl
              \mc{\b{BASE THEOREM}\\ >> \b{Basically Available:} \\The database will not block transactions \\ any and all requests will be responded to (can still fail though!) \\
              >> \b{Soft state:} \\ The state of the system can change even without input, \\ therefore we consider this to be a soft (not fixed) state \\ >> \b{Eventual consistency:}\\ The system will eventually
              become consistent (no more transactions)\\ }\\
              \hl
            \end{tabular}
            \begin{tabular}[t]{|l|}
              \hl
              \mc{\pic{220608-15}}\\
              \hl
              \mc{notable examples: Riak, Redis, Memcached, Berkeley DB \\ HamsterDB , Amazon DynamoDB, Project Voldemort}\\
              \hl
              \mc{Key/Value Stores are usually used for this: \\ >> Storing Session Information \\ >> User Profiles, Preferences, Configs \\ >> Shopping carts (LMAO) \\ They are \b{NOT} to be used in these cases: 
              \\ << relationship among data , multioperation transactions \\ << query by data, operation by sets}\\
              \hl
              \mcc{\b{NoSQL Aggregation database}}\\
              \hl
              \mc{> Arrays (Array Store) \\ > Dictionaries (Key/Value Stores) \\ > nested Structures(Document-databases)}\\
              \hl
              \mcc{\b{NoSQL Document Store}}\\
              \hl
              \mc{>> Maps a key to a structured Document \\ >> flexible schema , Document stored in \b{JSON} or BSON\\ >> Examples: \b{MongoDB}, CouchDB \\ term comparison to RDBMS \\ 
              \b{instance -> instance/database, table -> collection }\\ \b{row -> Document , rowid -> \_id/objectId} }\\
              \hl
              \mcc{\b{MongoDB}}\\
              \hl
              \mc{ >> JSON, partitioning via sharding, FOSS :) \\ >> own query language, Document Store (no Schema)}\\
              \hl
              \mc{\pic{220608-16} \\ \pic{220608-17}} \\ 
              \hl
              \mc{\b{The documents are actually stored as BSON} \\ \b{the binary version of JSON with a bit more data.} \\ Also there are certain restriction on naming...}\\
              \hl
              \mc{\pic{220608-18} \\ \pic{220608-19} \\ \pic{220608-20}}\\
              \hl
            \end{tabular}
            \begin{tabular}[t]{|l|}
              \hl
              \mcc{\b{NoSQL Graph Stores}}\\
              \hl
              \mc{made for compact objects with typed aggregations \\ sometimes without schema -> implicit schema \\ or not uniform data}\\
              \hl
              \mc{}\\
              \hl
            \end{tabular}
          \end{tabular}
        \end{tiny}
      \end{flushleft}
    \end{table}
  \pagebreak
    \begin{table}[ht!]
      \begin{flushleft}
        \begin{tiny}
          \begin{tabular}{lll}
            \begin{tabular}[t]{|l|}
              \hl
              \mc{}\\
              \hl
            \end{tabular}
            \begin{tabular}[t]{|l|}
              \hl
              \mc{}\\
              \hl
            \end{tabular}
            \begin{tabular}[t]{|l|}
              \hl
              \mc{}\\
              \hl
            \end{tabular}
          \end{tabular}
        \end{tiny}
      \end{flushleft}
    \end{table}
  \pagebreak
    \begin{table}[ht!]
      \begin{flushleft}
        \begin{tiny}
          \begin{tabular}{lll}
            \begin{tabular}[t]{|l|}
              \hl
              \mc{}\\
              \hl
            \end{tabular}
            \begin{tabular}[t]{|l|}
              \hl
              \mc{}\\
              \hl
            \end{tabular}
            \begin{tabular}[t]{|l|}
              \hl
              \mc{}\\
              \hl
        \mc{easiest way to convert PSQL to JSON is a temporary table. \\ \pic{220608-9} \\ you can also directly \b{select json\_build\_object }\\ or \b{select json\_build\_object(jsonb\_agg(tmp))} \\ 
        also note the jsonb\_agg(trim (proj.bezeichnung)) \\ for simplicity, the joins on the picture are removed.}\\
        \hl
        \mc{The jsonb agg is necessary for aggregation \\ otherwise the join wouldn't work for JSON.}\\
        \hl
        \mc{\pic{220608-10} \\ \pic{220608-11}}\\
        \hl
        \mcc{\b{JSON | SQL}}\\
        \hl
            \end{tabular}
          \end{tabular}
        \end{tiny}
      \end{flushleft}
    \end{table}
  \end{flushleft} 
\end{document}

