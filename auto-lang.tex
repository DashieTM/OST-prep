\documentclass{article}
\usepackage[left=-1.5mm, right=-1.5mm, top=-1mm, bottom=-1.5mm]{geometry}
\usepackage[document]{ragged2e}
\usepackage{graphicx}
\usepackage{pict2e}
\usepackage{amsmath, mathtools, nccmath, amssymb}
\usepackage[dvipsnames]{xcolor}
\usepackage{array, makecell}
\usepackage{moresize}
\usepackage[T1]{fontenc}
\usepackage{noto-sans}
\renewcommand\cellalign{cl}
\renewcommand\theadalign{lc}
\newcommand{\ns}{\\ \vspace{-0.03in}}
\newcommand{\mc}[1]{\makecell[cl]{#1}}
\newcommand{\mcc}{\makecell[{{c}}]}
\newcommand{\mcr}{\makecell[{{r}}]}
\newcommand{\pic}{\includegraphics[scale=0.3]}
\newcommand{\form}[1]{\vspace{-0.03in}\\ \small{\textcolor{Red}{\(#1\)}} \\ \vspace{-0.03in}}
\newcommand{\forms}[1]{\vspace{-0.01in}\\ \scriptsize{\textcolor{Red}{\(#1\)}\\ \vspace{-0.01in}}}
\newcommand{\noskipform}[1]{\small{\textcolor{Red}{\(#1\)}}}
\newcommand{\fpic}{\vspace{-0.05in} \\ \includegraphics[scale=0.3]}
\newcommand{\dx}[1]{\scriptsize{\textcolor{Red}{\(\dfrac{d}{dx}#1\)}}}
\newcommand{\dxs}[1]{\ssmall{\(\dfrac{d}{dx}#1\)}}
\newcommand{\dxo}{\dfrac{d}{dx}}
\newcommand{\hs}{\hspace{0.5in}}
\renewcommand{\b}{\textbf}
\renewcommand{\r}[1]{\textcolor{Red}{#1}}
\newcommand{\blu}[1]{\textcolor{Blue}{#1}}
\newcommand{\gre}[1]{\textcolor{Green}{#1}}
\newcommand{\yel}[1]{\textcolor{Yellow}{#1}}
\newcommand{\ora}[1]{\textcolor{Orange}{#1}}
\newcommand{\lims}[1]{\lim\limits_{#1}}
\newcommand{\atb}[1]{\int_a^b #1 \,dx}
%\renewcommand{\arraystretch}{1.5}
\graphicspath{{./Pictures/}}
\makeatletter
\DeclareRobustCommand{\bigplus}{%
  \mathop{\vphantom{\sum}\mathpalette\@bigplus\relax}\slimits@
}
\newcommand{\@bigplus}[2]{\smash{\vcenter{\hbox{\make@bigplus{#1}}}}}
\newcommand{\make@bigplus}[1]{%
  \sbox\z@{$\m@th#1\sum$}%
  \setlength{\unitlength}{\wd\z@}%
  \begin{picture}(1.4,1.4)
  %\roundcap
  \linethickness{.17ex}
  \Line(.7,.14)(.7,1.26)
  \Line(.14,.7)(1.26,.7)
  \end{picture}%
}
\DeclareRobustCommand{\bigtimes}{%
  \mathop{\vphantom{\sum}\mathpalette\@bigtimes\relax}\slimits@
}
\newcommand{\@bigtimes}[2]{\vcenter{\hbox{\make@bigtimes{#1}}}}
\newcommand{\make@bigtimes}[1]{%
  \sbox\z@{$\m@th#1\sum$}%
  \setlength{\unitlength}{\wd\z@}%
  \begin{picture}(1,1)
  %\roundcap
  \linethickness{.17ex}
  \Line(.1,.1)(.9,.9)
  \Line(.1,.9)(.9,.1)
  \end{picture}%
}
\makeatother
\begin{document}
    \begin{table}[ht!]
        \begin{normalsize}
          \begin{tabular}{cc}
          \begin{tabular}[t]{|@{\hskip1pt}p{10.475cm}|}
              \hline
              \mc{\r{Predicate:}\\ 
              a mathematical predicate can be True or False.\\
              predicates are functions with boolean return values:\\
              P, Q(n), R(x,y,z)
              }\\
              \hline
              \mc{\r{Logical Operators:}\\
              AND: \(P \land Q\) || OR: \(P \lor Q\) || NOT: \(\lnot P\)  \\ 
              Implication: \(P \implies Q = \lnot P \lor Q\)\\
              }\\
              \hline
              \mc{\r{Distributive Rule}\\
              \(P \land (Q \lor R) = (P \land Q) \lor (P \land R)\)\\
              \( P \lor (Q \land R) = (P \lor Q) \land (P \lor R) \)\\
              }\\
              \hline
              \mc{\begin{tabular}{ll}
              \mc{\hspace{-0.08in}\r{De Morgans Law}\\
              \(\lnot (P \land Q) = \lnot P \lor \lnot Q\)\\
              \(\lnot (P \lor Q) = \lnot Q \implies \lnot P\)\\
              \(P \implies Q = \lnot Q \implies \lnot P\)\\
            }\hspace{-0.09in} & \vline 
              \mc{\r{Implication}\\
                \(P \implies Q = True\) \& \(\lnot P \implies Q = True\)\\
              \(P \implies \lnot Q = False\) \\
              \(\lnot P \implies \lnot Q = True\)
              }
              \end{tabular}
              }\\
              \hline
              \mc{\r{Quantors}\\
              \begin{tabular}{cc}
              \mc{\r{OR:} \( \bigvee_{k=0}^n P_k\)\\
              P true for any \(k \in {0 .. n}\)}&
              \mc{\r{AND:} \(\bigwedge_{k=0}^n P_k\)\\
              P true for all \(k \in n\)}\\
              \mc{\r{All:} \(\forall k \in {0 .. n} = P_k \)\\
              for all k P = True}&
              \mc{\r{Exists:} \(\exists k \in {0 .. n} = P_k\)\\
              a k exists where P = True
              }\\
              \end{tabular}
              }\\
              \hline
              \mc{\r{Normalforms}\\
              \begin{tabular}{cc}
              \mc{\r{disjunctive}\\
              \((x1 \land x2 ) \lor (\overline{x1} \land x2) \lor (x1 \land \overline{x2})\)
              }&
              \mc{\r{conjunctive}\\
              \((x1 \lor x2) \land (\overline{x1} \lor x2) \land (x1 \lor \overline{x2})\)
              }\\
              \end{tabular}\\
              \footnotesize{These are useful for true and false tables}\\
              \footnotesize{This one would result to true if x1 or x2 is true.}\\
              }\\
              \hline
              \mc{\r{Quantities:}\\
              \(\emptyset = \{\}\) \hs \([n] = \{0 .. n\}\) \hs \{a .. z\}\\
              \begin{tabular}{ll}
                \mc{\r{Union:}\\ \(A \cup B = \{x | x \in A \lor x \in B\}\)} &
                \mc{\r{Intersection:}\\ \(A \cap B = \{x | x \in A \land x \in B\}\)}\\
                \mc{\r{Complement:} \\ \(\overline{A} = \{x | x \notin A\}\)}&
                \mc{\r{Difference:} \\ \(A \setminus B = \{x \in A | x \notin B\}\)}
              \end{tabular}
              }\\
              \hline
              \mc{\r{Pairs}
                \(A \times B = \{(a,b) | a \in A \land b \in B\}\)
              }\\
              \hline
              \mc{\r{n-Tuples}
              \(\bigtimes_{k=0}^n A_i = \{(a_o,a_i,..,a_n)| a_i \in A_i \}\)
              }\\
              \hline
              \mc{
              \r{Undirected Graph}\\ \footnotesize{doesn't have directions, and therefore can't have edges to itself}\\
              \r{Directed Graph} \\ \footnotesize{This does have directions, therefore an edge to itself is valid!}\\
              \pic{220617-4}\\
              Vertices V = \(\{v_1,v_2,...,v_n \}\) Edge e = \(\{v_3,v_4\}\) \\
              EdgeCount E = \(\{e | e Edge \}\)\\
              }\\
              \hline
              \mc{\r{Proofs:}\\
              \r{constructive Proof (proof by reforming)}\\
              \footnotesize{Consider \(ax^2 + bx + c = 0\), we can proof this to have 2 solutions by reforming.}\\
              \pic{220617-5}\\
              \footnotesize{If \(b^2 - 4ac > 0\) then we have 2 solutions!}\\
              \r{Proof by contradiction}\\
              \footnotesize{Take -2, it isn't a natural number. We can prove this by claiming the opposite.}\\
              \footnotesize{If -2 is a natural number, then it has all the attributes of a natural number.}\\
              \footnotesize{For example, it should be possible to take the square root of -2.}\\
              \(\sqrt{-2} = NaN\)\\
              \footnotesize{As you can see -2 does not have this attribute and is therefore}\\
              \footnotesize{NOT a natural number!}\\
              }\\
              \hline
            \end{tabular}
            \hspace{-0.1in}
            \begin{tabular}[t]{|@{\hskip1pt}p{10.475cm}|}
              \hline
              \mc{\r{Proof by Induction}\\
              \footnotesize{This is particularly useful if you want to check an attribute}\\
              \footnotesize{for a range of numbers such as n or n+1}\\
              \begin{tabular}{ll}
              \mc{\footnotesize{Base claim:}\\
              \(P(n) = \sum_{k=1}^n = \dfrac{n(n+1)}{2}\)\\
              \footnotesize{Anker: check for n=1}\\
              \(P(1) = \dfrac{1(1+1)}{2} = 1\)\\}&
              \mc{\footnotesize{Hypothesis: it also works for n+1}\\
              \pic{220617-6}}\\
              \end{tabular}
              }\\
              \hline
              \mc{\r{Alphabet and Word}\\
              \(\Sigma\) = Alphabet: Nonempty Quantity of characters\\
              \(\Sigma^n = \Sigma \times ... \times \Sigma\) = String\\
              \( w \in \Sigma^n\) An element in that string is a Word with length n. \\
              \r{\(\varepsilon \in \Sigma^0\) The empty word, don't forget the empty word!}\\
              Quantity of all words:\\
              \(\Sigma^* = \{\varepsilon\} \cup \Sigma \cup \Sigma^2 \cup \Sigma^3 \cup ... = \bigcup_{k=0}^\infty \Sigma^k\)\\
              }\\
              \hline
              \mc{\r{Language}
              \r{\(L \subset \Sigma^* \) = Language}\\
              \(L= \emptyset \subset \Sigma^* =\) Empty Language\\
              \(L = \Sigma^* \to \Sigma = \{0,1\}\) all binary strings.\\
              \r{A language is regular if a DFA can be formed out of it.}
              }\\
              \hline
              \mc{\r{Deterministic Finite Automaton (DFA/DEA)}\\
              \pic{220617-7}\\
              \footnotesize{A very simple machine that accepts a variety of inputs.}\\
              \footnotesize{Only requirement is that a D follows after T.}\\
              \footnotesize{This means all the following inputs are valid:}\\
              \footnotesize{\_ (empty word!), D, DD, TD, TTTTTTTD, DDDDDTD, DDDDD, ....}\\
              \begin{tabular}{ll}
              \mc{\pic{220617-8}} &
              \mc{Machine A: \{\r{Q},\blu{\(\Sigma\)},\gre{\(\delta\)},\(q_0\),\yel{F}\}\\
              State = \r{Q} -> \(\{q_1,q_2,...,q_n\}\)\\
              Alphabet = \blu{\(\Sigma\)}\\
              Transitioning-Function = \(\gre{\delta} : \r{Q} \times \blu{\Sigma} \to \r{Q}\)\\
              Starting State = \(q_0 = L(\varepsilon) = L\)\\
              Acceptable Endstates =\(\yel{F} \subset \r{Q}\)
              }\\
              \end{tabular}\\
              \pic{220617-9}\\
              \r{Language of DFA A:}\\
              \r{\(L(A) = \{ w \in \Sigma^* | A \text{ accepts } w \} = \{w \in \Sigma^* | \delta(q_0,w) \in F\}\)}\\
              \footnotesize{The language of a DFA is simply all accepted words!}\\\\
              \r{Error States in DFA}\\
              \pic{220617-10}\\
              \begin{tabular}{ll}
              \mc{\pic{220617-11}\\}&
              \mc{\footnotesize{from q, many paths lead to F}\\
                \footnotesize{This means 11, or 0 would be the "same"}\\
                \footnotesize{\(L(q) = \{0,10,11,12,...\} \)}\\
                \footnotesize{The same would obviously apply to P}
              }\\
              \end{tabular}
              }\\
              \hline
            \end{tabular}
          \end{tabular}
          \hfill
        \end{normalsize}
    \end{table}
    \pagebreak
    \begin{table}[ht!]
        \begin{normalsize}
          \begin{tabular}{cc}
            \begin{tabular}[t]{|@{\hskip1pt}p{10.475cm}|}
              \hline
              \mc{\r{Myhill-Nerode}\\
              \footnotesize{Adding a word to a word, to make it compatible with a language}\\
              \r{\(L(w) = \{w' | ww' \in L\} \text{ including: } L(\varepsilon)\)!}\\
              \begin{tabular}{ll}
                \mc{\pic{220617-12}\\} &
                \mc{\footnotesize{even and uneven amounts of 0s}\\
                \footnotesize{mod 2 zero's, 1's don't matter}\\
                }\\
              \end{tabular}\\
              \r{Detecting Nonregular Languages with Myhill}\\
              \footnotesize{The examples before always had a specific amount of words/characters}\\
              \footnotesize{that one had to add, in order to accept the word.}\\
              \footnotesize{However, there are languages that would need infinite states}\\
              \footnotesize{in order to find the entire language of a DFA}\\
              \footnotesize{A good example for this is the language \(1^n0^n\)}\\
              \begin{tabular}{ll}
              \hspace{-0.1in} \mc{\pic{220617-13}} &
              \mc{\footnotesize{for every 0 that we add, we need a 1} \\ 
              \footnotesize{this means that for n+k 0's we need k 1's}\\
              \footnotesize{as \(\lim_{k\to\infty}\) we need \(\infty \) states!}\\
              \footnotesize{not possible with a \r{Deterministic} automaton!}\\\\
              \footnotesize{also note: we have clear error states}\\
              \footnotesize{anything starting with 1 is an error.}
              }\\
              \end{tabular}
            }\\
              \hline
              \mc{}\\
              \hline
              \mc{}\\
              \hline
              \mc{}\\
              \hline
            \end{tabular}
            \hspace{-0.1in}
            \begin{tabular}[t]{|@{\hskip1pt}p{10.475cm}|}
              \hline
              \mc{}\\
              \hline
            \end{tabular}
          \end{tabular}
          \hfill
        \end{normalsize}
    \end{table}
    \pagebreak
    \begin{table}[ht!]
        \begin{normalsize}
          \begin{tabular}{cc}
            \begin{tabular}[t]{|@{\hskip1pt}p{10.475cm}|}
              \hline
              \mc{}\\
              \hline
            \end{tabular}
            \hspace{-0.1in}
            \begin{tabular}[t]{|@{\hskip1pt}p{10.475cm}|}
              \hline
              \mc{}\\
              \hline
            \end{tabular}
          \end{tabular}
          \hfill
        \end{normalsize}
    \end{table}
    \pagebreak
    \begin{table}[ht!]
        \begin{normalsize}
          \begin{tabular}{cc}
            \begin{tabular}[t]{|@{\hskip1pt}p{10.475cm}|}
              \hline
              \mc{}\\
              \hline
            \end{tabular}
            \hspace{-0.1in}
            \begin{tabular}[t]{|@{\hskip1pt}p{10.475cm}|}
              \hline
              \mc{}\\
              \hline
            \end{tabular}
          \end{tabular}
          \hfill
        \end{normalsize}
    \end{table}
\end{document}
