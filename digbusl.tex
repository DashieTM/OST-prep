\documentclass{article}
\usepackage[left=1mm, right=1mm, top=1mm, bottom=1mm]{geometry}
\usepackage[document]{ragged2e}
\usepackage{graphicx}
\usepackage{array, makecell}
\usepackage[T1]{fontenc}
\newcommand{\mc}{\makecell[{{p{1\linewidth}}}]}
\newcommand{\pic}{\includegraphics[scale=0.3]}
%\newcommand{\pc}[1]{\end{tabular}\\ #1 \\\begin{tabular}{|p{0.1\lindwidth}|p{0.755\linewidth}}
\graphicspath{{./Pictures/}}
\begin{document}
  \begin{flushleft}
      \begin{table}[h!]
        \begin{tabular}{|p{0.2\linewidth}|p{0.755\linewidth}|}
          \hline
        \textbf{Command | Function} & \textbf{Description} \\ 
          \hline\hline
          \mc{Digitization:} & \mc{transferal of already existing information/technologies into the digital world}\\
          \hline
          \mc{Digitalization:} & \mc{Creation of new technologies etc, that weren't possible before.}\\
          \hline
          \mc{It management} & \mc{Focuses on operational efficiency of a company}\\
          \hline
          \mc{Digital business} & \mc{Focuses on innovative solutions and customer satisfaction}\\
          \hline
          \mc{Market Value (Tauschwert)} & \mc{How much is the service/product worth in numbers}\\
          \hline
          \mc{Usage Value (Gebrauchswert)} & \mc{Worth of actually using it}\\
          \hline
          \mc{Job to be done} & \mc{The goals that a customer wants to achieve with the service / product}\\
          \hline
          \mc{Customer Value in Traditional Business} & \mc{The traditional business tries to create a product that appeals to as many people as possible.\\
          With the sale of said product, their interaction is over, since the customer now has the product.}\\
          \hline
          \mc{Customer Value in Digital Business} & \mc{In the digital world the customers opinion is more important as they can voice their opinion easily.
          Therefore the companies create a value proposition with their product (Wertversprechen), which in theory should be more than just a "promise".}\\
          \hline
          \mc{Digital Business definition} & \mc{Creation of new business structures which merge the physical and the digital world.}\\
          \hline
          \mc{Interactivity of customer Value (Interaktivität)} & \mc{The customer always has to have some sort of interaction with this product. Ex. a use case for it.}\\
          \hline
          \mc{Relativism of customer Value (Relativismus)} & \mc{Certain situations might cause the customer to want this product -> Hunger -> food.}\\
          \hline
          \mc{Preference dependency of customer Value (Präferenzabhängigkeit)} & \mc{Personal preferences also change the products the customer buys.}\\
          \hline
          \mc{Experience dependency of customer Value (Erfahrungsabhängikeit)} & \mc{Does the customer already know this product?}\\
          \hline\\
          \mc{}& \pic{220624-1}\pic{220624-2}\\
          \hline
          \mc{Digital Services} & \mc{Digital services increase the independency of place,time,ability for customers. Aka it gets easier and more available everywhere.
          It also increases the possibilities, new products etc.}\\
          \hline
          \mc{Resource Density (Resourcendichte)} & \mc{The ease of providing a service to increase customer value. Aka how easy to sell. Dimensions of Resource Density: Time, Place, abilities, efficiency.}\\
          \hline
          \mc{Time\\
        Place\\
        abilities\\
        efficiency} & \mc{The time specifies how fast you can get a product.\\
          The place specifies where you can buy said product.\\
        The abilities specify what you need to do to get a product. -> buy from store easy, rent a car harder\\
      The efficiency specifies how efficient the sale is for customers. Leads to easier buying for customers.}\\
          \hline
          \mc{Resource Density increase (Resoursenverdichtung)} & \mc{Make it easier for customers to buy, example. Online shopping instead of having to go to a shop.}\\
          \hline\\
          \mc{}& \pic{220624-3}\\
          \hline
          \mc{Immaterialität, Adaptivität, Vernetzbarkeit} & \mc{Digital business is immaterial, which makes it available everywhere,anytime.\\ Digital business is adaptive,
          products/services can have multiple purposes that fit the needs of pretty much everyone.\\ Digital Business is connected, which makes it possible to combine services into one.}\\
          \hline
      \end{tabular}
    \end{table}
      \begin{table}[h!]
        \begin{tabular}{|p{0,2\linewidth}|p{0.755\linewidth}|}
          \hline
          \mc{Splitting of Things and data} & \mc{Trennung von Dingen und daten means the physical product, and the accompanying manuals, costs, information, etc.
          The second is the data, which can be handled independently from the product and therefore used for different things, such as services.
        Example rental cars, it is possible for you to order any rental car online as this data is not stuck inside the car itself.}\\
          \hline
          \mc{Resource Intergration} & \mc{This means the integration of data from different things into one. Example, you want to rent a hotel, you can also rent a car and more at the same time!}\\
          \hline
          \mc{Unbundling} & \mc{Splitting of a product that was previously sold as one. Usually done to integrate with partners.}\\
          \hline
          \mc{Rebundling} & \mc{Packing different services into one.}\\
          \hline
          \mc{Reconfiguration} & \mc{The whole process of unbundling, then rebundling a service or a product}\\
          \hline
          \mc{Resource Mobility} & \mc{make something available digitally.}\\
          \hline
          \mc{Resource Mobilization} & \mc{Usage of previously not used resources. -> Nvidia open source kernel module now uses the power of pull requests.}\\
          \hline\\
          \mc{}& \pic{220624-4} \pic{220624-6}\\
          \hline
          \mc{Digital Disruption} & \mc{The overtake of a digital product from a regular one -> Uber deletes taxis.}\\
          \hline
          \mc{Disintermediation} & \mc{elimination of third party sellers that would usually be necessary. -> Lenovo direct seller instead of digitec.}\\
          \hline
          \mc{Reintermediation} & \mc{Comparison and Review Services for Products. Trivago, Toppreise}\\
          \hline
          \mc{the 4 types of Co-Creation\\
        cooperation, tinkering, Co-Design, Additions(Einreicherungen)} & \mc{Cooperation, the development/use of a product together. (Use -> Car sharing) \\
        Tinkering: Modifications on existing products by customers which can be integrated into a new product.\\
      Co-Design: Designing of things together, see dota skins\\
    Additions: Call for new ideas, additions etc from a company to customers.}\\
          \hline
          \mc{Long Tail Product} & \mc{A product that is niche and therefore doesn't sell that well, but if you can hit the right people, it will be profitable in bigger amounts.}\\
          \hline
          \mc{Crowd Economy: \\
        Crowdfunding, Sharing Economy, Crowd-Innovation, Crowdsourcing} & \mc{Crowd Economy: Creation of new forms of an economy by interaction of humans over the internet \\
        Sharing Economy: Usage of same resource together, Crowdsourcing: working together on a project -> FOSS }\\
          \hline\\
          \mc{}& \pic{220624-7} \pic{220624-8}\\
          \hline
          \mc{Product} & \mc{A product is a material object.}\\
          \hline
          \mc{Service (Default view)} & \mc{A service is an immaterial object}\\
          \hline
          \mc{Productoriented vs Customer oriented} & \mc{The only way to increase the value of a product with the first view is to either,
          increase the quality of the product, or decrease the cost of the production of said product.
        Obviously this view is trash since you could instead just be more customer centric and ask them directly what they would like to have.}\\
          \hline\\
            \mc{}& \mc{\pic{220624-9}\pic{220624-10}}\\
          \hline
          \mc{Servitization} & \mc{"As a service" Products that increasingly have a service dominance in them. You don't buy a phone, you rent it, etc...}\\
          \hline
      \end{tabular}
    \end{table}
      \begin{table}[h!]
        \begin{tabular}{|p{0,2\linewidth}|p{0.755\linewidth}|}
          \hline
          \mc{Service (Service View)}& \mc{The service is no longer just an immaterial object but describes each use of resources withing operations and processes.}\\
          \hline
          \mc{Resource (Service View)}& \mc{Resources aren't just used, instead they are carriers to services for a specific goal.}\\
          \hline\\
          \mc{}& \pic{220624-11} \pic{220624-12}\\
          \hline
          \mc{Value (Service View)}& \mc{The value of a service is only calculated based on the customer.}\\
          \hline
          \mc{Reason for Service View}& \mc{It is more individual for customers, it is more automated, it offers better integration with the cancer named as "pay-per-use".
          It has replaced the regular product centric view for digital business.}\\
          \hline\\
            \mc{}& \pic{220624-13}\pic{220624-14}\\
          \hline
          \mc{Digital Twin}& \mc{A virtual entity which is automatically connect with a real one. These are often used as virtual environments to test things. Or to automate sensor based
          reactions. A good example would be the creation of a new product with CAD then testing this with a digital twin instead of a real product!}\\
          \hline
          \mc{Problems with selling data}& \mc{Quality, Redundancy, Target -> is the data useful to US?}\\
          \hline
          \mc{The benefit and problem with data}& \mc{With data companies can target specific customers in a way that is beneficial for both, customer gets ads for products that the customer actually wants! wow! And therefore the company is actually going to sell something from that add. The problem? Privacy. In order to gain this amount of confidence about the usefulness of an add for a customer, you would need to know the entire life of that customer, which many are obviously not comfortable with and is also illegal in pretty much any country.
          In short, Data is knowledge, and knowledge is power, therefore companies with valuable data have an edge over others.}\\
          \hline\\
            \mc{}& \pic{220624-15}\pic{220624-16}\\\mc{}& \pic{220624-17} \pic{220624-18}\\
          \mc{Channels:\\
          Single,Multi,Cross,Omni}& \mc{Single Channel: Only 1 form of business, ex. only in person store.\\ Multi Channel: Services/Products available on multiple channels like in-person online, etc. However without interaction of these channels.\\ Cross Channel: Mutli channel with interaction -> order online, pickup from store\\ Omni Channel: Every Single Channel imaginable with interaction between them. Including social media etc. }\\
          \hline
          \mc{Demand-based customer strength}& \mc{By using the internet more and more, the companies are pushed to use it as well for their services and products.}\\
          \hline
          \mc{Information-based customer strength}& \mc{The internet makes it easy for customers to gain access to proper information, pushing companies to create better services.}\\
          \hline
      \end{tabular}
    \end{table}
      \begin{table}[h!]
        \begin{tabular}{|p{0,2\linewidth}|p{0.755\linewidth}|}
          \hline
          \mc{Network-based customer strength}& \mc{Behavior on social networks is extremely important, often companies get called out on social media and have to take action because of it.}\\
          \hline
          \mc{Crowd-based strength}& \mc{When customers actively work on a product they will naturally also have more say in the matter.}\\
          \hline\\
          \mc{}& \pic{220624-19}\pic{220624-20}\\
          \hline
          \mc{Networks}& \mc{Networks simply mean the interactivity of modern companies by combining services, sharing data and generally close interaction.}\\
          \hline
          \mc{Platform}& \mc{A Platform is a network with a clear governance-structure that defines who can do what. It also offers different sets of standards, offering more companies as well as customers this time, the ability to interact with each other.}\\
          \hline
          \mc{Platform attributes:\\
        Connection, drawing force (Anziehungskraft), Flow (Fluss)}& \mc{The connection means it must be easy for participants to join and interact with the platform.\\
        The drawing force defines that the platform must be attractive to use. \\ The Flow means that the platform must support Co-Creation. }\\
          \hline
          \mc{Parties involved}& \mc{Other than the host of the platform we have the participants: service offering and service seeking participants. \\
          As well as third party supporting participants that improve the platform, ex. Google or facebook integration.}\\
          \hline
          \mc{Different Forms of platforms:\\
          Serviceplatforms\\
          Social Platforms\\
          Mobilization Platforms}& \mc{A platform that offers an exchange of Resources. Either free or paid -> Stack overflow, fiver\\ 
          A platform that allows companies and users to exchange what is going on in their life.\\
        A platform that is created to follow a common goal. Usually crowd funding platforms like kickstarter, but also massdrop.}\\
          \hline
          \mc{More explicit types: \\
          ExchangePlatforms\\
          Aggregationplatforms\\
          Integreationplatforms\\
          Alliances}& \mc{Brings service seeking and offering together. -> Stack overflow, Ricardo \\ Offersa wide range of services and products from different companies -> Amazon, Digitec\\
          Brings many different services into one, to the customer only the company that is integrating these services is visible -> Could Databases from google etc.\\
          A collection of participants that have created a deep connection to offer services as one. -> Star Alliance Lufthansa}\\
          \hline\\
            \mc{}& \pic{220624-21}\pic{220624-22}\\
          \hline
          \mc{Platform vs Portals}& \mc{Portals don't have network effects, Portals are used for a customer to interact with a company and only the company. It is therefore a Many-to-One relationship,
          where the platform is a Many-to-Many relationship.}\\
          \hline
          \mc{}& \mc{}\\
          \hline
          \mc{}& \mc{}\\
          \hline
          \mc{}& \mc{}\\
          \hline
      \end{tabular}
    \end{table}
      \begin{table}[h!]
        \begin{tabular}{|p{0,2\linewidth}|p{0.755\linewidth}|}
          \hline
          \mc{}& \mc{}\\
          \hline
          \mc{}& \mc{}\\
          \hline
          \mc{}& \mc{}\\
          \hline
          \mc{}& \mc{}\\
          \hline
          \mc{}& \mc{}\\
          \hline
      \end{tabular}
    \end{table}
  \end{flushleft}
\end{document}

