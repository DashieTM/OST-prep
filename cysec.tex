\documentclass{article}
\usepackage[left=-1.5mm, right=-1.5mm, top=-1mm, bottom=1mm]{geometry}
\usepackage[document]{ragged2e}
\usepackage{graphicx}
\usepackage{amsmath, mathtools, nccmath, amssymb}
\usepackage[dvipsnames]{xcolor}
\usepackage{array, makecell}
\usepackage{moresize}
\usepackage[T1]{fontenc}
\usepackage{noto-sans}
\renewcommand\cellalign{cl}
\renewcommand\theadalign{lc}
\newcommand{\ns}{\\ \vspace{-0.03in}}
\newcommand{\mc}[1]{\makecell[cl]{#1}}
\newcommand{\mcc}{\makecell[{{c}}]}
\newcommand{\mcr}{\makecell[{{r}}]}
\newcommand{\pic}{\includegraphics[scale=0.3]}
\newcommand{\form}[1]{\vspace{-0.03in}\\ \small{\textcolor{Red}{\(#1\)}} \\ \vspace{-0.03in}}
\newcommand{\forms}[1]{\vspace{-0.01in}\\ \scriptsize{\textcolor{Red}{\(#1\)}\\ \vspace{-0.01in}}}
\newcommand{\noskipform}[1]{\small{\textcolor{Red}{\(#1\)}}}
\newcommand{\fpic}{\vspace{-0.05in} \\ \includegraphics[scale=0.3]}
\newcommand{\dx}[1]{\scriptsize{\textcolor{Red}{\(\dfrac{d}{dx}#1\)}}}
\newcommand{\dxs}[1]{\ssmall{\(\dfrac{d}{dx}#1\)}}
\newcommand{\dxo}{\dfrac{d}{dx}}
\renewcommand{\b}{\textbf}
\renewcommand{\r}[1]{\textcolor{Red}{#1}}
\newcommand{\gre}[1]{\textcolor{Green}{#1}}
\newcommand{\blu}[1]{\textcolor{Blue}{#1}}
\newcommand{\ora}[1]{\textcolor{Orange}{#1}}
\newcommand{\lims}[1]{\lim\limits_{#1}}
\newcommand{\atb}[1]{\int_a^b #1 \,dx}
%\renewcommand{\arraystretch}{1.5}
\graphicspath{{./cysec-Pictures/}}
\begin{document}
    \begin{table}[ht!]
          %\vspace{0.075in}
          \begin{tabular}{ccc}
          \begin{tabular}[t]{|@{\hskip1pt}p{10.45cm}|}
              \hline  
              \mc{\r{Asset}\\
              \footnotesize{Anything within the organization that is worth protecting:}\\
              \footnotesize{Information, Systems, Devices, Facilities, Personnel, Intellectual Property}\\
              }\\
              \hline
              \mc{\r{Confidentiality:}\\
              \blu{Military / Government confidentiality:}\\
              -- \blu{Top Secret:} \footnotesize{drastic effects / grave damage to national security}\\
              -- \blu{Secret:} \footnotesize{significant effects / critical damage to national security}\\
              -- \blu{Confidential:} \footnotesize{noticeable effects / serious damage to national security}\\
              -- \blu{Sensitive but unclassified:} \footnotesize{internal use}\\
              -- \blu{Unclassified:} \footnotesize{public data}\\
              \gre{Commercial / private confidentiality}\\
              -- \blu{Coonfidential / private:} \footnotesize{drastic effects on the competitiveness}\\
              -- \blu{Sensitive}\\ 
              -- \blu{Public}\\
              \footnotesize{In general classified is used as a term to describe anything but public data.}\\
              }\\
              \hline
              \mc{\r{Deleting Data}\\
                \footnotesize{There are multiple ways to delete data with massive difference in effectiveness:}\\
                -- \blu{Erasing:} \footnotesize{removes only the link to the storage point, actual data remains}\\
                \, \,\,\footnotesize{\r{Simple tools can recover this data!}}\\
                -- \blu{Clearing:} \footnotesize{Overwrites the data to delete it, usually with a single character}\\
                \, \,\,\footnotesize{Can not be recovered using simple tools}\\
                -- \blu{Purging:} \footnotesize{Clearing, but overwrites multiple times to make recovery harder}\\
                \, \,\,\footnotesize{This method might also include others such as degaussing. It is Irrecoverable}\\
                -- \blu{Degaussing:} \footnotesize{Strong magnetic field that can wipe data from an HDD}\\
                \, \,\,\footnotesize{(SSD, CD, etc not affected!)}\\
                -- \blu{Destruction:} \footnotesize{Destroy the physical hardware of said data. Irrecoverable.}\\
              }\\
              \hline
              \mc{\r{Tracing and Hiding sensitive Data}\\
                -- \blu{Steganography:} \footnotesize{Embedding a Message within a file}\\
                -- \blu{Watermarking:} \footnotesize{unique identifier, that usually can't copied, or mutated.}\\
              \, \,\,\footnotesize{Often used in Documents, movies, etc to stop counterfeits.}\\
              }\\
              \hline
              \mc{\r{STRIDE Model}\\
                -- \r{S}\blu{poofing:} \footnotesize{Using a false identity to gain access to a system.}\\
                -- \r{T}\blu{ampering:} \footnotesize{unauthorized changes/manipulation of data}\\
                -- \r{R}\blu{epudation:} \footnotesize{The ability to deny having performed an attack, }\\
              \, \,\,\footnotesize{ to others being blamed.}\\
              -- \r{I}\blu{nformation Disclosure:} \footnotesize{unauthorized revelation of classified }\\
              \, \,\,\footnotesize{ private information.}\\
              -- \r{D}\blu{enial of Service:} \footnotesize{Prevent or restrict access to a service by flooding it.}\\
              -- \r{E}\blu{levation of Privilege:} \footnotesize{Gaining unauthorized privileges on a system}\\
              \, \,\,\footnotesize{For example: becoming root as regular user}\\
              }\\
              \hline
              \mc{\r{Copyright}\\
              Protects these works: Literay, musical, dramatic, choreographic,\\
              graphical,sculptural,audiovisual, recordings, architectural works\\
              \footnotesize{Computer software is in the literary works category}\\
              \footnotesize{Note only the source code is protected, not the idea itself.}\\
              \footnotesize{Implicit Copyright is automatically granted if you are the creator of said work.}\\
              \footnotesize{The explicit copyright can be obtained from the government and }\\
              \footnotesize{lets you use the symbol: \copyright}\\
              \footnotesize{This copyright lasts for 70 years after the death of the last copyright holder.}\\
              }\\
              \hline
              \mc{\r{Trademark}\\
              Protects: Slogans, Logos, words used to identify something\\
              \footnotesize{Implicit Trademark is automatically granted if you use the \texttrademark{} symbol}\\
              \footnotesize{Explicit Trademark has to be obtained from the government }\\
              \footnotesize{and lets you use the \textregistered{} symbol.}\\
              }\\
              \hline
              \mc{\r{Patents}\\
              Patents protect intellectual Property for 20 years.\\
              For a patent a product must be \r{useful,new,not be obvious}\\
              \footnotesize{After the 20 years the patents enters into the public domain.}\\
              \footnotesize{Patents are the worst thing ever.}\\
              }\\
              \hline
              \mc{\r{Trade Secrets}\\
              \footnotesize{Copyright, Patents and Trademarks require you to disclose what is protected}\\
              \footnotesize{Because of this, many companies simply hide this information from the public}\\
              \footnotesize{This means, while you are not protected by the law, as long as this information }\\
              \footnotesize{stays within your company, it will be protected forever.}\\
              \footnotesize{Likely the "best way" to protect computer software, aka proprietary trash}\\
              }\\
              \hline
            \end{tabular}
            \hspace{-0.1in}
            \begin{tabular}[t]{|@{\hskip1pt}p{10.45cm}|}
              \hline
              \mc{\r{CIA triad}\\
              The primary goal of Security infrastructure:\\
              -- \r{C}\blu{onfidientiality:} \footnotesize{prevention of unauthorized access to data}\\
              \, \,\,\footnotesize{no matter if the data is \r{in transit, in storage or in process}}\\
              \, \,\,\footnotesize{Usually done with \r{encryption and access control}}\\
              -- \r{I}\blu{ntegrity:} \footnotesize{prevention of unauthorized alteration of data}\\
              \, \,\,\footnotesize{\gre{Data integrity:} data is complete, consistent and accurate}\\
              \, \,\,\footnotesize{\gre{System integrity:} System only does what it was intended to do}\\
              \, \,\,\footnotesize{Usually  done with \r{hash verification, intrusion detection}}\\
              -- \r{A}\blu{vailability:} \footnotesize{Systems should be accessible at all times}\\
              \, \,\,\footnotesize{Usuall done with \r{Redundancy, backups}}\\
              }\\
              \hline
              \mc{\r{Nonrepudation \& Accountability}\\
              \footnotesize{Nonrepudation ensures that every action can be traced to the actual source}\\
              \footnotesize{This prevents attackers from covering their actions.}\\
              \footnotesize{Accountability ensures that every being is responsible for their actions.}\\
              \footnotesize{E.G. Nonrepudation ensures Accountability.}\\
              \footnotesize{Usually done with certificates, session identifiers, logs, etc}\\
              }\\
              \hline
              \mc{\r{Data Classification}\\
                -- \blu{Data in Use, Transit, Rest}\\
              \, \,\,\footnotesize{used by application, tranit via x, rest = storage, HDD, SSD}\\
              -- \blu{Personally Identifiable Information PII}\\
              \, \,\,\footnotesize{information to identify a person, name, social security number etc}\\
              -- \blu{Protected Health Information PHI}\\
              \, \,\,\footnotesize{Health Information recorded in any form.}\\
              -- \blu{Proprietary(Trash) Data}\\
              \, \,\,\footnotesize{microtroll windoof, appul}\\
              }\\
              \hline
              \mc{\r{Threats}\\
                -- \blu{Threat:} \footnotesize{A potential danger to an asset}\\
                -- \blu{Threat intelligence:} \footnotesize{Knowledge about emerging or existing threats}\\ 
                -- \blu{Thrat event:} \footnotesize{Accidental or malicious exploitation of a vulnerability}\\
              }\\
              \hline
              \mc{\r{Vulnerability}\\
              -- \blu{Common Vulnerabilities and Exposures (CVE)}\\
              \, \,\,\footnotesize{Industry wide standard identification number for Vulnerabilities}\\
              -- \blu{Common Vulnerability Scoring System (CVSS) }\\
              \, \,\,\footnotesize{Uses the CIA triad to score vulnerabilities on severity}\\
              }\\
              \hline
              \mc{\r{Risk}\\
              \footnotesize{Assessment of the possibility that a threat will exploit a vulnerability}\\
              \r{Realized Risk}\\
              \footnotesize{A threat actor has now taken advantage of a vulnerability}\\
              \footnotesize{The whole point of security is to stop exactly this.}\\
              }\\
              \hline
              \mc{\r{Risk Management}\\
                -- \blu{Identifying vulnerabilities}\\
                -- \blu{evaluating importance of data and countermeasure cost}\\
                -- \blu{Implementing cost effective countermeasures}\\
              \footnotesize{Risk management is the balance of threat/risk and usability}\\
              \footnotesize{In other words, only implement measures that are necessary}\\
              \footnotesize{Some data is not worth protecting, and some systems can be replaced easily.}\\
              \footnotesize{Others are sensitive, or fundamental, these must be preserved at all costs.}\\
              }\\
              \hline
              \mc{\r{Risk Analysis}\\
              -- \blu{Evaluation, assessment, assigment of value to assets}\\
              -- \blu{Examining environment for risks}\\
              -- \blu{Evaluating the likelihood of a threat event occurring}\\
              -- \blu{Assessing the cost of countermeasures}\\
              -- \blu{present cost/benefit report to upper management}\\
              }\\
              \hline
              \mc{\r{Asset Valuation}\\
              \footnotesize{The representation of an asset in currency}\\
              \footnotesize{This can include things such as repair costs, maintenance costs,etc}\\
              }\\
              \hline
              \mc{\r{Exposure}\\
              \footnotesize{Possibility of an asset loss due to a threat}\\
              \footnotesize{exposure factor (EF) checks how serious that loss would be}\\
              }\\
              \hline
              \mc{\r{Attack}\\
              \footnotesize{The \r{intentional} try to exploit a vulnerability}\\
              }\\
              \hline
              \mc{\r{Breach}\\
              \footnotesize{The circumvention of security measures by a threat actor.}\\
              \footnotesize{A \r{breach combined with an attack}, can result in \r{penetration}}\\
              }\\
              \hline
            \end{tabular}
          \end{tabular}
          \hfill
    \end{table}
    \pagebreak
    \begin{table}[ht!]
          \begin{tabular}{ccc}
            \begin{tabular}[t]{|@{\hskip1pt}p{10.45cm}|}
              \hline
              \mc{\\
              \pic{220620-1}}\\
              \hline
              \mc{\r{Quantitative and Qualitative Risk Analysis}\\
              --\gre{Quantitative Risk Analysis:}\\
              \footnotesize{assignment of real dollar figures to loss of assets}\\
              --\gre{Qualitative Risk Analysis}\\
              \footnotesize{The subjective / intangible worth value to the loss of assets}\\
              \footnotesize{\r{Both methodologies are necessary for complete risk analisys!}}\\
              }\\
              \hline
              \mc{\r{Quantitative Risk Analysis}\\
              \pic{220620-2}\\
              -- \blu{Exposure Factor (EF):} \footnotesize{percentage of loss by realized risk}\\
              -- \blu{Single Loss Expectancy (SLE):} \footnotesize{Cost of single realized  risk}\\
              \, \,\,\ora{\footnotesize{SLE = EF * Asset Value (AV)}}\\
              -- \blu{Annualized Rate of Occurrence (ARO):} \\
              \, \,\,\footnotesize{expected frequency of a risk within a year}\\
              -- \blu{Annualized Loss Expectancy (ALE):} \\
              \, \,\,\footnotesize{Expected yearly cost of Losses due to realized risks}\\
              \, \,\,\ora{\footnotesize{ALE = SLE * ARO}}\\
              \footnotesize{If you implemented a safeguard, you have to recalculate the ARO.}\\
              \footnotesize{The entire idea of of security is to reduce the ARO!!}\\
              \footnotesize{The EF usually remains the same}\\
              -- \r{Safeguard Costs}\\
              \footnotesize{First, compile a list of safeguards against each threat.}\\
              \footnotesize{Assign each safeguard a deployment value -> \r{Annual Cost of Safeguard (ACS)}}\\
              -- \r{Safeguard Cost/Benefit}\\
              \footnotesize{ALE without safeguard - ALE with safeguard - ACS = Value of safeguard}\\
              \footnotesize{if the value of safeguard is below 0, then it is financially irresponsible}\\
              \footnotesize{Note that this only takes in the financial damage since this is Quantitative!}\\
              }\\
              \hline
              \mc{\r{Dealing with Risk}\\
              -- \blu{Risk Mitigation} \footnotesize{Implementation of safeguards in order to}\\
              \, \,\,\footnotesize{eliminate vulnerabilities or block threats}\\
              -- \blu{Risk Assignment / Risk Transferring} \\
              \, \,\,\footnotesize{purchasing insurance or outsourcing. Transfer the cost of risk to other entity}\\
              -- \blu{Risk Acceptance} \footnotesize{Doing nothing as cost/benefit would be low}\\
              -- \blu{Risk Deterrence} \footnotesize{auditing, cameras, security guards, warnings, etc}\\
              -- \blu{Risk Avoidance} \footnotesize{Not using the system associated with the risk}\\
              -- \blu{Risk Rejection} \footnotesize{Simply ignore risk \r{without cost/benefit analysis!}}\\
              -- \blu{Residual Risk} \footnotesize{A countermeasure might not fully eliminate a risk}\\
              \, \,\,\footnotesize{This is the remaining risk that we have decided to accept.}\\
              }\\
              \hline
            \end{tabular}
            \hspace{-0.1in}
            \begin{tabular}[t]{|@{\hskip1pt}p{10.45cm}|}
              \hline
              \mc{\r{Privacy} \\
                -- \blu{GDPR} \\
                \, \,\, - \footnotesize{Companies have inform authorities in case of serious data breaches}\\
                \, \,\, - \footnotesize{Individuals have the right to demand their data from companies}\\
                \, \,\, - \footnotesize{Individuals have the right to be forgotten (deletion of data)}\\
                \, \,\, - \footnotesize{EU tries to enforce this globally}\\
                \, \,\, - \footnotesize{enforces pseudonomization}\\
                -- \blu{Patriot act}\\
                \, \,\, - \footnotesize{blanked authorization of surveillance of an individual with 1 warrant}\\
                \, \,\, - \footnotesize{ISP have to provide data}\\
                \, \,\, - \footnotesize{easier wiretapping}\\
                -- \blu{pseudonomization} \footnotesize{replacement of data with aliases}\\
                \, \,\, \footnotesize{This makes it harder to identify a person from said data}\\
                -- \blu{anonymization} \footnotesize{Complete obfuscation of an identity}\\
              }\\
              \hline
              \mc{\r{Access Control}\\
              -- \blu{Subject} \footnotesize{The Requesting party}\\
              \, \,\,\footnotesize{Note, this can be a person, a program, a process, etc.}\\
              -- \blu{Object} \footnotesize{The requested "object"}\\
              \, \,\,\footnotesize{Databases, programs, files, etc}\\
              -- \blu{Access Control} \footnotesize{The management of the relationship}\\
              \, \,\,\footnotesize{Between Subject and Object}\\
              -- \blu{Preventive Access Control}\\
              \, \,\,\footnotesize{Stops unwanted access to Objects}\\
              -- \blu{Detective Access Control}\\
              \, \,\,\footnotesize{Detects unwanted access to Objects after it has occurred}\\
              -- \blu{Corrective Access Control}\\
              \, \,\,\footnotesize{Restores the last "correct" state after unwanted access}\\
              -- \blu{Deterrent Access Control}\\
              \, \,\,\footnotesize{discourages unwanted access to Objects }\\
              -- \blu{Directive Access Control}\\
              \, \,\,\footnotesize{Directive issued by company. -> Don't click on links.}\\
              -- \blu{Compensating Access Control}\\
              \, \,\,\footnotesize{Controls used in addition/ as a replacement. Can be any control measure}\\
              -- \blu{Recovery Access Control}\\
              \, \,\,\footnotesize{More advanced version of corrective control}\\
              -- \blu{Physical Control}\\
              \, \,\,\footnotesize{Control of items that one can physically touch}\\
              -- \blu{Technical / logical Control}\\
              \, \,\,\footnotesize{Hardware or Software mechanism for access control}\\
              -- \blu{Administrative Control}\\
              \, \,\,\footnotesize{Policies and Procedures created by a Company}\\
              }\\
              \hline
              \mc{\r{Steps of Access Control}\\
              \blu{1. Identification}\\
              \, \,\,\footnotesize{Providing an identity to enforce Nonrepudation.}\\
              \, \,\,\footnotesize{ is usually public information.}\\
              \, \,\,\footnotesize{username, tokens, fingerprints, facial recognition}\\
              \blu{2. Authentication}\\
              \, \,\,\footnotesize{Often used in combination with Identification.}\\
              \, \,\,\footnotesize{Verifies the identity. Usually a password, or token.}\\
              \blu{3. Authorization}\\
              \, \,\,\footnotesize{Controls what a user is allowed to do and what they aren't}\\
              \, \,\,\footnotesize{There are different ways of controlling this.(Check next page)}\\
              \blu{4. Auditing}\\
              \, \,\,\footnotesize{Record all actions by all Subjects in order to hold them accountable}\\
              \blu{5. Accounting (Accountability)}\\
              \, \,\,\footnotesize{This would mean taking actions against a person,}\\
              \, \,\,\footnotesize{that has acted in bad faith/maliciously }\\
              \, \,\,\footnotesize{Keep in mind that you need a strong system of identification,}\\
              \, \,\,\footnotesize{ and auditing}\\
              \, \,\,\footnotesize{in order to actually win in a court of law against the bad actor!}\\
              }\\
              \hline
              \mc{\r{Authentication Factors}\\
              -- \blu{Type 1: Something you know}\\
              -- \blu{Type 2: Something you have, ex. a Device, smartcard, etc}\\
              -- \blu{Type 3: Something you are/do, ex. Fingerprint, face, etc.}\\
              \ora{\footnotesize{Type 1 is the weakest form of authentication and type 3 the strongest!}}\\
              }\\
              \hline
            \end{tabular}
          \end{tabular}
          \hfill
    \end{table}
    \pagebreak
    \begin{table}[ht!]
          \begin{tabular}{ccc}
            \begin{tabular}[t]{|@{\hskip1pt}p{10.45cm}|}
              \hline
              \mc{-- \gre{Location}\\
              \footnotesize{This is a form that is used in combination with others.}\\
              \footnotesize{Ex: A certain IP might be a requirement to log into one of your systems}\\
              \footnotesize{Or your bank might block a transaction if it isn't from your country of residence.}\\
              }\\
              \hline
              \mc{\r{Multifactor Authentication}\\
              \footnotesize{Multifactor Authentication simply uses more than one type to authenticate}\\
              \footnotesize{This might be a password(Type1) and a token(Type2)}\\
              \ora{\footnotesize{The strongest form of authentication if therefore all 3 types together!}}\\
              }\\
              \hline
              \mc{\r{Passwords}\\
                \footnotesize{Passwords are never stored in plain text,} \\
              \footnotesize{they might be stolen easily with 1 breach}\\
              \footnotesize{Instead, they are saved with a hash-algorithm, }\\
              \footnotesize{makes them nearly useless to threat actors}\\
              \footnotesize{Unless they also have the access to said algorithm.}\\
              }\\
              \mc{\r{Types of passwords:}\\
              -- \blu{Plain Password}\\
              \, \,\,\footnotesize{use numbers, characters and special symbols with a length of at least 10.}\\
              \, \,\,\footnotesize{10 characters -> 928 years of cracking!!}\\
              \, \,\,\footnotesize{Don't use personal information such as names etc.}\\
              -- \blu{PassPhrases}\\
              \, \,\,\footnotesize{Passphrases are chained words that are easy to remember}\\
              \, \,\,\footnotesize{It is important to have a longer passphrase than a regular password!}\\
              \, \,\,\footnotesize{Still use special symbols. Also try to include random upper-lower case spelling}\\
              \, \,\,\footnotesize{Or leadspeak s1nc3 that 1s qu1te 3ff3ct1v3!!}\\
              -- \blu{Cognitive Passwords}\\
              \, \,\,\footnotesize{A series of personal questions. -> what is the name of your first pet?}\\
              \, \,\,\footnotesize{Best way to do this is letting users create both the question and the answer!}\\
              -- \blu{SmartCards}\\
              \, \,\,\footnotesize{Card used for identification / authentication. Often integrates key encryption}\\
              \, \,\,\footnotesize{Usually temper resistant}\\
              \, \,\,\footnotesize{One downside, loss of card might give a threat a window of exploitation!}\\
              -- \blu{Tokens}\\
              \, \,\,\footnotesize{Generated by a \r{special Device}.}\\
              \, \,\,\footnotesize{Regular password generators: any device can take that place. ex. smartphone}\\
              -- \blu{Synchronous Dynamic Token}\\
              \, \,\,\blu{Time-based One-Time Password (TOTP)}\\
              \, \,\,\footnotesize{Device generates Token every x seconds.}\\
              -- \blu{Asynchronous Dynamic Token}\\
              \, \,\,\blu{HMAC-based One-Time Password (HOTP)}\\
              \, \,\,\footnotesize{Device generates one time token based on algorithm. Stays until used!}\\
              }\\
              \hline
              \mc{\r{Access Control Models (Authorization)}\\
              -- \blu{Discretionary Access Control (DAC)}\\
              \, \,\,\footnotesize{Every object has an owner, and that owner can to grant or deny permissions}\\
              \, \,\,\footnotesize{NTFS from windoof uses this}\\
              -- \blu{Role Based Access Control}\\
              \, \,\,\footnotesize{Permissions are assigned to roles not users. Users are assigned to roles.}\\
              \, \,\,\footnotesize{Users who are in a role with said privileges can use them.}\\
              -- \blu{Rule Based Access Control}\\
              \, \,\,\footnotesize{Global rules that are applied to all subjects}\\
              \, \,\,\footnotesize{Good example is a firewall, which applies said rules equally to all subjects}\\
              -- \blu{Attribute Based Access Control}\\
              \, \,\,\footnotesize{Similar to Rule based Control but with additional attributes}\\
              \, \,\,\footnotesize{This could give one subject more rights than another}\\
              -- \blu{Mandatory Access Control}\\
              \, \,\,\footnotesize{Use of labels applied to both subject and Object}\\
              \, \,\,\footnotesize{If user has the same label as a file, then user has access to it.}\\
              }\\
              \hline
              \mc{\r{Authorization Mechanism}\\
              -- \blu{Implicit Deny}\\
              \, \,\,\footnotesize{Deny everything that hasn't been specifically allowed}\\
              \, \,\,\footnotesize{\r{Most used!}}\\
              -- \blu{Constrained Interference}\\
              \, \,\,\footnotesize{Applications might hide functionality based on the privileges of a user}\\
              }\\
              \hline
            \end{tabular}
            \hspace{-0.1in}
            \begin{tabular}[t]{|@{\hskip1pt}p{10.45cm}|}
              \hline
              \mc{
              -- \blu{Access Control Matrix}\\
              \, \,\,\footnotesize{This writes Objects,Subjects and privileges into a table}\\
              \, \,\,\footnotesize{If Subject tries to access an object the table for said object is checked}\\
              -- \blu{Capability Table}\\
              \, \,\,\footnotesize{This is the same as the Access Control Matrix but with a subject focus}\\
              \, \,\,\footnotesize{In this table the subject and all accessible Objects are written down}\\
              -- \blu{Content-Dependent Control}\\
              \, \,\,\footnotesize{Constrict Access to the data within an Object}\\
              \, \,\,\footnotesize{In a database a user might be able to check table 1 but not table 2.}\\
              \, \,\,\footnotesize{While the object is the entire database!}\\
              -- \blu{Context-Dependent Control}\\
              \, \,\,\footnotesize{Give a subject access depending on what the subject does}\\
              \, \,\,\footnotesize{Ex. The checkout button in an online shop only works, if you have something}\\
              \, \,\,\footnotesize{in the shopping cart.}\\
              -- \blu{Need to Know}\\
              \, \,\,\footnotesize{Subjects should only have access to Objects they need to do their job.}\\
              -- \blu{Least Privilege}\\
              \, \,\,\footnotesize{Subjects should only have the privileges they need to do their job.}\\
              -- \blu{Separation of Duties and Responsibilities}\\
              \, \,\,\footnotesize{\r{No single person should have total control over the entire System!}}\\
              }\\
              \hline
              \mc{\r{Common Access Control Attacks}\\
              -- \blu{Access Aggregation Attacks (Passive Attacks)}\\
              \, \,\,\footnotesize{This is the collection of nonsensitive data, that combined could give}\\
              \, \,\,\footnotesize{a threat actor the opportunity to launch a proper attack.}\\
              \, \,\,\footnotesize{Ex. IP address, open ports, Operating System -> specific exploit}\\
              -- \blu{Password Attacks (Brute Force)}\\
              \, \,\,\footnotesize{Spam random sequences until you get the right one}\\
              -- \blu{Dictionary Attacks (Brute Force)}\\
              \, \,\,\footnotesize{Try passwords from a list of passwords, example leaked password list.}\\
              \, \,\,\footnotesize{Can also be done with list of common passwords, or slightly changed}\\
              \, \,\,\footnotesize{previous passwords (One-Upped-Passwords -> 1 character changed)}\\
              -- \blu{Birthday Attack (Brute Force)}\\
              \, \,\,\footnotesize{Try to get the same hash as the password with a different sequence}\\
              \, \,\,\footnotesize{Can be mitigated by using better hashing algorithms. SHA-3 instead of md5}\\
              \, \,\,\footnotesize{\r{Note the attacker needs access to the hash in order for this to work!}}\\
              -- \blu{Rainbow Table Attacks}\\
              \, \,\,\footnotesize{Combines the Birthday attack with a table of precomputed hashes.}\\
              \, \,\,\footnotesize{This is then used to compare to a password hash list.}\\
              -- \blu{Sniffer Attacks}\\
              \, \,\,\footnotesize{Threat actor analyzes data sent over network with a sniffer tool.}\\
              \, \,\,\footnotesize{A good example for this is wireshark}\\
              \, \,\,\footnotesize{Can be mitigated by using encryption and One-Time passwords}\\
              \, \,\,\footnotesize{encryption makes the data useless and One-Time passwords are as well}\\
              -- \blu{Spoofing Attacks}\\
              \, \,\,\footnotesize{Pretending to be something/someone else. Ex. pretending to be router.}\\
              -- \blu{Social Engineering Attacks}\\
              \, \,\,\footnotesize{Gaining and then misusing trust of someone. }\\
              \, \,\,\footnotesize{*Indian accent* you get refund if you buy me 2 cards from target}\\
              -- \blu{Shoulder Surfing (Social Engineering)}\\
              \, \,\,\footnotesize{Reading information on a screen from a persons back.}\\
              -- \blu{Phishing (Social Engineering)}\\
              \, \,\,\footnotesize{Trick a Person to click on a fake link to log in, giving the attacker}\\
              \, \,\,\footnotesize{all the credentials to log-in}\\
              -- \blu{Spear Phishing (Social Engineering)}\\
              \, \,\,\footnotesize{Targeted Phishing at a group. Ex. Employees at company x.}\\
              -- \blu{Whaling (Social Engineering)}\\
              \, \,\,\footnotesize{"Phishing für grosse Fisch" -> CEOs etc}\\
              -- \blu{Vishing (Social Engineering)}\\
              \, \,\,\footnotesize{Phishing via VOIP or instant messaging}\\
              }\\
              \hline
              \mc{\r{Protection Mechanism}\\
              -- \blu{Layering (defense in depth)}\\
              \, \,\,\footnotesize{Multiple Controls in Layers, if one fails, there are still the other ones}\\
              -- \blu{Abstraction}\\
              \, \,\,\footnotesize{Combining Objects into groups in order to simplify permission management}\\
              }\\
              \hline
            \end{tabular}
          \end{tabular}
          \hfill
    \end{table}
    \pagebreak
    \begin{table}[ht!]
          \begin{tabular}{ccc}
            \begin{tabular}[t]{|@{\hskip1pt}p{10.45cm}|}
              \hline
              \mc{
              -- \blu{Data Hiding}\\
              \, \,\,\footnotesize{Storing Objects in compartments that can't be seen / accessed }\\
              \, \,\,\footnotesize{by an unauthorized subject}\\
              -- \blu{Security through Obscurity}\\
              \, \,\,\footnotesize{Not informing a subject about an object, and hoping it will stay hidden}\\
              -- \blu{Encryption}\\
              \, \,\,\footnotesize{Turning data into gibberish via algorithms.}\\
            }\\
              \hline
              \mc{\r{Red-Team}\\
              Offensive Cyber-Security: simulate attacks\\
              -- \r{Think outside the box}\\
              \, \,\,\footnotesize{Find new ways and tools and attack systems to show the flaws}\\
              -- \r{Deep Knowledge of Systems}\\
              \, \,\,\footnotesize{Deep Knowledge about systems, flaws, exploits, methodologies,etc}\\
              \, \,\,\footnotesize{always up-to-date with technology}\\
              -- \r{Software Development}\\
              \, \,\,\footnotesize{Learn how to develop your own tools.}\\
              -- \r{Penetration testing}\\
              \, \,\,\footnotesize{Identify vulnerabilities and potential threats}\\
              -- \r{Social Engineering}\\
              }\\
              \hline
              \mc{\blu{Blue-Team}\\
              Defensive Cyber-Security: prevent attacks\\
              -- \blu{Organized and detail-oriented}\\
              \, \,\,\footnotesize{Prevent gaps by thinking about EVERYTHING}\\
              -- \blu{Cybersecurity Analysis and threat profile}\\
              \, \,\,\footnotesize{Assess the security of an organization. Create Risk/Threat Profiles. }\\
              -- \blu{Hardening Techniques}\\
              \, \,\,\footnotesize{Reduce the attack surface hackers might exploit}\\
              -- \blu{Knowledge about detection Software}\\
              \, \,\,\footnotesize{Be familiar with software that recognizes unauthorized actions}\\
              \, \,\,\footnotesize{low skill application would be rkhunter. }\\
              -- \blu{Security Information, Event Management (SIEM)}\\
              \, \,\,\footnotesize{Software that allows real-time analysis of security events}\\
              }\\
              \hline
              \mc{\textcolor{Yellow}{Yellow-Team}\\
              \footnotesize{Builds defensive software against attackers}\\
              }\\
              \hline
              \mc{\textcolor{Purple}{Purple-Team}\\
              \footnotesize{Improves organization security posture}\\
              }\\
              \hline
              \mc{\r{Linux}\\
              \footnotesize{\blu{Adding user:} usermod -m username -s /path/to/shell}\\
              \footnotesize{\blu{Change shell:} chsh -s /path/to/shell}\\
              \footnotesize{\blu{Change password:} passwd username}\\
              \footnotesize{\blu{Add user to group:} usermod -a -G groupname username}\\
              \footnotesize{\blu{Change file permission:} chmod permission file}\\
              \footnotesize{\blu{Change file owner:} chown file user/group }\\
              \footnotesize{\blu{Check IP address:} ip addr / ip -c -brie a }\\
              \footnotesize{\blu{DNS query:} dig domain (dig shitgaem.online) }\\
              -- \r{File System Permissions}\\
              \footnotesize{\ora{r = read, w = write , x = execute}}\\
              \footnotesize{Every single file has these attributes.}\\
              \footnotesize{These attributes are also duplicated for 3 different types of users.}\\
              \footnotesize{1. owner, 2. group of owner, 3. other}\\
              \footnotesize{This means the actual permission would look like this:}\\
              \pic{220620-3}\\
              \pic{220620-4}
              }\\
              \hline
            \end{tabular}
            \hspace{-0.1in}
            \begin{tabular}[t]{|@{\hskip1pt}p{10.45cm}|}
              \hline
              \mc{
              \footnotesize{\blu{Manually set IP address (abando):} edit /etc/network/interfaces}\\
              \footnotesize{\blu{Configure SSH:} edit /etc/ssh/sshd\_config}\\
              \, \,\footnotesize{For birbs sake, use a nonstandard port for ssh :)}\\
              \footnotesize{\blu{Check current shells:} ps}\\
              \footnotesize{\blu{Check current processes:} htop }\\
              \footnotesize{\blu{grep} read lines }\\
              \, \,\footnotesize{grep 'Warning' /var/log/rkhunter.log}\\
              \footnotesize{\blu{Read Lines:} awk 'sshd.*invalid user/ \{print \$11\}' auth.log}\\
              \footnotesize{\blu{bit-by-bit copy:} dd if=<media/partition> of<image\_files>}\\
              \, \,\footnotesize{mount - o ro,noexec,loop evidence\_01 /mnt/investigation}\\
              \, \,\footnotesize{mount with read only and no execution}\\
              \footnotesize{\blu{Check recently changed files } ls -lasrt }\\
              }\\
              \hline
              \mc{\r{PID}\\
              \footnotesize{The unique identifier the kernel gives each process}\\
              \footnotesize{This shows both background and foreground applications}\\
              }\\
              \hline
              \mc{\r{logs}\\
              \footnotesize{Logs capture every single action on linux, this can be used to detect bad actors.}\\
              \footnotesize{However, logs can also be spoofed, which means you always have to be sure, that}\\
              \footnotesize{everything is being logged, and that no logs have been tampered with.}\\
              \footnotesize{notable logging systems/files: syslog, rsyslog,var/log ,auth.log}\\
              \r{\footnotesize{The shred command with -f and -n force deletes log files.}}\\
              \footnotesize{Mainly used like this: shred -f -n 15 /var/log/auth.log*}\\
              \footnotesize{This shreds 15 lines from every log file with the name auth.log(something)}\\
              \footnotesize{other things to look out for:}\\
              \footnotesize{-- \blu{set-UID} Rogue Files}\\
              \footnotesize{-- \blu{Directories with .something} Hidden...}\\
              \footnotesize{-- \blu{Regular files in the /dev directory}}\\
              \footnotesize{-- \blu{Recently modified files ls -lasrt}}\\
              }\\
              \hline
              \mc{\r{Schedules Tasks}\\
              \footnotesize{Schedules tasks can be written either in cron.d or with systemd}\\
              }\\
              \hline
              \mc{\r{IP-Tables}\\
                \footnotesize{name one reason not to use ufw...}\\
                \footnotesize{\blu{Flush all rules:} iptables -F}\\
                \footnotesize{\blu{Block Input:} iptables -P INPUT DROP}\\
                \footnotesize{\blu{Block Output:} iptables -P OUTPUT DROP}\\
                \footnotesize{\blu{Block Forward:} iptables -P FORWARD DROP}\\
                \footnotesize{\blu{Allow Port:} iptables -A INPUT/OUTPUT -p port ...other shit...}\\
                \footnotesize{\blu{Show rules:} iptables -L -n -v -- line-numbers}\\
              }\\
              \hline
              \mc{\r{Sticky Bit}\\
              \footnotesize{This is a single bit in front of rwx. -> 1777 sticky set, 0777 sticky not set}\\
              \footnotesize{The interpretation of this bit depends on the file type}\\
              \footnotesize{For directories, it means that any files within that folder}\\
              \footnotesize{May only be renamed or deleted by the owner.}\\
              \footnotesize{For files this bit is deprecated!}\\
              }\\
              \hline
              \mc{\r{Security Enhanced Linux (SELinux)}\\
              \footnotesize{This is a module created by the NSA that implements types,}\\
              \footnotesize{which mark files based on the type of a subject.}\\
              \footnotesize{Ex. a top-secret process can create a file with chmod 777,}\\
              \footnotesize{but a confidential process still can't open it.}\\
              \ora{\footnotesize{This is called MLS in SELinux and is related to Multi Category Security (MCS)}}\\
              }\\
              \hline
              \mc{\r{Snort}\\
              Detection software like rkhunter\\
              }\\
              \hline
              \mc{\r{NetCat}\\
              \footnotesize{This can be used for anything dealing with TCP and UDP.}\\
              \footnotesize{You can also use it to control compromised systems...}\\
              \blu{Reverse Shell}\\
              \footnotesize{The idea is, since the starting connection comes from the victim,}\\
              \footnotesize{Not only do we not have NAT and firewall problems, the connection}\\
              \footnotesize{also looks more legit than when we connect.}\\
              \footnotesize{This gives a hacker some sort of legitimacy on that system.}\\
              }\\
              \hline
              \mc{\r{Scapy}\\
              \footnotesize{Tool used to send, sniff, dissect and forge IP packets.}\\
              \footnotesize{You can probe, scan and attack networks}\\
              \footnotesize{You can attack signature for IDS/IPS systems}\\
              }\\
              \hline
            \end{tabular}
          \end{tabular}
          \hfill
    \end{table}
    \pagebreak
    \begin{table}[ht!]
          \begin{tabular}{ccc}
            \begin{tabular}[t]{|@{\hskip1pt}p{10.45cm}|}
              \hline
              \mc{\r{Encryption Terms}\\
                -- \blu{Plain Text:} \footnotesize{unencrypted message}\\
                -- \blu{Ciphertext:} \footnotesize{encrypted message}\\
                -- \blu{Cipher:} \footnotesize{Algorithm used to encrypt}\\
                -- \blu{Cryptographic key:} \footnotesize{Just a number to decrypt a message}\\
                \, \,\,\footnotesize{The range is defined by the algorithm. \(0 \text{ to } 2^n\)}\\
                \, \,\,\footnotesize{A key with 128 bits would have a range of: \(0 \text{ to } 2^{128}\)}\\
                \, \,\,\footnotesize{\r{It is critical to keep the keys secret!}}\\
                -- \blu{One-Way Function:}\\
                \, \,\,\footnotesize{mathematical function that produces output in a way that}\\
                \, \,\,\footnotesize{ input can't be retrieved.}\\
                \, \,\,\ora{\footnotesize{There is no TRUE One-Way-Function}}\\
                \, \,\,\footnotesize{Cryptography works on the believe that it can't be broken RIGHT NOW}\\
                \, \,\,\footnotesize{However, this does not mean it will stay so forever, see already broken ciphers}\\
                -- \blu{Reversability:} \footnotesize{The option of encryption....}\\
                -- \blu{Nonce:} \footnotesize{A Public,unique One-Time-Use Number}\\
                \, \,\,\footnotesize{Makes sure a key is not re-used twice!}\\
                -- \blu{Initialization Vector (IV):} \footnotesize{A random bit string}\\
                \, \,\,\footnotesize{Same length as the block size and is 'XORed with the message'}\\
                \, \,\,\footnotesize{IVs are used to create a unique ciphertext with the same key}\\
                -- \blu{Confusion:} \\
                \, \,\,\footnotesize{This is the case when encryption is so complicated,}\\
                \, \,\,\footnotesize{that merely reforming the string doesn't reveal the message}\\
                \, \,\,\footnotesize{Aka bruteforce doesn't work anymore.}\\
                -- \blu{Diffusion:} \\
                \, \,\,\footnotesize{A change in the plaintext will result in multiple changes in the ciphertext.}\\
                -- \blu{The Kerckhoff's Principle} \\
                \, \,\,\footnotesize{This means everything about the system is public but the key}\\
                \, \,\,\footnotesize{It therefore requires the system to be secure even under these circumstances}\\
                \, \,\,\footnotesize{The idea is that public algorithms may hasten the improvements on them}\\
                -- \blu{Permutation} \footnotesize{Swapping Bytes around}\\
                -- \blu{Byte Substitution} \footnotesize{Replacing bytes with others}\\
                -- \blu{SP-Networks} \footnotesize{algorithm that uses repeated Permutations and Substitutions}\\
                \, \,\,\footnotesize{Permutations and Substitutions are combined to a round}\\
                \, \,\,\footnotesize{Rounds are then repeated many times}\\
              }\\
              \hline
              \mc{\r{Caesar Cipher or ROT3}\\
              \footnotesize{One of the earliest encryption systems}\\
              \footnotesize{Simply shifts a chracter by 3  A to D, B to E...}\\
              }\\
              \hline
              \mc{\r{One-Time-Pad}\\
                \footnotesize{Create a key with the same length as the message}\\
                \footnotesize{XOR each message bit with each key bit}\\
                \footnotesize{\r{This Cipher is UNBREAKABLE!}}\\
                \footnotesize{However it is not practical.. 1GB file 1GB key...}\\
                \footnotesize{No proper way to transmit, store a key}\\
                \footnotesize{Using a key twice == Cipher broken}\\
              }\\
              \hline
              \mc{\r{Symmetric Cryptography}\\
              \pic{220620-5}\\
              -- \blu{Same key for encrypting and decrypting}\\
              -- \blu{Shared key for all parties involved!}\\
              \, \,\,\footnotesize{If one leaks the key, the cipher is broken!}\\
              -- \blu{Doesn't confirm identity!}\\
              \, \,\,\footnotesize{Anyone who has the key can pretend do be another}\\
              }\\
              \hline
            \end{tabular}
            \hspace{-0.1in}
            \begin{tabular}[t]{|@{\hskip1pt}p{10.45cm}|}
              \hline
              \mc{\r{Stream Ciphers}\\
              \pic{220620-6}\\
            \footnotesize{\blu{+ Encryption of long continuous streams, of possible unknown length}}\\
              \footnotesize{\blu{+ Extremely fast with low memory footprint, ideal for low power devices}}\\
              \footnotesize{\blu{+ If designed well, it can seek to any location in the stream}}\\
              \footnotesize{\r{-- The keystream must appear statistically random}}\\
              \footnotesize{\r{-- You must never reuse a key + nonce}}\\
              \footnotesize{\r{-- Stream ciphers do not protect the ciphertext (no guaranteed integrity)}}\\
              }\\
              \hline
              \mc{\r{Substitution / Permutation box}\\
              \pic{220620-7}
              }\\
              \hline
              \mc{\r{Block Cipher}\\
              \footnotesize{Takes in an input of a fixed size and returns an output of the same size}\\
              -- \blu{Diffusion and Confusion}\\
              -- \blu{SP-Network}\\
              \gre{Advanced Ecryption Standard (AES) is a Block Cipher}\\
              Here is a Block Cipher works:\\
              \pic{220620-8}\\
              }\\
              \hline
              \mc{\r{AES}\\
              \footnotesize{Built around the Rijndael algorithm}\\
              \footnotesize{Superceedes the DES as a standard}\\
              -- \blu{SP-Network with 128-bit block size}\\
              \, \,\,\footnotesize{>> Key length 128,192,256 bit}\\
              \, \,\,\footnotesize{>> 10, 12 or 14 rounds}\\
              \, \,\,\footnotesize{>> Each Round: Substitute Bytes, ShiftRows, MixColumns, KeyAddition}\\
              \pic{220620-9}\\
              }\\
              \hline
            \end{tabular}
          \end{tabular}
          \hfill
    \end{table}
    \pagebreak
    \begin{table}[ht!]
          \begin{tabular}{ccc}
            \begin{tabular}[t]{|@{\hskip1pt}p{10.45cm}|}
              \hline
              \mc{
              \vspace{-0.1in}
              \\
              \pic{220620-10}\\
              \pic{220620-11}\\
              \pic{220620-12}\\
              }\\
              \hline
              \mc{\r{Block Cypher with random input length}\\
                \footnotesize{Obviously we want to encrypt more than just one block}\\
                \footnotesize{How do we do that?}\\
                -- \blu{Electronic Code Block (ECB)}\\
                -- \blu{Cipher Block Chaining (CBC)}\\
                -- \blu{Counter Mode (CTR)}\\
              }\\
              \hline
              \mc{\r{Electronic Code Block (ECB)}\\
              \footnotesize{Just encrypt block after block.}\\
              \footnotesize{However, this might give away the bigger picture}\\
              \footnotesize{Aka the pattern of the data is still visible!!}\\
              \pic{220620-13}\\
              }\\
              \hline
              \mc{\r{Cipher Block Chaining (CBC)}\\
              \footnotesize{XOR each output with the next block.}\\
              \footnotesize{\ora{not parallelizable}, but more secure than ECB}\\
              \pic{220620-14}\\
              }\\
              \hline
              \mc{\r{Counter Mode (CTR)}\\
              \footnotesize{Encrypt a counter (Nonce) to produce a stream cypher}\\
              \footnotesize{Encrypted Nonce is then XORed with the plain text}\\
              \footnotesize{\ora{parallelizable!!}}\\
              \footnotesize{\r{Standard for all AES ciphers!}}\\
              \pic{220620-15}\\
              }\\
              \hline
            \end{tabular}
            \hspace{-0.1in}
            \begin{tabular}[t]{|@{\hskip1pt}p{10.45cm}|}
              \hline
              \mc{\r{Cons and Pros of symmetric Cryptography}\\
              \r{-- Key distribution}\\
              \, \,\,\footnotesize{Keys have to be shared securely, anyone who has the key can }\\
              \, \,\,\footnotesize{encrypt and decrypt all messages (sent by that key)}\\
              \r{-- No Nonrepudation}\\
              \, \,\,\footnotesize{Because everyone who has the key can encrypt and decrypt,}\\
              \, \,\,\footnotesize{there is no guarantee that this message is from a trusted source.}\\
              \r{-- No message Integrity}\\
              \, \,\,\footnotesize{If the message gets damaged, then there is no recovery inbuilt.}\\
              \gre{+ Speed}\\
              \, \,\,\footnotesize{often 1000 to 10000 times faster than asymmetric algorithms}\\
              \, \,\,\footnotesize{Lots of processors have an AES intruction set.}\\
              Alternatives: Chacha20 cipher\\
              }\\
              \hline
              \mc{\r{Diffie-Hellman}\\\
              \footnotesize{With this method the problem of sharing a key over the internet is solved}\\
              \footnotesize{We can now do so without any worries of giving a malicious third party access}\\
              \footnotesize{Every TLS handshake is in some way powered by this.}\\
              \footnotesize{\ora{We are not actually exchanging a key, only some mathematical part of it!!}}\\
              }\\
              \hline
              \mc{\r{Discrete Logarithm}\\
              \footnotesize{A logarithm that is implicit when using mod.}\\
              \r{\(a^b (mod \, n ) = c == b = \log_{a,n} (c)\)}\\
              \footnotesize{\ora{These are harder to calculate than regular ones!}}\\
              \footnotesize{Which is why they are used in Diffie-Hellman!}\\
              \pic{220620-16}
              }\\
              \hline
              \mc{\r{Primitve Root}\\
              \footnotesize{A number g is a primitive root of p when:}\\
              \r{\(\bigvee_{k=0}^{p} g^k \, mod \, p = \text{Distinct from each other} \)}\\
              \footnotesize{\ora{In other words, every single result from the modulo must be different!}}\\
              }\\
              \hline
              \mc{\r{Diffie-Hellman Example}\\
              \blu{1. Agree on Parameters}\\
              \footnotesize{Alice and Bob agree on a \r{large prime p} and a second \gre{prime / primitive root g}}\\
              \footnotesize{p is usually at least 2048 or 4096 bits}\\
              \blu{2. Select Private Numbers}\\
              \footnotesize{Alice picks the random number \r{a}}\\
              \footnotesize{Bob picks the random number \r{b}}\\
              \footnotesize{>> private numbers are between 1 and \r{p}}\\
              \footnotesize{>> If p is 2048 bits, then you are guessing a number with 2048 bits, have fun :)}\\
              \footnotesize{>> They \r{NEVER tell each other the private number}}\\
              \blu{3. Alice and Bob each calculate a Public Key}\\
              \footnotesize{>> Alice calculates key:\ora{\(g^a \, mod \, p\)}}\\
              \footnotesize{>> Bob calculates key:\ora{\(g^b \, mod \, p\)}}\\
              \, \,\,\footnotesize{\r{Because we are using a discrete logarithm, it is mathematically infeasible}}\\
              \, \,\,\footnotesize{\r{to get the private numbers by calculation.}}\\
              \blu{4. Alice and Bob exchange the Public Keys}\\
              \footnotesize{These are simple the calculated versions of keys.}\\
              \blu{5. Alice and Bob calculate the shared key}\\
              \footnotesize{for both this is: \ora{\(g^{ab} \, mod \, p\)}}\\
              \footnotesize{The shared key is therefore the same for both parties}\\
              \blu{6. Calculate Master Secret}\\
              \footnotesize{The shared key is also called the Pre-Master}\\
              \footnotesize{This is because the shared key is quite big and}\\ 
              \footnotesize{not often used to encrypt directly}\\
              \footnotesize{It is instead used to control sessions after it has been hashed}\\
              \footnotesize{The hashed shared key is then called the Master Secret}\\
              }\\
              \hline
            \end{tabular}
          \end{tabular}
          \hfill
    \end{table}
    \pagebreak
    \begin{table}[ht!]
          \begin{tabular}{ccc}
            \begin{tabular}[t]{|@{\hskip1pt}p{10.45cm}|}
              \hline
              \mc{
              \vspace{-0.1in}
              \\
              \pic{220620-17}\\
              }\\
              \hline
              \mc{\r{Eliptic Curve Cryptography}\\
              \pic{220620-18}
              }\\
              \hline
              \mc{\r{Ephemeral Mode}\\
              \footnotesize{For Diffie-Hellman this means calculating a new key for every session}\\
              \footnotesize{This is also called \r{Perfect Forwarding Secrecy}}\\
              \footnotesize{The reason for this is that calculating this key costs close to no power}\\
              \footnotesize{So why not just create a new one for every session to increase security?}\\
              \footnotesize{This is usually done every browser refresh etc.}\\
              \footnotesize{It is also automatically done after a certain time, again to improve security}\\
              }\\
              \hline
              \mc{\r{RSA Rivest-Shamir-Adleman}\\
              -- \blu{Public Key cryptosystem}\\
              -- \blu{widely used for secure data transmission}\\
              -- \blu{Most common method for public cryptography}\\
              -- \blu{Offers Nonrepudation!!}\\
              -- \blu{Reverseable Keys}\\
              \, \,\,\footnotesize{Both the public key and the private key can be used to encrypt or decrypt.}\\
              \, \,\,\footnotesize{It just has to be the inverse at the other end. }\\
              \, \,\,\footnotesize{Encrypt pub-key -> decrypt priv-key | encrypt priv-key -> decrypt pub-key}\\
              >> \ora{The reverseable keys lead to 2 operating Modes}\\
              \gre{1. Encrypt so only the receiver can read}\\
              \footnotesize{If I want to send a message to the server that only said server can read, then}\\
              \footnotesize{I can encrypt my message with the servers public key.}\\
              \footnotesize{The server then uses its private key to decrypt. \ora{Not even you can decrypt it :P}}\\
              \gre{2. Nonrepudation Mode}\\
              \footnotesize{By encrypting with the private key, everyone who has the public key can read.}\\
              \footnotesize{However, it guarantees that said message is from you, and no one else!}\\
              }\\
              \mc{\r{Prime Factorization}\\
              \footnotesize{Every non-prime number has a prime factorization}\\
              \footnotesize{This means every non-prime number can be created by multiplying x primes}\\
              \footnotesize{Prime factorization of 30 -> 5 * 3 * 2}\\
              \footnotesize{\r{Calculating the prime factorization is EXTREMELY HARD for big numbers}}\\
              \footnotesize{In other words, it is not feasible to calculate it with a current Computer}\\
              \footnotesize{Which is why it is used by the RSA algorithm}\\
              }\\
              \hline
              \mc{\r{RSA Functionality}\\
              \begin{tabular}{cc}
              \mc{
              Public keys: \blu{e} \r{n}\\
              Prime factors: p1,p2\\
              Private key: \gre{d}\\
              Message: \ora{m}\\
              }&
              \mc{
              \footnotesize{\blu{e} is almost always 3 or 65537}\\
              \footnotesize{\r{n} is a random Prime factorization of 4096 bits}\\
              \footnotesize{\(\gre{d} = \dfrac{k * \phi ( \r{n}) + 1}{\blu{e}}\)}\\
              \footnotesize{where k is a random integer}\\
              \footnotesize{\r{p1,p2,d must be private!!}}
              }\\
              \end{tabular}
              }\\
              \hline
            \end{tabular}
            \hspace{-0.1in}
            \begin{tabular}[t]{|@{\hskip1pt}p{10.45cm}|}
              \hline
              \mc{
              \begin{tabular}{ccc}
              \mc{
              -- \blu{Encrypting Public:}\\
              \, \,\,\(c = \)\ora{\(m^{\blu{e}} \, \)}\( mod \, \)\r{\(n\)}\\
              -- \blu{Decrypting Private:}\\
              \, \,\,\ora{\(m\)}\( = c^{\gre{d}} \, mod \, \)\r{\(n\)}\\
              }&
              \mc{
              -- \blu{Encrypting Private:}\\
              \, \,\,\(c = \)\ora{\(m^{\gre{d}} \, \)}\( mod \, \)\r{\(n\)}\\
              -- \blu{Decrypting Public:}\\
              \, \,\,\ora{\(m\)}\( = c^{\blu{e}} \, mod \, \)\r{\(n\)}\\
              }&
              \mc{
              -- \blu{Combined}\\
              \, \,\,\ora{\(m^{ed} \, mod \, n = m\)}\\
              \\
              \\
              }\\
              \end{tabular}
              }\\
              \hline
              \mc{\r{The PHI Function}\\
              \pic{220621-1}\\
              \pic{220621-2}\\
              }\\
              \hline
              \mc{\r{Using RSA}\\
              \gre{1. Choose two very large Primes}\\
              \pic{220621-3}\\
              \gre{2. Calculate PHI}\\
              \r{\(\phi (n) = \phi (a) * \phi (n)\)}\\
              \r{a and b are the prime factorization primes!}\\
              \ora{if number is prime, then: \(\phi (a) = a - 1 \)}\\
              \footnotesize{Ex. n = 77, a = 11, b = 7 -> \(\phi (77) = \phi (11) * \phi (7) = 10 * 6 = 60\)}\\
              \gre{3. Choose k and e to calculate d}\\
              \r{\(d = \dfrac{k * \phi (n) + 1}{e}\) with k being an integer}\\
              \footnotesize{Ex. n=55, e=7, k=4, \(\phi (n) = 40\) \(d = \dfrac{4 * \phi(55) + 1}{7} = 23 \)}\\
              \gre{4. Encrypt and Decrypt...}\\
              }\\
              \hline
              \mc{\r{RSA quirks}\\
              -- \blu{Very weak with short messages}\\
              \, \,\,\footnotesize{\r{To mitigate this, padding is added}}\\
              \, \,\,\footnotesize{Optimal assymetric Encryption Padding (OAEP) is used}\\
              \, \,\,\footnotesize{pseudo random padding that introduces an IV then hashes it}\\
              \, \,\,\footnotesize{Server has to create same padding to check if it matches up}\\
              -- \blu{Not common to encrypt with RSA!}\\
              \, \,\,\footnotesize{TLS used RSA before but no longer.}\\
              \, \,\,\footnotesize{RSA is used more for signing! Something that Diffe can't!!}\\
              -- \r{RSA is 1000x slower than symmetric crypto systems!!}\\
              -- \blu{Message Integrity}\\
              \, \,\,\footnotesize{>> Message is first hashed to shorten it}\\
              \, \,\,\footnotesize{>> Said hash is sent alongside the message}\\
              \, \,\,\footnotesize{>> Receiver hashes the message as well on arrival,}\\
              \, \, \,\footnotesize{  to check whether the 2 hashes match}\\
              \, \,\,\footnotesize{\ora{Simply decrypting the hash is sadly not possible.}}\\
              }\\
              \hline
              \mc{\r{Digital Signature Challenges}\\
              \footnotesize{Before communicating with a server, we want said server to prove}\\
              \footnotesize{its identity,  we do this by sending the server a message that}\\
              \footnotesize{the server then has to sign with its private key.}\\
              \footnotesize{\ora{This challenge is also part of Transport Layer Security (TLS)}}\\
              }\\
              \hline
              \mc{\r{Digital Signature Algorithm (DSA)}\\
              \footnotesize{RSA will soon become unusable due to the ever growing messages.}\\
              \footnotesize{In order to solve this the DSA has been developed, which is much faster!}\\
              \footnotesize{\ora{The downside is that it can't be used to encrypt, but we have Diffie for that}}\\
              \footnotesize{\gre{It also supports Elliptic Curves as it has similar mathematics as Diffie}}\\
              }\\
              \hline
            \end{tabular}
          \end{tabular}
          \hfill
    \end{table}
    \pagebreak
    \begin{table}[ht!]
          \begin{tabular}{ccc}
            \begin{tabular}[t]{|@{\hskip1pt}p{10.45cm}|}
              \hline
              \mc{\r{Hash Function}\\
              -- \blu{Fixed Length}\\
              \, \,\,\footnotesize{A hash function always has a fixed length. Meaning that if we have a 128 bit}\\
              \, \,\,\footnotesize{hash function, then it will always return 128 bits}\\
              \, \,\,\footnotesize{no matter how many inputs we enter}\\
              \begin{tabular}{cc}
              \mc{
              -- \blu{Iterative}\\
              \, \,\,\footnotesize{We iteratively hash over the message }\\
              \, \,\,\footnotesize{by block until we get to the end}\\
              \, \,\,\footnotesize{Then we take the current hash as the final hash.}\\
              }&
              \hspace{-0.2in}
              \mc{
              \pic{220621-4}\\
              }\\
              \end{tabular}\\
              -- \blu{Indistinguishable from Noise}\\
              \, \,\,\footnotesize{The output of a hash function should not be interpretable.}\\
              -- \blu{Diffusion}\\
              \, \,\,\footnotesize{Best way is to make small changes in the message }\\
              \, \,\,\footnotesize{in a big change in the hash.}\\
              -- \blu{Balance between speed and security}\\
              \, \,\,\footnotesize{A good hash function needs to be quick enough to use}\\
              \, \,\,\footnotesize{But also needs to offer proper protection!!}\\
              -- \blu{Needs to be infeasible to Revert}\\
              \, \,\,\footnotesize{Just like with encryption, there is no way to guarantee irreversibility}\\
              \, \,\,\footnotesize{We can however make it infeasible to do this with current computers!!}\\
              -- \blu{few Collisions}\\
              \, \,\,\footnotesize{The best would be to have 0 collisions, but that is impossible to guarantee}\\
              \, \,\,\footnotesize{Instead we simply push the probability to a very, very small number!}\\
              -- \blu{Used for Integrity}\\
              \, \,\,\footnotesize{Often a message gets corrupted, or even attacked by a threat actor}\\
              \, \,\,\footnotesize{In both cases we need a way to check integrity.}\\
              \, \,\,\footnotesize{We do this by hashing the message and sending the hash alongside it}\\
              \, \,\,\footnotesize{The receiver can then hash the message as well and compare the hashes.}\\
              }\\
              \hline
              \mc{\r{Current Hashing Algorithms}\\
              -- \blu{SHA-2 256/512}\\
              \, \,\,\footnotesize{\ora{Current standard!} }\\
              -- \blu{SHA-3}\\
              \, \,\,\footnotesize{Same quality as SHA-2, designed as a backup in case SHA-2 is broken}\\
              \pic{220621-5}\\
              }\\
              \hline
              \mc{\r{Hashes and Passwords}\\
              \footnotesize{These hashes alone are not good enough for passwords!}\\
              \footnotesize{Hacker will simly use hashed passwords to compare to your hash table}\\
              \gre{Solution: Password-Based Key Derivation Function 2 (PBKDF2)}\\
              \footnotesize{\blu{This hashes the password with SHA-2 5000 times.}}\\
              \footnotesize{It is therefore 5000times slower, but also much more secure!}\\
              \footnotesize{Inside password managers, the iterations go up to 100 000 times!}\\
              \footnotesize{The alternative is called 'Blowfish'}\\
              }\\
              \hline
              \mc{\r{HMAC and the length extension attack}\\
              \footnotesize{using only 1 hash might give attackers the opportunity to}\\
              \footnotesize{add more data to the end of the hash, as long as the hash is still "random"}\\
              \footnotesize{With HMAC we split the message in half and hash each side with both}\\
              \footnotesize{The public and the private key. This prohibits length extension attacks.}\\
              \footnotesize{\ora{MAC means Message Authentication Code, not Multimedia Access Control lmao}}\\
              }\\
              \hline
              \mc{\r{Message Authentication Code (MAC)}\\
              \begin{tabular}{cc}
                \mc{\hspace{-0.12in}
              \pic{220621-6}\\
              \r{The MAC is the hash that is signed.}
              }& \hspace{-0.35in}
              \mc{\pic{220621-17}}\\
              \end{tabular}\\
              }\\
              \hline
            \end{tabular}
            \hspace{-0.1in}
            \begin{tabular}[t]{|@{\hskip1pt}p{10.45cm}|}
              \hline
              \mc{\r{Certification Terms}\\
              -- \blu{Public Key Infrastructure}\\
              \, \,\,\footnotesize{\gre{The entire Processof requesting, receiving and holding a certificate}}\\
              \, \,\,\footnotesize{\ora{This means the Environment: hardware, software, policies, roles, procedures.}}\\
              -- \blu{Certificate Signing Request (CSR)}\\
              \, \,\,\footnotesize{A request to a CA to sign your certificate}\\
              -- \blu{Certification Authority (CA)}\\
              \, \,\,\footnotesize{CA is the signing third party organization: letsencrypt, Google, Digicert, etc.}\\
              -- \blu{Root Certificate}\\
              \, \,\,\footnotesize{Public key of a CA that will be used to decrypt and check the}\\
              \, \,\,\footnotesize{Certificates of websites. Note the CA must match!}\\
              -- \blu{Trust Service Provider (TSP)}\\
              \, \,\,\footnotesize{An organization that provides Certificates, validation, etc}\\
              \, \,\,\footnotesize{letsencrypt, digicert etc.}\\
              -- \blu{Trust Services (TS)}\\
              \, \,\,\footnotesize{All services of a TSP -> CA,VA,RA}\\
              -- \blu{Validation Authority (VA)}\\
              \, \,\,\footnotesize{TSPs Validation part}\\
              -- \blu{Registration Authority (RA)}\\
              \, \,\,\footnotesize{TSPs Registration part}\\
              }\\
              \hline
              \mc{\r{Digital Certificate}\\
                \ora{A digital certificate is nothing more than data that includes:}\\
              -- \blu{Information about the key}\\
              -- \blu{The identity of the owner (subject)}\\
              -- \blu{Issuer information}\\
              \footnotesize{With public and private keys, we can only guarantee that the message}\\
              \footnotesize{comes from PC x, but not that said PC is actually trustworthy!}\\
              \footnotesize{Or that said PC is actually from the person it is claiming to be!!}\\
              \footnotesize{This is where digital certificates come into play}\\
              \footnotesize{They are signed by a third party and are therefore considered to be trustworthy}\\
              \footnotesize{\ora{The third party is usually a Public Key infrastructure}}\\
              }\\
              \hline
              \mc{\r{Qualities of Certificates}\\
              \ora{There are different levels of certificate quality,}\\
              \ora{the higher, the more trust you can have in a certificate:}\\
              -- \blu{1. Domain Validated (DV) -- 2.23.140.1.2.1}\\
              \, \,\,\footnotesize{It is issued after the owner hash shown proof that they}\\
              \, \,\,\footnotesize{have the right to use their domain. (automatic check)}\\
              \, \,\,\footnotesize{letsencrypt only offers this level! 80\% of internet}\\
              -- \blu{2. Organization Validated (OV) --2.23.140.1.2.2}\\
              \, \,\,\footnotesize{This is issued after the CA has validated the company name,}\\
              \, \,\,\footnotesize{domain name and other information through public databases}\\
              \, \,\,\footnotesize{Used by bigger companies and Emails}\\
              -- \blu{3. Extended Validation (EV) -- 2.23.140.1.1}\\
              \, \,\,\footnotesize{This requires strict authentication procedure}\\
              \, \,\,\footnotesize{Used by banks, Governments, etc}\\
              -- \blu{4. Qualified Website Authentication Certificate (QWAC)}\\
              \, \,\,\footnotesize{Not supported by browsers}\\
              \, \,\,\footnotesize{mixture of OV and EV, wanted standard}\\
              }\\
              \hline
              \mc{\r{Certificate Issuance}\\
              \gre{1. Create a private/public key for RSA or DSA}\\ 
              \gre{2. Create a CSR and send it to a RA}\\
              \, \,\,\footnotesize{in Base64-PEM format}\\
              \gre{3. The RA does identification checks}\\
              \, \,\,\footnotesize{These vary heavily depending on the type of CA and}\\
              \, \,\,\footnotesize{the amount of trust you want to have from a client}\\
              \, \,\,\footnotesize{letsencrypt is free and easy to use, but offers only minimal identification!}\\
              \, \,\,\footnotesize{Digicert and others offer in person identification and liability}\\
              \, \,\,\footnotesize{However these also cost money and are more time consuming to create}\\
              \gre{4. The CA signs the public key, you now have a certificate}\\
              \, \,\,\footnotesize{Now every time someone visits your website, they see the }\\
              \, \,\,\footnotesize{certificate signed by the CA with their private key}\\
              \gre{5. The VA checks the validity of the certificate}\\
              }\\
              \hline
            \end{tabular}
          \end{tabular}
          \hfill
    \end{table}
    \pagebreak
    \begin{table}[ht!]
          \begin{tabular}{ccc}
            \begin{tabular}[t]{|@{\hskip1pt}p{10.45cm}|}
              \hline
              \mc{\r{Certificate Use}\\
              \footnotesize{When you navigate to a website, you get the certificate of said website during}\\
              \footnotesize{a TLS handshake, after that you need to decrypt the certificate with }\\
              \footnotesize{public key of the issuer CA. This key is stored within your browser or OS}\\
              \footnotesize{Then you can use this decrypted certificate, which in the end is just a public key}\\
              \footnotesize{To decrypt all messages sent by the server, which are now guaranteed to be from}\\
              \footnotesize{said server}\\
              \r{This is called Chain of Trust!}\\
              \footnotesize{\r{If you don't have the Signing CAs public key stored,}}\\
              \footnotesize{\r{then the connection is rejected!!}}\\
              }\\
              \hline
              \mc{\r{X509 Certificate}\\
              \pic{220621-23}\\
              -- \blu{Extension Requirements:}\\
              \pic{220621-24}\\
              -- \blu{Serial Number}\\
              \, \,\,\footnotesize{Positive Integer unique for each certificate issued by CA}\\
              -- \blu{Signature Algorithm}\\
              \, \,\,\footnotesize{Algorithm used by CA to sign the certificate}\\
              }\\
              \hline
              \mc{\r{x509 Validity Checks}\\
              -- \blu{Method 1: Certification Revocation List (CRL)}\\
              \, \,\,\footnotesize{Most browsers perform this check}\\
              \, \,\,\footnotesize{CRL Distribution point included in the certificate}\\
              -- \blu{Method 2: Online Certification Status Protocol (OCSP)}\\
              \, \,\,\footnotesize{The Authorative Information Access (AIA) field}\\
              \, \,\,\footnotesize{provides information about the CA.}\\
              \ora{Other Terms:}\\
              -- \gre{Certificate Policies}\\
              \, \,\,\footnotesize{Link to governance rules of the CA}\\
              -- \gre{Authority Key Identifier (AKI)}\\
              \, \,\,\footnotesize{Key identifier of the CA}\\
              -- \gre{Subject Alternative Name (SAN)}\\
              \, \,\,\footnotesize{various values associated with the certificate owner}\\
              -- \gre{Subject Key Identifier (SKI)}\\
              \, \,\,\footnotesize{Hash value of the certificate}\\
              }\\
              \hline
              \mc{\r{Certificate Pinning}\\
              \footnotesize{When a website pins a CAs root, or intermediate CA,}\\
              \footnotesize{it forces the browser to only use this specific certificate}\\
              \footnotesize{This prevents modified RootCAs from connecting to this website.}\\
              \footnotesize{The problem with this is that, when the pinned CA is compromised,}\\
              \footnotesize{there is no check the browser can do to stay secure.}\\
              \footnotesize{For this reason, the pinned CAs are usually only valid for 60 days.}\\
              \ora{Pinnable CAs}: root CA, Intermediate CA, End Certificate (SSH)
              }\\
              \hline
            \end{tabular}
            \hspace{-0.1in}
            \begin{tabular}[t]{|@{\hskip1pt}p{10.45cm}|}
              \hline
              \mc{\r{Root Store and CA/B Browser Forum}\\
              \footnotesize{The root store defined what TSP/CA we trust.}\\
              \footnotesize{But who decides what CA is in said store??}\\
              \footnotesize{While every browser has their own Root Store Policy,}\\
              \footnotesize{There is a forum that tries to unify this:}\\
              \pic{220621-25}
              }\\
              \hline
            \end{tabular}
          \end{tabular}
          \hfill
    \end{table}
    \pagebreak
    \begin{table}[ht!]
          \begin{tabular}{ccc}
            \begin{tabular}[t]{|@{\hskip1pt}p{10.45cm}|}
              \hline
              \mc{\r{Authenticated Encryption with associated data (AEAD)}\\
              -- \blu{is an AEAD protocol}\\
              \footnotesize{In order to guarantee confidentiality as well as integrity,}\\
              \footnotesize{we add a Message authentication code at the end of our message}\\
              \pic{220621-8}\\
              \blu{One such MAC is the Galois Counter Mode (GCM)}\\
              \footnotesize{It calculates a MAC over the message and the additional data}\\
              \pic{220621-9}
              }\\
              \hline
              \mc{\r{ChaCha20\_Poly1305}\\
              -- \blu{is an AEAD protocol}\\
              -- \blu{allowed on TLS 1.3}\\
              \, \,\,\footnotesize{The only one next to AES.}\\
              -- \blu{needs a Nonce as ChaCha20 is a Stream Cipher}\\
              -- \blu{Used for Android}\\
              \, \,\,\footnotesize{It is used mostly for speed purposes as ChaCha20 is faster}\\
              \, \,\,\footnotesize{when the CPU doesn't have an AES instruction set.}\\
              }\\
              \hline
              \mc{\r{Protocol Handshakes}\\
              \gre{1. Handshake phase}\\
              \, \,\,\footnotesize{>> Check Certificate with root CA certificate}\\
              \, \,\,\footnotesize{>> Agree on a set of cryptographic protocols: AES 123, DSA, RSA, etc}\\
              \, \,\,\footnotesize{>> Perform key exchange to obtain session keys and other values like IVs}\\
              \, \,\,\footnotesize{>> Verify authenticity using the public key}\\
              \gre{2. Transport/Record phase}\\
              \, \,\,\footnotesize{when sending encrypt packets with the agreed method}\\
              \, \,\,\footnotesize{\ora{always include a MAC for integrity!}}\\
              }\\
              \hline
              \mc{\r{Common Protocol Issues}\\
              -- \blu{Cipertexts that aren't secured with MAC}\\
              \, \,\,\footnotesize{This means the message can be altered in transport}\\
              -- \blu{Messages without a timestamp}\\
              \, \,\,\footnotesize{This means we can copy and resend this message }\\
              \, \,\,\footnotesize{from another PC!!}\\
              -- \blu{Protocols without public key -> without authenticity}\\
              \, \,\,\footnotesize{These are vulnerable to man in the middle attacks}\\
              -- \blu{Reuse of Nonce}\\
              \, \,\,\footnotesize{might lead to a broken cipher...}\\
              \pic{220621-10}\\
              \pic{220621-11}
              }\\
              \hline
            \end{tabular}
            \hspace{-0.1in}
            \begin{tabular}[t]{|@{\hskip1pt}p{10.45cm}|}
              \hline
              \mc{\r{Transport Layer Security (TLS)}\\
              -- \blu{Confidentiality}\\
              -- \blu{Integrity}\\
              -- \blu{Authentication}\\
              -- \blu{Fragmentation}\\
              -- \blu{Compression}\\
              -- \r{Primary protocol for HTTPS!!}\\
              \, \,\,\footnotesize{However, it can be used by everywhere!}\\
              -- \blu{Other:}\\
              \, \,\,\footnotesize{Versions: SSL1.0, SSL2.0, SSL3.0, TLS1.0, TLS1.1, TLS1.2, TLS1.3}\\
              \, \,\,\footnotesize{\r{DO NOT USE SSL! In fact USE TLS 1.3!!}}\\
              \, \,\,\footnotesize{\ora{>> currently at version 1.3}}\\
              \, \,\,\footnotesize{\ora{Previously called Secure Socket Layer (SSL)}}\\
              }\\
              \hline
              \mc{\r{TLS Connection}\\
              \, \,\,\footnotesize{>> peer to peer}\\
              \, \,\,\footnotesize{>> transient, they are not stored}\\
              \, \,\,\footnotesize{>> each connection is associated with exactly 1 session}\\
              -- \blu{Server and client random number}\\
              -- \blu{Server and client MAC private key}\\
              -- \blu{Server and client encryption private key}\\
              -- \blu{Initialization Vectors}\\
              \, \,\,\footnotesize{For block ciphers IV are maintained for each key.}\\
              -- \blu{Sequence numbers}\\
              \, \,\,\footnotesize{Same as with TCP}\\
              \, \,\,\footnotesize{When change cipher spec is used, the sequence number is reset to 0}\\
              \, \,\,\footnotesize{Sequence numbers may not exceed \(2^{64}-1\)}\\
              }\\
              \hline
              \mc{\r{TLS Session}\\
              \, \,\,\footnotesize{>> Defines the cryptographic parameters used}\\
              \, \,\,\footnotesize{>> Sessions are used to avoid expensive negotiation of security parameters}\\
              \, \, \,\footnotesize{on every new connection}\\
              -- \blu{Session Identifier}\\
              \, \,\,\footnotesize{an arbitrary bite sequence chosen by the server to identify }\\
              \, \,\,\footnotesize{an active or resumable session state}\\
              \, \,\,\footnotesize{\r{Can also be a previous session to resume!!!}}\\
              -- \blu{Peer certificate}\\
              \, \,\,\footnotesize{certificate of the peer, may be null}\\
              -- \blu{Compression method}\\
              \, \,\,\footnotesize{The algorithm used to compress data prior to encryption}\\
              -- \blu{Cipher spec}\\
              \, \,\,\footnotesize{Specifies: encryption algorithm, hash algorithm, hash size, Signing algorithm}\\
              -- \blu{Master secret}\\
              \, \,\,\footnotesize{shared key that is used for encryption -> Diffie-Hellman}\\
              -- \blu{Is resumable -> can be stored, unlike connections}\\
              \, \,\,\footnotesize{a flag that indicates that this session can be resumed}\\
              }\\
              \hline
              \mc{\r{TLS and TCP}\\
              \pic{220621-12}\\
              \pic{220621-13}\\
              }\\
              \hline
            \end{tabular}
          \end{tabular}
          \hfill
    \end{table}
    \pagebreak
    \begin{table}[ht!]
          \begin{tabular}{ccc}
            \begin{tabular}[t]{|@{\hskip1pt}p{10.45cm}|}
              \hline
              \mc{\r{TLS handshake}\\
              \footnotesize{The handshake allows server and client to negotiate}\\
              \footnotesize{all necessary algorithms and specifications}\\
              \footnotesize{\ora{Note that this example is for older versions of TLS, NOT TLS3.0!}}\\
              \pic{220621-14}\\
              \r{Rs and Rc are the random numbers}\\
              }\\
              \hline
              \mc{\r{Ephemeral Mode in TLS}\\
              \footnotesize{We have already learned that we calculate a new key for every session}\\
              \footnotesize{This now applies to TLS, which conveniently works with sessions as well}\\
              \footnotesize{This means we now calculate a new key for every session we have with the server}\\
              \pic{220621-15}\\
              }\\
              \hline
              \mc{\r{Master secret to Signing and encryption key}\\
              \footnotesize{The master secret is used to calculate all the other keys,}\\
              \footnotesize{this makes it possible that the server and client only need}\\
              \footnotesize{to share keys once, after that they can simply}\\
              \footnotesize{send a new random number for a new set of keys!! Super simple!!}\\
              \pic{220621-16}\\
              }\\
              \hline
              \mc{\r{Different Combinations in TLS}\\
              \footnotesize{With Diffie-Hellman you can encrypt only encrypt/decrypt}\\
              \footnotesize{With RSA/DSA you can sign or encrypt/decrypt}\\
              \footnotesize{The most usual combination is to use RSA/DSA and Diffie-Hellman!}\\
              \footnotesize{However, what do you do first? Encrypt or sign?}\\
              \footnotesize{You can choose this depending on your needs.}\\
              \footnotesize{If you want 100\% anonymity then use encrypt first}\\
              \footnotesize{If you want authenticity then use signing first.}\\
              \pic{220621-18}\\
              }\\
              \hline
              \mc{\r{TLS allowed Combinations:}\\
              \pic{220621-19}
              }\\
              \hline
            \end{tabular}
            \hspace{-0.1in}
            \begin{tabular}[t]{|@{\hskip1pt}p{10.45cm}|}
              \hline
              \mc{\r{TLS alerts:}\\
              \ora{Consists of 2 bytes: severity and code}\\
              -- \gre{Severity}\\
              \, \,\,\footnotesize{There are 2 severity levels, warning and fatal}\\
              \, \,\,\footnotesize{\r{on fatal the connection terminates immediately}}\\
              \, \,\,\footnotesize{Warning simply shows the alert code}\\
              -- \gre{Code}\\
              \, \,\,\footnotesize{Specific alert code.}\\
              }\\
              \hline
              \mc{\r{Heartbeat Protocol}\\
              -- \blu{Keep-alive Protocol}\\
              \, \,\,\footnotesize{assures the server that the client is still alive}\\
              -- \blu{Runs on top of TLS}\\
              \, \,\,\footnotesize{negotiated on handshake}\\
              \r{>> Heartbleed (Heartbeat exploit) <<}\\
              \footnotesize{This is done by requesting a bigger size package than the heartbeat actually is}\\
              \footnotesize{This can trick the server into sending data that it wasn't supposed to send.}\\
              \footnotesize{If done often enough and with lots of luck, the server might send data,}\\
              \footnotesize{that is critical, such as private keys and password hash-tables!}\\
              }\\
              \hline
              \mc{\r{TLS 1.3}\\
              -- \blu{Removal of broken features}\\
              \, \,\,\footnotesize{Removed broken ciphers like MD5, SHA-1, Kerberos, etc}\\
              \, \,\,\footnotesize{\ora{Removed compression and renegotiation}}\\
              \, \,\,\footnotesize{Remove static RSA/DH}\\
              -- \blu{Allows 1-RTT and 0-RTT}\\
              -- \blu{Improved security by using modern techniques}\\
              -- \blu{Privacy encrypt more of the protocol}\\
              \, \,\,\footnotesize{Almost all handshake messages}\\
              -- \blu{Continuity >> backwards compatibility}\\
              \pic{220621-20}\\
              \pic{220621-21}\\
              }\\
              \hline
            \end{tabular}
          \end{tabular}
          \hfill
    \end{table}
    \pagebreak
    \begin{table}[ht!]
          \begin{tabular}{ccc}
            \begin{tabular}[t]{|@{\hskip1pt}p{10.45cm}|}
              \hline
              \mc{\r{Penetration Testing}\\
              \footnotesize{Penetration testing is done to find and eliminate vulnerabilities}\\
              \footnotesize{Usually companies hire a specialist to do this work}\\
              \footnotesize{For legal reasons it is very important that a rigorous contract is signed}\\
              \footnotesize{The reason for this, is that attacking a system is illegal in many countries}\\
              \footnotesize{So you need explicit permission from the company to do so.}\\
              \footnotesize{The contracts are usually called Statements of Work}\\
              \footnotesize{They include: Activities to be performed, pen testing timeline}\\
              \footnotesize{Scope, and Location of work}\\
              \footnotesize{Usually you also have to sign a Non-Disclosure agreement}\\
              \footnotesize{This makes sure the vulnerabilities do not get public!}\\
              \pic{220621-22}
              }\\
              \hline
              \mc{\r{Zero-Day attacks}\\
              \footnotesize{This is an exploitation of a vulnerability that has been present since}\\
              \footnotesize{the creation of the program/service. However, it has not been found yet.}\\
              \footnotesize{The time frame of hackers figuring out the vulnerability and the developers}\\
              \footnotesize{patching said vulnerability, is called the \ora{window of vulnerability} }\\
              \footnotesize{>> Zero-Day is a big problem, because system administrators often wait}\\
              \footnotesize{  with applying updates to systems, which leaves a bigger window for attackers}\\
              }\\
              \hline
              \mc{\r{Malicious Code}\\
              \footnotesize{Code that is written to exploit vulnerabilities on networks, OSes, etc}\\
              -- \blu{Viruses,Worms,Trojans}\\
              \, \,\,\footnotesize{These are the most typical forms of malicious code.}\\
              \, \,\,\footnotesize{Most of them are only received through misuse of a computer}\\
              \, \,\,\footnotesize{\blu{Viruses usually have 2 functions, Propagation and Destruction}}\\
              }\\
              \hline
              \mc{\r{Sources of Malicious Code}\\
              -- \blu{Script Kiddie}\\
              \, \,\,\footnotesize{downloads crap and spams it into the internet}\\
              -- \blu{Drive-by Download}\\
              \, \,\,\footnotesize{This exploits a vulnerability where the browser downloads}\\
              \, \,\,\footnotesize{something in the background without your knowledge}\\
              -- \blu{Advanced Persistent Threat (APT)}\\
              \, \,\,\footnotesize{These are often state sponsored actors with large amount of resources.}\\
              \, \,\,\footnotesize{They often use Zero-Day attacks against very specific/high value targets}\\
              \, \,\,\footnotesize{An example is the RUAG Stuxnet attack.}\\
              }\\
              \hline
              \mc{\r{Virus Propagation Techniques}\\
              -- \blu{Master Boot Record Infection}\\
              \, \,\,\footnotesize{Very nasty, although the MBR is not big enough to host the virus}\\
              \, \,\,\footnotesize{It is perfect to launch it without a users knowledge}\\
              \, \,\,\footnotesize{It simply points to a location somewhere on a disk,}\\
              \, \,\,\footnotesize{where the virus is stored}\\
              -- \blu{File Infection}\\
              \, \,\,\footnotesize{These viruses change existing files to include their own code.}\\
              \, \,\,\footnotesize{Often executables are used, although there are many other variants.}\\
              \, \,\,\footnotesize{Most standard injectors are easily detected with antivirus software}\\
              -- \blu{Companion Virus}\\
              \, \,\,\footnotesize{A Virus that has a similar name to legitimate system services.}\\
              -- \blu{Service Injection}\\
              \, \,\,\footnotesize{Instead of regular file injection, these viruses inject themselves}\\
              \, \,\,\footnotesize{into critical system services in order to escape detection from }\\
              \, \,\,\footnotesize{antivirus software. Can be nasty to remove}\\
              -- \blu{Macro Infection}\\
              \, \,\,\footnotesize{This is an infection inside an application that has a scripting language}\\
              \, \,\,\footnotesize{For example Excel has the visual basic crap language inside it.}\\
              \, \,\,\footnotesize{A virus can be written in that language and be configured to}\\
              \, \,\,\footnotesize{launch at startup. Which means click on file = pwned}\\
              \, \,\,\footnotesize{Although this specific variant was fixed, there are }\\
              \, \,\,\footnotesize{always new ways in which this happens again and again}\\
              }\\
              \hline
            \end{tabular}
            \hspace{-0.1in}
            \begin{tabular}[t]{|@{\hskip1pt}p{10.45cm}|}
              \hline
              \mc{\r{Malware Variants}\\
              \r{Note: Malware can have multiple variants at once!}\\
              -- \blu{Multipartite Viruses}\\
              \, \,\,\footnotesize{Viruses that use more than 1 propagation technique}\\
              -- \blu{Stealth Viruses}\\
              \, \,\,\footnotesize{These are viruses that use tricks to fool the OS and}\\
              \, \,\,\footnotesize{antimalware software that everything is working fine.}\\
              \, \,\,\footnotesize{An example is an MBR infection virus that returns a }\\
              \, \,\,\footnotesize{clean MBR if the antimalware software checks the MBR}\\
              \, \,\,\footnotesize{but on boot it selects the infected MBR}\\
              -- \blu{Polymorphic Viruses}\\
              \, \,\,\footnotesize{These viruses change their own code on each infection}\\
              \, \,\,\footnotesize{This is similar to how an actual virus gets mutations}\\
              \, \,\,\footnotesize{\ora{The effect is that the signature of the virus changes!!}}\\
              \, \,\,\footnotesize{This makes it hard for antimalware software to detect it!}\\
              -- \blu{Encrypted Viruses}\\
              \, \,\,\footnotesize{Uses encryption to hide itself}\\
              \, \,\,\footnotesize{On each system, a new key is used in order to change appearance}\\
              \, \,\,\footnotesize{The \ora{Virus decryption routine} handles encryption and decryption}\\
              -- \blu{Logic Bombs}\\
              \, \,\,\footnotesize{This malware stays dormant until a certain time, }\\
              \, \,\,\footnotesize{or certain prerequisites are active.}\\
              -- \blu{Trojan Horse}\\
              \, \,\,\footnotesize{Software that appears kind, but carries a malicious \ora{payload behind the scenes}}\\
              \, \,\,\footnotesize{It is often used in combination with a keylogger to get critical information}\\
              -- \blu{Key stroke logger}\\
              \, \,\,\footnotesize{Logs every single keypress the victim does on a system}\\
              \, \,\,\footnotesize{Can also be in form of hardware!}\\
              -- \blu{Ransomware}\\
              \, \,\,\footnotesize{Malicious Code that encrypts the entire drive of a victim and }\\
              \, \,\,\footnotesize{prompts them to pay a ransom for their files}\\
              \, \,\,\footnotesize{If the user doesn't pay in time, the files are deleted.}\\
              \, \,\,\footnotesize{Examples: Petya, wannacry, cryptlocker}\\
              -- \blu{Worms}\\
              \, \,\,\footnotesize{Malware that propagates without any human interaction/intervention}\\
              \, \,\,\footnotesize{Often used in combination with other types.}\\
              -- \blu{Adware}\\
              \, \,\,\footnotesize{Often targeted at browsers. Launches and shows random adds and websites}\\
              \, \,\,\footnotesize{Were often bundled with legitimate software in the 2010s.}\\
              -- \blu{Spyware}\\
              \, \,\,\footnotesize{Reports \r{(all)} information on your system to an attacker}\\
              -- \blu{CryptoMiner}\\
              \, \,\,\footnotesize{Software that mines cryptocurrency for an attacker in the background}\\
              }\\
              \hline
              \mc{\r{Antivirus / Antimalware}\\
              \r{Most antimalware software uses multiple methods!}\\
              -- blu{Signature based}\\
              \, \,\,\footnotesize{Checks for specific signatures (hashes) from a regularly updated database}\\
              -- \blu{Heuristics based}\\
              \, \,\,\footnotesize{Checks for specific malicious behavior}\\
              -- \blu{Data integrity based}\\
              \, \,\,\footnotesize{Checks for suspicious file changes on the system}\\
              \, \,\,\footnotesize{Usually this is a change to an executable without updating it.}\\
              }\\
              \hline
              \mc{\r{Application Attacks}\\
              -- \blu{Buffer Overflow Attack}\\
              \, \,\,\footnotesize{Send more data than the server anticipated.}\\
              \, \,\,\footnotesize{Unless the server is modfied to drop the rest of the data,}\\
              \, \,\,\footnotesize{it is written as a block to memory. In that additional data,}\\
              \, \,\,\footnotesize{there might be instructions, commands that the attacker wants to run.}\\
              \, \,\,\footnotesize{input field : 5 char -> input = "hellonetcat 192.168.1.1 ..." (reverse shell)}\\
              -- \blu{Time of Check to Time of Use (TOCTTOU)}\\
              \, \,\,\footnotesize{If an admin revokes a users permission while said user is online,}\\
              \, \,\,\footnotesize{then the change won't happen until said user loggs off.}\\
              \, \,\,\footnotesize{This user can now simply choose to not log off and keep all permissions.}\\
              \, \,\,\footnotesize{A simple way to fix this is, is to force log off said user.}\\
              }\\
              \hline
            \end{tabular}
          \end{tabular}
          \hfill
    \end{table}
    \pagebreak
    \begin{table}[ht!]
          \begin{tabular}{ccc}
            \begin{tabular}[t]{|@{\hskip1pt}p{10.45cm}|}
              \hline
              \mc{
              -- \blu{Back Doors}\\
              \, \,\,\footnotesize{\r{an undocumented way of accessing a system}}\\
              \, \,\,\footnotesize{This can either be done \ora{via malicious software,}}\\
              \, \,\,\footnotesize{or \ora{intended backdoors by developers.}}\\
              \, \,\,\footnotesize{The developer backdoors are often only used in testing, and removed later}\\
              \, \,\,\footnotesize{The removing part can of course simply be forgotten :)}\\
              -- \blu{Escalation of Privilege}\\
              \, \,\,\footnotesize{Use exploits to elevate the priviliges on a victims system}\\
              \, \,\,\footnotesize{often done with a rootkit, but does not necessarily have to be root!}\\
              \, \,\,\footnotesize{For example, an elevation from user to moderator would also suffice.}\\
              -- \blu{Rootkit}\\
              \, \,\,\footnotesize{Uses various exploits to gain root access on a system.}\\
              }\\
              \hline
              \mc{\r{Web Application Attacks}\\
              -- \blu{Cross Site Scripting}\\
              \, \,\,\footnotesize{A browser opens scripts with the <script>script-name</script> notation}\\
              \, \,\,\footnotesize{Normally this is used to get input from users and to spawn other things}\\
              \, \,\,\footnotesize{However, these scripts can also be integrated into a link directly}\\
              \, \,\,\footnotesize{In this case an attacker might send you a link to a legitimate website,}\\
              \, \,\,\footnotesize{however, inside that link is also a script that gets loaded, }\\
              \, \,\,\footnotesize{is not from that website}\\
              \, \,\,\footnotesize{This means that the hacker could get your login information, }\\
              \, \,\,\footnotesize{if you enter that on this website}\\
              -- \blu{Cross Site Forgery (XSRF or CSRF)}\\
              \, \,\,\footnotesize{This sends a command to another website that is currently open.}\\
              \, \,\,\footnotesize{For example, if you are currently logged in into your bank account,}\\
              \, \,\,\footnotesize{and you have that website open, then clicking on such a malicious link}\\
              \, \,\,\footnotesize{might steal funds from you without you noticing!!}\\
              -- \blu{SQL Injection Attacks}\\
              \, \,\,\footnotesize{form username: password: in username you write:}\\
              \, \,\,\footnotesize{dashie';DELETE * from acc;}\\
              \, \,\,\footnotesize{you end the first command and then enter another, both commands:}\\
              \, \,\,\footnotesize{SELECT * FROM acc WHERE name = 'dashie'; DELETE * FROM acc;}\\
              \, \,\,\footnotesize{bye bye table :). Luckily this is easy to combat}\\
              \, \,\,\footnotesize{\r{Use prepared statements, limit privileges and perform input validation}}\\
              }\\
              \hline
              \mc{\r{Network Attacks}\\ 
              -- \blu{Denial of Service (DoS)}\\
              \, \,\,\footnotesize{Simple resource consumption attack that tries to use all bandwidth.}\\
              \, \,\,\footnotesize{This makes the server unable to provide the service to other users.}\\
              -- \blu{SYN Flooding}\\
              \, \,\,\footnotesize{Spam a server with SYN floods. Aka pretend to open 999999 TCP connections}\\
              \, \,\,\footnotesize{The server responds to these SYN requests, but you never answer,}\\
              \, \,\,\footnotesize{instead you send new SYN requests}\\
              \, \,\,\footnotesize{Without protection against this, the server can't open new connections!}\\
              -- \blu{Service Request Floods}\\
              \, \,\,\footnotesize{Similar to the SYN flood with the difference being actual open connections}\\
              \, \,\,\footnotesize{This means you actually respond and open the connection, only to then}\\
              \, \,\,\footnotesize{use this connection to spam application requests.}\\
              -- \blu{Application level DoS}\\
              \, \,\,\footnotesize{Exploits a vulnerability in a software to spam a server with bad packages.}\\
              \, \,\,\footnotesize{These bad packages might overload the server, preventing access.}\\
              -- \blu{Permanent DoS}\\
              \, \,\,\footnotesize{Crippling the hardware itself so the service can't be accessed.}\\
              \, \,\,\footnotesize{A good example if installing faulty firmware (Phlashing attack)}\\
              -- \blu{Botnets}\\
              \, \,\,\footnotesize{This is a network of bots(systems) that have been compromised.}\\
              \, \,\,\footnotesize{It is then used to perform a combined attack on a victim,}\\
              \, \,\,\footnotesize{or simply to add more systems to the botnet.}\\
              -- \blu{Distributed DoS}\\
              \, \,\,\footnotesize{A combined targeted DoS against a service}\\
              \, \,\,\footnotesize{Much harder to combat than single DoS, often done via Botnets}\\
              -- \blu{Man-in-the-Middle Attack}\\
              \, \,\,\footnotesize{Forcing traffic to your own system instead of router etc.}\\
              \, \,\,\footnotesize{This allows package manipulation, eavesdropping etc.}\\
              }\\
              \hline
            \end{tabular}
            \hspace{-0.1in}
            \begin{tabular}[t]{|@{\hskip1pt}p{10.45cm}|}
              \hline
              \mc{
              -- \blu{Man-in-the-Browser Attack}\\
              \, \,\,\footnotesize{Trojan installed via plugin, used to steal credentials,}\\
              \, \,\,\footnotesize{inject scripts and hijack authentication sessions}\\
              -- \blu{Eavesdropping}\\
              \, \,\,\footnotesize{Monitoring the traffic of a system to the net.}\\
              \, \,\,\footnotesize{Hard to detect as it has no symptoms.}\\
              -- \blu{Impersonation/Masquerading}\\
              \, \,\,\footnotesize{Pretending to be someone else in order to gain access to a system}\\
              \, \,\,\footnotesize{requires authentication credentials to be stolen!}\\
              -- \blu{Replay Attacks}\\
              \, \,\,\footnotesize{Gaining access to a session by sending copies of older messages.}\\
              \, \,\,\footnotesize{Possible after eavesdropping a session.}\\
              -- \blu{Modification Attacks}\\
              \, \,\,\footnotesize{Capture and then modify packages.}\\
              \, \,\,\footnotesize{It is essentially a man-in-the-middle attack.}\\
              }\\
              \hline
              \mc{\r{Session IDs and attacks}\\
              -- \gre{Session IDs}\\
              \, \,\,\footnotesize{Session IDs are a quick way of knowing who the server is dealing with}\\
              \, \,\,\footnotesize{It allows for storage of tokens, cookies etc.}\\
              \, \,\,\footnotesize{And it reduces the amount of times we have to enter passwords}\\
              -- \blu{Hijacking with Session ID Predicting}\\
              \, \,\,\footnotesize{The attacker tries to bruteforce guess the Session ID.}\\
              \, \,\,\footnotesize{Attacker sends thousands of guesses based on previous sessions.}\\
              -- \blu{Session Hijacking with Cross-Site Scripting}\\
              \, \,\,\footnotesize{Get the session ID with a cross-site script}\\
              -- \blu{Session Hijacking with Cross-Site Request Forgery}\\
              \, \,\,\footnotesize{Get the session with a malicious command}\\
              -- \blu{TCP/IP Hijacking}\\
              \, \,\,\footnotesize{During a TCP handshake, respond to quicker to the server than the victim.}\\
              \, \,\,\footnotesize{or force the victim to reset the handshake you start the session instead.}\\
              }\\
              \hline
              \mc{\r{Ethical Hacking and regular Hacking}\\
              -- \gre{Ethical Hacking}\\
              \, \,\,\footnotesize{System auditing, reporting and testing}\\
              \, \,\,\footnotesize{Vulnerability reporting}\\
              -- \gre{Hacking}\\
              \, \,\,\footnotesize{Security control compromise}\\
              \, \,\,\footnotesize{Produce behaviors in a software outside of original intent}\\
              \, \,\,\footnotesize{Exploitation of vulnerabilities}\\
              }\\
              \hline
              \mc{\r{Types of Hax0rz11!}\\
              -- \gre{Black Hat}\\
              \, \,\,\footnotesize{malicious/destructive hacker}\\
              -- \gre{Grey Hat}\\
              \, \,\,\footnotesize{possessing black hat skills, but focus on both offense and defense}\\
              -- \gre{White Hat}\\
              \, \,\,\footnotesize{Defense focused hacker}\\
              -- \gre{Cyber Terrorist}\\
              \, \,\,\footnotesize{Advancing Ideology with cyber attacks}\\
              }\\
              \hline
              \mc{\r{Email Security S/MIME}\\
                \ora{Multipurpose Internet Mail Extension MIME}\\
                -- \blu{Application Layer Protocol}\\
                \pic{220621-26}\\
              }\\
              \hline
            \end{tabular}
          \end{tabular}
          \hfill
    \end{table}
    \pagebreak
    \begin{table}[ht!]
          \begin{tabular}{ccc}
            \begin{tabular}[t]{|@{\hskip1pt}p{10.45cm}|}
              \hline
              \mc{
                -- \blu{S/MIME signing format 1}\\
                \pic{220621-27}\\
                -- \blu{S/MIME supports Multiple Signatures}\\
                \pic{220621-28}\\
                -- \blu{S/MIME Signing format 2}\\
                \pic{220621-29}\\
                \, \,\,\footnotesize{\gre{+ This format is not affected by encoding enforced by mail providers}}\\
                \, \,\,\footnotesize{\r{-- In order to read this format, the mail client must support S/MIME}}\\
                -- \blu{S/MIME Encrypted format:}\\
                \pic{220621-30}\\
                -- \blu{S/MIME Multi Encryption}\\
                \pic{220621-31}\\
                \footnotesize{Just like before, if you \ora{sign before you encrypt you have authenticity}, }\\
                \footnotesize{if you \ora{encrypt before you sign you have anonymity}.}\\
                \footnotesize{For more info, check the encryption and signing part of this doc.}\\
              }\\
              \hline
            \end{tabular}
            \hspace{-0.1in}
            \begin{tabular}[t]{|@{\hskip1pt}p{10.45cm}|}
              \hline
              \mc{\r{SPAM :)}\\
              \pic{220621-32}\\
              \pic{220621-33}\\
              -- \blu{False Negatives vs False Positives:}\\
              \pic{220621-34}\\
              \footnotesize{First doesn't cover all spam, but guarantees that no legitimate mail is missed}\\
              \footnotesize{Second blocks nearly all spam, but also blocks some legitimate mail! }\\
              }\\
              -- \blu{Sender Policy Framework (SPF)}\\
              \, \,\,\footnotesize{\r{TXT records in DNS}}\\
              \, \,\,\footnotesize{Mail client does DNS query for the mail sender.}\\
              \pic{220621-35}\\
              \hline
              \mc{\r{DKIM Domain Keys Identified Mail}\\
              \pic{220621-36}\\
              }\\
              \hline
              \mc{\r{Other Methods}\\
              -- \blu{Blacklists}\\
              \, \,\,\footnotesize{Filter based on a list of bad actors}\\
              \, \,\,\footnotesize{\gre{+ Very fast with no false positives}}\\
              \, \,\,\footnotesize{\r{-- List must be updated constantly, new senders not blocked}}\\
              -- \blu{Greylisting}\\
              \, \,\,\footnotesize{Check Sender IP, sender mail address and recipient address}\\
              \, \,\,\footnotesize{Test this against database. If it is unknown return failure to sender.}\\
              \, \,\,\footnotesize{\gre{+ blocks ALL spam}}\\
              \, \,\,\footnotesize{\r{-- also blocks legitimate mails that are unknown}}\\
              \, \,\,\footnotesize{\r{-- senders with multiple IP can't send mail}}\\
              \, \,\,\footnotesize{\r{-- Check can slow down mails to minutes or hours}}\\
              }\\
              \hline
            \end{tabular}
          \end{tabular}
          \hfill
    \end{table}
    \pagebreak
    \begin{table}[ht!]
          \begin{tabular}{ccc}
            \begin{tabular}[t]{|@{\hskip1pt}p{10.45cm}|}
              \hline
              \mc{\r{Authentication}\\
              -- \blu{A1 Plain Password}\\
              \, \,\,\footnotesize{Regular Passwords, easy to sniff if they aren't encrypted.}\\
              -- \blu{A2 One time Passwords}\\
              \, \,\,\footnotesize{Can't be sniffed and used, but Man in the middle attacks are possible}\\
              -- \blu{A3 Challange Response Protocol}\\
              \, \,\,\footnotesize{Server sends a random value as 'challenge'.}\\
              \, \,\,\footnotesize{The user then hashes his password and turns it into a MAC.}\\
              \, \,\,\footnotesize{The user then sends the MAC alongside his own Random number and his ID.}\\
              \, \,\,\footnotesize{The server calculates the MAC as well and compares it.}\\
              \, \,\,\footnotesize{The server has the password saved as well, aka it just hashes that}\\
              \, \,\,\footnotesize{This also means that a brute force is possible if the password table is stolen}\\
              \, \,\,\footnotesize{For better protection you send a signed MAC instead, user authenticity!}\\
              \pic{220622-2}\\
              -- \blu{A4 Anonymous Key Exchange}\\
              \, \,\,\footnotesize{Essentially encrypted password}\\
              -- \blu{A5 Zero Knowledge Passwords}\\
              \, \,\,\footnotesize{Challenge Responses with encrypted random numbers.}\\
              \, \,\,\footnotesize{Server also sends the MAC back to user to double check!}\\
              \pic{220622-4}\\
              -- \blu{A6 Certificate-based Server Authentication}\\
              \, \,\,\footnotesize{Server authenticates itself, then encrypted password used.}\\
              -- \blu{A7 Mutual Public Key Authentication}\\
              \, \,\,\footnotesize{You authenticate yourself and the server authenticates itself}\\
              \, \,\,\footnotesize{Recommended if you want a certain machine to have access.}\\
              \, \,\,\footnotesize{Often combined with encrypted passwords.}\\
              \pic{220622-1}\\
              }\\
              \hline
              \mc{\r{Kerberos}\\
              -- \blu{Realm == Domain}\\
              -- \blu{Key Distribution Center (KDC)}\\
              \, \,\,\footnotesize{The Server that handles key distribution and Kerberos tickets}\\
              -- \blu{Kerberos Tickets}\\
              \, \,\,\footnotesize{Mediates the communication between principles.}\\
              -- \blu{Principles (Users)}\\
              \, \,\,\footnotesize{Every user in the network is called a principle}\\
              -- \blu{Shared key = Principles master key}\\
              \, \,\,\footnotesize{Key that is assigned to a principle. Connects to the KDC}\\
              \pic{220622-3}\\
              \footnotesize{Shared secrets like Diffie-Hellman and time-stamped.}\\
              }\\
              \hline
            \end{tabular}
            \hspace{-0.1in}
            \begin{tabular}[t]{|@{\hskip1pt}p{10.45cm}|}
              \hline
              \mc{\vspace{-0.1in}\\
                \pic{220622-5}\\
              -- \blu{KDC Slaves}\\
              \, \,\,\footnotesize{In order to save the KDC from overload, we create some read-only}\\
              \, \,\,\footnotesize{KDC slave server, that can create tickets and authenticate users}\\
              \, \,\,\footnotesize{The only thing they can't do is create new passwords for users etc.}\\
              -- \blu{Inter-Realm Authentication}\\
              \pic{220622-6}
              }\\
              \hline
              \mc{\r{Federation}\\
              \footnotesize{The linking of 2 domains alongside their login systems.}\\
              \footnotesize{Example: kerberos inter-realm authentication.}\\
              \footnotesize{Formal Definition: \ora{a set of oranizations agreeing on a common}}\\
              \footnotesize{\ora{set of rules and standards.}}\\
              -- \blu{Shibboleth}\\
              \, \,\,\footnotesize{>> based on SAML, opensource, EDU-SWITCH, most used protocol}\\
              \, \,\,\footnotesize{>> used for WLAN, E-Learning, Libraries, etc}\\
              -- \blu{W3C - XML Signature/Encryption}\\
              \, \,\,\footnotesize{xmlns:ds = signature, xmlns:enc = encryption}\\
              -- \blu{Secure Assertion Markup Language (SAML)}\\
              \, \,\,\footnotesize{Parts: Assertions and Protocols, Bindings, Profiles, Metadata, Authentication-}\\
              \, \,\,\footnotesize{Context, Conformance Requirements, Security/Privacy, Glossary}\\
              \, \,\,\footnotesize{used with xml : xmlns:saml / xmlns:samlp}\\
              -- \blu{OAuth 2 Authorization Framework}\\
              \, \,\,\footnotesize{>> Used to connect services to accounts, ex. use Picasa resources from google}\\
              \, \,\,\footnotesize{>> based on https, security handled by TLS, used by big companies}\\
              \pic{220622-7}\\
              -- \blu{Open-ID Connect Authentication Layer}\\
              \, \,\,\footnotesize{>> on top of O-Auth 2.0, solves multiple IDs for multiple services}\\
              \, \,\,\footnotesize{>> used by many big companies, implementation up to service}\\
              \, \,\,\footnotesize{>> SwissID also uses this.}\\
              \pic{220622-8}\\
              }\\
              \hline
            \end{tabular}
          \end{tabular}
          \hfill
    \end{table}
    \pagebreak
    \begin{table}[ht!]
          \begin{tabular}{ccc}
            \begin{tabular}[t]{|@{\hskip1pt}p{10.45cm}|}
              \hline
              \mc{\r{Technical Vulnerability Management}\\
              \footnotesize{The idetification, classification, remedation and mitigation of vulnerabilities}\\
              -- \blu{Plan}\\
              \, \,\,\footnotesize{Identify Assets that could be exposed to risk}\\
              \, \,\,\footnotesize{Establish who or what will conduct the vulnerability tests}\\
              \, \,\,\footnotesize{Establish how the tests will flow into software development}\\
              -- \blu{Discover}\\
              \, \,\,\footnotesize{Monitor sources about known vulnerabilities that might apply to your system}\\
              -- \blu{Scan}\\
              \, \,\,\footnotesize{>> run automated tests based on your previous results}\\
              \, \,\,\footnotesize{>> hire professionals to do penetration testing}\\
              \, \,\,\footnotesize{>> Compare these results back-to-back}\\
              \, \,\,\footnotesize{\r{Note, certain tests might disrupt the system!}}\\
              \, \,\,\footnotesize{And, there might be false positives! Especially with the automated tests}\\
              -- \blu{Log \& Report}\\
              \, \,\,\footnotesize{Log results based on severity...}\\
              -- \blu{Remediate}\\
              \, \,\,\footnotesize{>> Ideally, every discovered vulnerability should be patched immediately}\\
              \, \,\,\footnotesize{>> However, performing a proper cost analysis as explained in the first 2 pages}\\
              \, \, \,\footnotesize{ is often preferred as not all vulnerabilities are worth fixing.}\\
              }\\
              \hline
              \mc{\r{Security evens and Security Incidents}\\
              -- \blu{Security Event}\\
              \, \,\,\footnotesize{A security event is just a simple logging mechanism for noteworthy changes:}\\
              \, \,\,\footnotesize{password changes, certificate renewal, new certificates, new root CAs etc}\\
              \, \,\,\footnotesize{These can be ignored if they were intended.}\\
              \, \,\,\footnotesize{They need to be configured, as you don't want your logs spammed with trash}\\
              -- \blu{Security Incident}\\
              \, \,\,\footnotesize{\r{a Security Incident is a detected breach!! Take action!!}}\\
              -- \blu{Scanning for Security Incidents}\\
              \, \,\,\footnotesize{>> Pattern matching: find a collection of events with similar cause}\\
              \, \,\,\footnotesize{>> Scan detection: attacks often begin with port scans etc!}\\
              \, \,\,\footnotesize{>> Threshold: if x events with y port scans, then ALERT}\\
              \, \,\,\footnotesize{>> Event correlation: assign the attack to a known attack variant}\\
              }\\
              \hline
              \mc{\r{PLAN-ASSESS-SIMPLIFY-DEPLOY}\\
              \ora{Best practises for security event management}\\
              -- \blu{PLAN}\\
              \, \,\,\footnotesize{>> What systems should be monitored?}\\
              \, \,\,\footnotesize{>> What events are important, which can be discarded?}\\
              \, \,\,\footnotesize{>> Where should we store the logs, and for how long do we store it?}\\
              \, \,\,\footnotesize{>> How will the performance of this system be monitored?}\\
              -- \blu{ASSESS}\\
              \, \,\,\footnotesize{Understand your priorities}\\
              -- \blu{SIMPLIFY}\\
              \, \,\,\footnotesize{Order the priorities and simplify them}\\
              -- \blu{DEPLOY}\\
              \, \,\,\footnotesize{Turn your priorities into action}\\
              }\\
              \hline
              \mc{\r{Cyber Attack Kill Chain}\\
              \pic{220622-9}\\
              \ora{How to deal with each step:}\\
              -- \blu{Reconnaissance}\\
              \, \,\,\footnotesize{>> use firewalls, whitelists, Network segmentation}\\
              \, \,\,\footnotesize{>> simple port scans can also be deterred by using high random ports}\\
              -- \blu{Weaponization}\\
              \, \,\,\footnotesize{>> patching a vulnerability before it is exploited....}\\
              }\\
              \hline
            \end{tabular}
            \hspace{-0.1in}
            \begin{tabular}[t]{|@{\hskip1pt}p{10.45cm}|}
              \hline
              \mc{
              -- \blu{Delivery}\\
              \, \,\,\footnotesize{>> Firewalls, antimalware software, good policies}\\
              \, \, \,\footnotesize{ Intrusion Prevention Software (IPS), Web Application Firewall (WAF)}\\
              -- \blu{Exploit}\\
              \, \,\,\footnotesize{>> Host-based Intrusion Detection System (HIDS), Backups, frequent patching}\\
              -- \blu{Installation}\\
              \, \,\,\footnotesize{>> antimalware software, HIDS, proper security events -> privilege escalation}\\
              -- \blu{Command \& Control}\\
              \, \,\,\footnotesize{Firewall, Network-based Intrusion Detection System (NIDS)}\\
              -- \blu{Actions}\\
              \, \,\,\footnotesize{you are fucked, restore the backup.}\\
              }\\
              \hline
              \mc{\r{Incident Management Process}\\
              -- \blu{PREPARE}\\
              \, \,\,\footnotesize{>> WHO-WHEN-WHAT}\\
              \, \,\,\footnotesize{>> develop Internal and External reporting procedures}\\
              \, \,\,\footnotesize{>> Guidelines for interacting with external parties (PLAYBOOKS)}\\
              \, \,\,\footnotesize{>> Define service by Incident Response Team (IRT)}\\
              \, \,\,\footnotesize{>> Establish and maintain accurate notification mechanism}\\
              \, \,\,\footnotesize{>> Establish which incidents take precedence}\\
              -- \blu{ANALYSE}\\
              \, \,\,\footnotesize{\ora{Magnitude? Severity? Urgency?}}\\
              -- \blu{CONTAINMENT}\\
              \, \,\,\footnotesize{Depending on the previous analysis, take action and contain}\\
              \, \,\,\footnotesize{the damage or the attacker.}\\
              -- \blu{RECOVERY}\\
              \, \,\,\footnotesize{Restore the system to regular state and patch it if necessary}\\
              }\\
              \hline
              \mc{\r{Digital Forensics}\\
              \pic{220622-10}\\
              }\\
              \hline
              \mc{\r{Threat Incident management best practises}\\
              \pic{220622-11}\\
              }\\
              \hline
              \mc{\r{OWASP TOP 10}}\\
              \gre{Exploit: How easy is it to exploit it?}\\
              \gre{Impact: If exploited, how harsh are the consequences?}\\
              \ora{The top 10 vulnerabilities, redefined each year}\\
              -- \blu{A1: Broken Access Control}\\
              \, \,\,\footnotesize{Attackers have access to accounts they shouldn't have}\\
              \, \,\,\footnotesize{Mitigations: Proper access control..., input validation}
              \, \,\,\footnotesize{occurred: 318k, Exploit: 6.9, Impact: 5.9}\\
              -- \blu{A2: Cryptographic Failures}\\
              \, \,\,\footnotesize{Missing or broken Ciphers like SSL, MD5, etc}\\
              \, \,\,\footnotesize{Problem right now: TLS 1.2 with ineffective configuration}\\
              \, \,\,\footnotesize{Mitigation: Use proper crypto tools like TLS 1.3, verify your hardware}\\
              \, \,\,\footnotesize{Make sure encryption is used correctly}\\
              \, \,\,\footnotesize{Occurred: 234k, Exploit: 7.3, Impact: 6.8}\\
              \hline
            \end{tabular}
          \end{tabular}
          \hfill
    \end{table}
    \pagebreak
    \begin{table}[ht!]
          \begin{tabular}{ccc}
            \begin{tabular}[t]{|@{\hskip1pt}p{10.45cm}|}
              \hline
              \mc{              
              -- \blu{A3: Injection}\\
              \, \,\,\footnotesize{Why is this still happening?}\\
              \, \,\,\footnotesize{Just do input validation or use prepared statements ffs!}\\
              \, \,\,\footnotesize{Output sanitation is also useful, simply encode the dangerous characters!}\\
              \pic{220622-12}\\
              \, \,\,\footnotesize{More info in the first 2 pages...}\\
              \, \,\,\footnotesize{Occurred: 274k, Exploit: 7.3, Impact: 7.2}\\
              -- \blu{A4: Insecure Design}\\
              \, \,\,\footnotesize{Use Threat models, better frameworks, lots of testing}\\
              \, \,\,\footnotesize{most costly problem to fix later, so do it now!}\\
              \, \,\,\footnotesize{Occurred: 262k, Exploit: 6.5, Impact: 6.8}\\
              -- \blu{A5: Security Misconfiguration}\\
              \, \,\,\footnotesize{Outdated and vulnerable software in production, default admin accs}\\
              \, \,\,\footnotesize{weak passwords, bad ports, leaking info with error messages}\\
              \, \,\,\footnotesize{Mitigation: Use software to detect misconfiguration and act accordignly.}\\
              \, \,\,\footnotesize{Harden Systems, deploy in testing first, update systems}\\
              \, \,\,\footnotesize{\r{Follow Recommendations!}}\\
              \, \,\,\footnotesize{Occurred: 208k, Exploit: 8.1, Impact: 6.6}\\
              -- \blu{A6: Vulnerable/Outdated Components}\\
              \, \,\,\footnotesize{Nothing to say other than don't use old versions for critical systems..}\\
              \, \,\,\footnotesize{Occurred: 132k, Exploit: 7.4, Impact: 6.5}\\
              -- \blu{A7: Identification/Authentication Failures}\\
              \, \,\,\footnotesize{These are login breaches.}\\
              \, \,\,\footnotesize{Mitigation: use better hash, use one time passwords}\\
              \, \,\,\footnotesize{Make sure tokens etc are correctly configured}\\
              \, \,\,\footnotesize{Make sure no part of the authentication is in plaint text}\\
              \, \,\,\footnotesize{Perhaps internal leak?}\\
              \, \,\,\footnotesize{Occurred: 262k, Exploit: 6.5, Impact: 6.8}\\
              -- \blu{A8: Software and Data Integrity Design}\\
              \, \,\,\footnotesize{Corruption or malicious change of files}\\
              \, \,\,\footnotesize{Mitigation: Raid 0, backups, validity checks, no plain text packages}\\
              \, \,\,\footnotesize{Occurred: 47.9k, Exploit: 6.9, Impact: 7.9}\\
              -- \blu{A9: Security Logging and Monitoring Failures}\\
              \, \,\,\footnotesize{Useless errors that only say: oops error sorry}\\
              \, \,\,\footnotesize{This could be an attacker but who knows, the error won't tell you}\\
              \, \,\,\footnotesize{In other words, the attacker is protected, because you aren't informed!}\\
              \, \,\,\footnotesize{Mitigation: Offer more information to the user, at the very least on request.}\\
              \, \,\,\footnotesize{Occurred: 53.6k, Exploit: 6.9, Impact: 5.0}\\
              -- \blu{A10: Server Side Request Forgery}\\
              \, \,\,\footnotesize{Server loads URL without checking, making it possible to load malicious stuff}\\
              \, \,\,\footnotesize{Including commands to execute on website, like banks, hence forgery}\\
              \, \,\,\footnotesize{Mitigations: deny by default firewall, Sanitize URLs, whitelist for URLs}\\
              \, \,\,\footnotesize{disable https redirections, DON'T use Blacklists!}\\
              \, \,\,\footnotesize{Occurred: 9.5k, Exploit: 8.2, Impact: 6.7}\\
              }\\
              \hline
              \\\\\\\\
              \\\\\\\\
              \\\\\\\\
              \\\\\\\\
              \\\\\\\\
              \hline
            \end{tabular}
            \hspace{-0.1in}
          \end{tabular}
          \hfill
    \end{table}
\end{document}
